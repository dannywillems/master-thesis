\chapter{Introduction}

La programmation modulaire est un principe de développement visant à séparer une
application en composants plus petits appelés \textit{modules}. Le langage de
programmation OCaml contient un langage de modules qui permet aux développeurs
d'utiliser la programmation modulaire. Dans ce langage de modules, un module est
un ensemble de types et de valeurs, les types des valeurs pouvant dépendre des
types définis dans le même module.
OCaml étant un langage fortement typé, les modules possèdent également
un type, appelé dans ce cas \textit{signature}.

Bien que les modules soient bien intégrés dans OCaml, une
distinction est faite entre le langage de base, contenant les types dits
\og primitifs \fg{}
comme les entiers, les chaines de caractères ou les fonctions, et le
langage de module. En particulier, le terme \textit{foncteur} est employé à la
place de \textit{fonction} pour parler des valeurs prenant un module en
paramètres et en retournant un autre. De plus, il n'est pas possible de définir
des valeurs prenant un module et un type primitif et retournant un module (ou un
type primitif).

D'un autre côté, dans les types primitifs d'OCaml se trouvent les
\textit{enregistrements}. Ces derniers sont des ensembles de couples
\textit{(label, valeur)}, et ressemblent en quelques points aux modules.

Ce mémoire vise à fournir un calcul typé, construit à partir du $\lambda$-calcul
simplement typé, dans lequel le langage de modules est confondu avec le langage
de base grâce aux enregistrements. Une preuve de la sûreté de ce calcul ainsi
qu'un interpréteur avec un algorithme de sous-typage et d'inférence de type
sont fournis.

Les chapitres sont organisés afin de comprendre la construction d'un tel calcul
à partir du plus simples des calculs, le $\lambda$-calcul.

Dans le chapitre 1, nous présentons \textit{le $\lambda$-calcul non typé}, un calcul
minimal qui contient des termes pour les variables, pour les abstractions (afin
de représenter des fonctions) et des applications (afin de représenter
l'application d'une fonction à un paramètre). Nous discuterons également de la
sémantique que nous attribuons à ce calcul.

Dans le chapitre 2, nous introduisons la notion de type et nous l'appliquons au
$\lambda$-calcul, ce qui nous donnera \textit{le $\lambda$-calcul simplement typé}. Nous
discuterons de la notion de \textit{sureté du typage} à travers \textit{les théorèmes
de préservation et de progression} que nous démontrons pour ce calcul typé.

Dans le chapitre 3, nous parlerons de la notion de type récursif et nous
étudierons \textit{le $\lambda$-calcul simplement typé avec type récursif}.

Dans les chapitres 4, 5 et 6, nous enrichissons le $\lambda$-calcul simplement
typé avec la notion de polymorphisme qui permet d'attribuer plusieurs types à un
terme. Le chapitre 4 se concentre sur \textit{le polymorphisme avec sous-typage},
illustré avec les enregistrements. Dans le chapitre 5, nous parlons de
\textit{polymorphisme paramétré} qui, combiné au $\lambda$-calcul simplement typé, forme
le calcul appelé \textit{System F}.
Le chapitre 6 se charge de combiner ces deux notions de polymorphisme dans un
calcul appelé \textit{System $F_{<:}$}.

Pour finir, dans le chapitre 7, nous complétons les enregistrements définis
dans le chapitre 4 avec les \textit{types chemins dépendants} qui offre la possibilité
d'ajouter des types dans les enregistrements. Nous montrerons comment ce dernier
calcul permet de résoudre le problème posé.

Chaque calcul possède une implémentation écrite en OCaml dont les choix et les
difficultés d'implémentation ainsi qu'un lien vers celle-ci sont donnés dans une
section dans le chapitre correspondant au calcul.

%Le sujet de ce mémoire est un sujet de recherche proposé à l'INRIA aux étudiants
%du MPRI.
