\chapter*{Introduction}

La programmation modulaire est un principe de développement consistant à séparer une
application en composants plus petits appelés \textit{modules}. Le langage de
programmation OCaml contient un langage de modules qui permet aux développeurs
d'utiliser la programmation modulaire. Dans ce langage de module, un module est
un ensemble de types et de valeurs, les types des valeurs pouvant dépendre des
types définis dans le même module.
OCaml étant un langage fortement typé, les modules possèdent également
un type, appelé dans ce cas \textit{signature}.

%% TODO exemple d'un module OCaml et de sa signature.

Bien que les modules soient bien intégrés dans OCaml, une
distinction est faite entre le langage de base, contenant les types dits \og de
bases \fg comme les entiers, les chaînes de caractères ou les fonctions, et le
langage de module. En particulier, le terme \textit{foncteur} est employé à la
place de \textit{fonction} pour parler des fonctions prenant un module en
paramètres et en retournant un autre. De plus, il n'est pas possible de définir
des fonctions prenant un module et un type de base et retournant un module (ou un
type de base).

% Pour changer le nom du début de la légende d'un code.
\renewcommand\listingscaption{Code OCaml}

\begin{listing}
  \inputminted{OCaml}{codes/intro_point2D.ml}
  \caption{Exemple d'un module nommé Point2D contenant un type t pour
    réprésenter un point par ses coordonnées cartésiennes dans un enregistrement
    et d'une fonction add retournant un point dont les coordonnées sont
    la somme de deux points donnés en paramètres.}
\end{listing}

\begin{listing}
  \caption{MakePoint2D est un foncteur qui permet de rendre polymorphique notre
    module Point2D.}
  \inputminted{OCaml}{codes/intro_point2D_polymorphic.ml}
\end{listing}

\begin{listing}
  \caption{Application de notre foncteur au module des entiers.}
  \inputminted{OCaml}{codes/intro_point2D_polymorphic_app.ml}
\end{listing}


D'un autre côté, dans les types de bases d'OCaml se trouvent les
\textit{enregistrements}. Ces derniers sont des ensembles de couples
\textit{(label, valeur)}, et ressemblent aux modules. Cependant, la différence
majeure entre eux se situent dans la possibilité de définir des types dans un
module.

Ce mémoire vise à définir dans un premier temps un calcul typé dans lequel le
langage de modules est confondu avec le langage de base grâce aux
enregistrements et dans un second temps, de fournir une implémentation en OCaml
des algorithmes de typage et de sous-typage.

%Une preuve de la sûreté de ce calcul ainsi
%qu'un interpréteur avec un algorithme de sous-typage et d'inférence de type est
%fourni.

Les chapitres sont organisés afin de comprendre la construction d'un tel calcul
à partir du plus simple des calculs, le $\lambda$-calcul.

Dans le chapitre 1, nous présentons \textit{le $\lambda$-calcul non typé}, un calcul
minimal qui contient des termes pour les variables, pour les abstractions (afin
de représenter des fonctions) et des applications (afin de représenter
l'application d'une fonction à un paramètre). Nous discuterons également de la
sémantique que nous attribuons à ce calcul. 

Dans le chapitre 2, nous introduisons la notion de type et nous l'appliquons au
$\lambda$-calcul, ce qui nous donnera \textit{le $\lambda$-calcul simplement typé}. Nous
discuterons de la notion de \textit{sureté du typage} à travers \textit{les théorèmes
de préservation et de progression} que nous démontrons pour ce calcul typé.

Dans les chapitres 3, 4 et 5, nous enrichissons le $\lambda$-calcul simplement
typé avec la notion de polymorphisme qui permet d'attribuer plusieurs types à un
terme. Le chapitre 3 se concentre sur \textit{le polymorphisme avec sous-typage},
illustré avec les enregistrements. Nous parlerons aussi de l'implémentation de
ce calcul et nous montrerons qu'un travail est nécessaire pour passer de la
théorie à l'implémentation.
Dans le chapitre 4, nous parlons de
\textit{polymorphisme paramétré} qui, combiné au $\lambda$-calcul simplement typé, forme
le calcul appelé \textit{System F}.
Le chapitre 5 se charge de combiner ces deux notions de polymorphismes dans un
calcul appelé \textit{System $F_{<:}$}.
Une preuve des théorèmes de préservation et de progression seront donnés pour
les calculs définis dans les chapitres 3 et 4.

Dans le chapitre 6, nous parlerons de la notion de type récursif et nous
étudierons \textit{le $\lambda$-calcul simplement typé avec type récursif}.

Pour finir, dans le chapitre 7, nous complétons les enregistrements définis
dans le chapitre 3 avec les \textit{types chemins dépendants} qui offre la possibilité
d'ajouter des types dans les enregistrements. Ce dernier chapitre comportera en
plus des types chemins dépendants, chaque notion étudiée précédemment,
c'est-à-dire le $\lambda$-calcul simplement typé, les types récursifs, les
enregistrements, le polymorphisme par sous-typage et le polymorphisme paramétré.

La principale difficulté de ce travail se trouve dans l'étude des types
chemins dépendants, sujet de recherche récent et moins bien compris que les
calculs comme \textit{System $F$} ou \textit{System $F_{<:}$}.

% Le sujet de ce mémoire est un sujet de recherche proposé à l'INRIA aux étudiants
%du MPRI.