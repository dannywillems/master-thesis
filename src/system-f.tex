\chapter{Système F}
\label{chapter:system-f}

Dans ce chapitre, nous allons introduire une autre de notion de polymorphisme
appelée le \textit{polymorphisme paramétré}\footnote{Plus d'informations peuvent
être trouvées dans \cite{tapl-system-f}}

Nous avons vu dans le chapitre précédent que le polymorphisme par sous-typage
nous permet de rendre notre relation de typage plus flexible en donnant la
possibilité d'attribuer plusieurs types à un terme grâce à la règle T-SUB. Cette
méthode nous permet alors d'éviter
d'écrire plusieurs fonctions qui ont le même corps mais qui différent uniquement
par le type du paramètre.

Supposons que nous ayons $\naturel$ et $\real$ comme types dans notre langage et
prenons l'exemple de la fonction identité sur les réels
\begin{equation*}
  id_{1} = \lambdaExpr{x : \real}{x}
\end{equation*}

Le polymorphisme avec sous-typage nous permet de passer en paramètre un
naturel
car $\naturel <: \real$. Dans tous les cas, la fonction $id_{1}$ étant de type
$\real \rightarrow \real$, le type de la valeur de retour sera $\real$, et non
$\naturel$ si nous passons un naturel à la fonction.
Dans le cas des enregistrements, cela implique que nous perdons des
champs de l'enregistrement passé en paramètre.

De plus, il n'est pas autorisé de passer un
enregistrement ayant un champ $x$ de type $\naturel$ car $\real$ n'est pas
un sous-type de $\left\{ x : \naturel \right\}$. Nous devons alors définir une
nouvelle fonction pour l'identité:
\begin{equation*}
  id_{2} = \lambdaExpr{x : \left\{ x : \naturel \right\}}{x}
\end{equation*}

Et nous pouvons continuer de la sorte avec les enregistrements qui ont
uniquement le champ $y$ ou le champ $z$ et ainsi de suite.

Le polymorphisme paramétré résout ce problème en définissant de nouveaux termes
et de nouvelles règles de typage et d'évaluation au $\lambda$-calcul simplement
typé. Le calcul qui en résulte est appelé \textit{Système F}.

\section{Syntaxe}

La syntaxe de \textit{Système F} est très proche de celle du $\lambda$-calcul
simplement typé : nous ajoutons un terme qui permet de créer une fonction
prenant un type et retournant un terme (l'équivalent de l'abstraction sur les
termes) ainsi qu'un terme qui permet d'appliquer un terme à un type.

Du côté des types, nous ajoutons des variables de type qui nous servent
dans les abstractions de type. Nous supposons, comme pour les variables de
terme, qu'il existe un nombre dénombrable de variables de type.
Nous ajoutons également un type pour les abstractions de type, appelé \textit{type universel}.

Arbitrairement, nous ajoutons aussi les abstractions de type comme valeurs.

\begin{minipage}{0.45\textwidth}
  \begin{align*}
    t ::= & \, & \text{terme} \\
          & \; x & \text{var} \\
          & \; t \, t & \text{app} \\
          & \; \lambdaExpr{x : T}{t} & \text{abs} \\
          & \; \lambdaExprType{X}{t} & \text{type abs} \\
          & \; t[T] & \text{type app}
  \end{align*}
\end{minipage}
\begin{minipage}{0.45\textwidth}
  \begin{align*}
    T ::= & \, & \text{type} \\
          & \; X & \text{type var} \\
          & \; T \rightarrow T & \text{fonction} \\
          & \; \forall X . \, T & \text{universel}
  \end{align*}
\end{minipage}
\begin{align*}
  v ::= & \, & \text{valeur} \\
        & \; \lambdaExpr{x : T}{t} & \text{abs} \\
        & \; \lambdaExprType{X}{t} & \text{type abs}
\end{align*}

Par exemple, nous pouvons définir une fonction qui prend un type en paramètre et
qui renvoie la fonction identité pour ce type:

\begin{equation*}
  id_{poly} = \lambdaExprType{X}{(\lambdaExpr{x : X}{x})}
\end{equation*}

Nous étendons également de manière naturelle la notion de substitution aux
nouveaux termes $t[T]$ et $\lambdaExprType{X}{t}$.

\begin{align*}
  [x := s] \, (t[T]) & = ([x := s]t)[T] \\
  [x := s] \, (\lambdaExprType{X}{t}) & = \lambdaExprType{X}([x := s]t)
\end{align*}

Nous définissons également une fonction de substitution de variable de type pour
un type donné, notée $[X \rightarrow S]$. Comme les types peuvent apparaître
aussi bien dans les termes que dans les types, cette fonction est définie sur
ces deux ensembles.

\begin{definition}
  Soit $X$ une variable de type et $S$ un type. On définit la fonction de
  substitution de $X$ par $S$, noté $[X \rightarrow S]$, sur les termes et types
  de la manière suivante :

  \begin{align*}
    [X \rightarrow S]x & = x \\
    [X \rightarrow S]\lambdaExpr{x : T} t & = \lambdaExpr{x : [X \rightarrow S]T}{[X \rightarrow S]t} \\
    [X \rightarrow S](t_{1} \; t_{2}) & = ([X \rightarrow S]t_{1}) \; ([X \rightarrow S]t_{2}) \\
    [X \rightarrow S](\lambdaExprType{Y} t) & = \lambdaExprType{Y}([X \rightarrow S] t) \\
    [X \rightarrow S](t[T]) & = ([X \rightarrow S] t)[[X \rightarrow S] T] \\
    \\
    [X \rightarrow S]X & = S \\
    [X \rightarrow S]Y & = Y \\
    [X \rightarrow S](T_{1} \rightarrow T_{2}) & = ([X \rightarrow S]T_{1}) \rightarrow ([X \rightarrow S] T_{2}) \\
    [X \rightarrow S](\forall Y .\, T) & = \forall Y . \, [X \rightarrow S]T
  \end{align*}
\end{definition}
\section{Sémantique}

Nous ajoutons également des règles d'évaluation pour les nouveaux termes
définis. Celles-ci sont très semblables à (E-APPABS) et (E-APP).

\begin{mathpar}
  \inferrule*[Left=(E-T-APP)]
  {t \rightarrow t'}
  {t[T] \rightarrow t'[T]}
  \and
  \inferrule*[Right=(E-T-ABS)]
  {(\lambdaExprType{X}{t})[T] \rightarrow [X \rightarrow T]t}
  {}
\end{mathpar}

(E-T-APP) est l'équivalent de (E-APP1) pour les applications de types tandis que
(E-T-ABS) nous dit comment les abstractions de types sont évaluées.

Par exemple, nous pouvons utiliser $id_{poly}$ et les types $\real$ et
$\naturel$ pour obtenir la fonction identité sur les réels et les naturels.

\begin{mathpar}
  \inferrule
  {id_{poly}[\naturel] \rightarrow \lambdaExpr{x : \naturel} x}
  {}
  \and
  \inferrule
  {id_{poly}[\real] \rightarrow \lambdaExpr{x : \real} x}
  {}
\end{mathpar}

\section{Contexte de typage}

Avant de donner les règles de typage, il est important de remarquer que nous
devons également changer notre définition du contexte. En effet, le but initial
du contexte est de contenir les variables libres d'un terme. Cependant, des
variables de type sont également définies et peuvent être également libres dans des
termes comme dans $\lambdaExpr{z : \left\{ x : X ; y : X \right\}}{z}$.
Cela nous amène à ajouter les variables dans le contexte afin de garder une
trace de leur définition. Le contexte de typage
est donc défini comme une liste contenant des couples $(x : T)$ et des variables
de type $X$.

Nous supposons que si une variable $X$ est dans le contexte, alors elle ne peut
pas apparaître à la gauche de sa définition. Par exemple, le contexte $x : X, X$
n'est pas valable, mais $X, x : X$ l'est.

\section{Règles de typage}

Enfin, nous ajoutons des règles de typage pour les nouveaux termes
$\lambdaExprType{X}{T}$ et $t[T]$.

\begin{mathpar}
  \inferrule* [Left=(T-T-ABS)]
  {\Gamma, X \vdash t : T}
  {\Gamma \vdash \lambdaExprType{X}{t} : \forall X . \, T}
  \and
  \inferrule* [Right=(T-T-APP)]
  {\Gamma \vdash t_{1} : \forall X . \, T}
  {\Gamma \vdash t_{1}[T'] : [X \rightarrow T']T}
\end{mathpar}

Ces règles nous permettent de typer nos termes $id_{poly}$, $id_{poly}[\real]$ et
$id_{poly}[\naturel]$ qui sont respectivement de types $\forall X . \, X
\rightarrow X$, $\real \rightarrow \real$ et $\naturel \rightarrow \naturel$.

\section{Sûreté}

Les théorèmes de préservation et de progression peuvent également être démontrés
pour Système F. Comme nous avons pu le remarquer, Système F est une extension
relativement simple du $\lambda$-calcul simplement typé. Les nouvelles règles de typage
et d'évaluation ont de l'influence uniquement sur les nouveaux termes et les
nouveaux types. Les preuves des théorèmes sont donc très semblables. Comme pour
le chapitre précédent, les preuves seront moins détaillées.

\subsection*{Progression}

Nous commençons par ajouter de nouveaux cas dans le lemme d'inversion des règles
de typage \ref{lemma:simply-typed-lambda-calculus-inversion}, les cas du
$\lambda$-calcul simplement typé restant vrais.

\begin{lemma} [d'inversion des règles de typage]
  \begin{enumerate}
    \item Si $\Gamma \vdash \lambdaExprType{X}{t} : \forall X . \, T$, alors
      $\Gamma, X \vdash t : T$.
    \item Si $\Gamma \vdash t[T_{1}] : T_{2}$, alors $\Gamma \vdash t : \forall X .
      \, T$ où $T_{2} = [X \rightarrow T_{1}] T$.
  \end{enumerate}
\end{lemma}
  
\begin{proof}
  Direct vu les règles de typage.
\end{proof}

Nous montrons également un lemme des formes canoniques comme nous l'avons fait
pour le $\lambda$-calcul avec sous-typage.

\begin{lemma} [des formes canoniques]
  \begin{enumerate}
  \item Si $\Gamma \vdash v : \forall X . \, T$, alors $v =
    \lambdaExprType{X}{t}$ et $\Gamma, X \vdash t : T$.
  \item Si $\Gamma \vdash v : T_{1} \rightarrow T_{2}$, alors $v =
    \lambdaExpr{x : T_{1}}{t}$ et $\Gamma, x : T_{1} \vdash t : T_{2}$.
  \end{enumerate}
\end{lemma}

\begin{proof}
  Direct vu les règles de typage.
\end{proof}

\begin{theorem} [de progression]
  Soit $t$ un terme bien typé sans variable libre. Alors, soit $t$ est une
  valeur, soit il existe $t'$ tel que $t \rightarrow t'$.
\end{theorem}

\begin{proof}
  Les seuls nouveaux cas sont $t = \lambdaExprType{X}{t_{1}}$ et $t =
  t_{1}[T_{2}]$.

  Dans le premier cas, $t$ est une valeur donc le résultat est direct.

  Dans le second cas, par le lemme d'inversion, nous avons $\vdash t_{1} :
  \forall X . \, T_{1}$. L'hypothèse de récurrence nous dit que soit $t_{1}$ est
  une valeur, soit il existe $t'_{1}$ tel que $t_{1} \rightarrow t'_{1}$.

  \begin{itemize}
  \item Si $t_{1}$ est une valeur, par le lemme des formes canoniques, $t_{1} =
    \lambdaExprType{X}{t}$ et (E-T-ABS) s'applique.
  \item Si $t_{1}$ se réduit en $t'_{1}$, alors (E-T-APP) s'applique.
  \end{itemize}
\end{proof}

\subsection*{Préservation}

%Comme premier lemme, nous renforçons
%\ref{thm:simply-typed-lambda-calculus-preservation-substitution} en ajoutant un
%contexte $\Delta$.
%
%\begin{lemma}
%  \label{thm:system-f-preservation-substitution-term}.
%  Soit $\Gamma, x : S, \Delta \vdash t : T$ et $\Gamma \vdash s : S$.
%
%  Alors $\Gamma, \Delta \vdash [x \rightarrow s] t : T$
%\end{lemma}
%
%\begin{proof}
%  
%\end{proof}

\begin{lemma} [de substitution de variable de type]
  \label{thm:system-f-preservation-substitution-type}
  Soit $S$ un type.

  Si

  \begin{equation*}
    \Gamma, X, \Delta \vdash t : T
  \end{equation*}
  alors
  \begin{equation*}
    \Gamma, [X \rightarrow S]\Delta \vdash [X \rightarrow S]t : [X \rightarrow S] T 
  \end{equation*}
  où $[X \rightarrow S]\Delta$ signifie que nous remplaçons toutes les
  occurences de $X$ par $S$ dans le contexte $\Delta$.
\end{lemma}

\begin{proof}
  \label{proof:system-f-preservation-substitution-type}
  La preuve se réalise par récurrence sur l'arbre de dérivation. Nous ne
  démontrons que le cas (T-ABS) où $\Delta$ est nécessaire.

%  \begin{itemize}
   (T-ABS). Nous avons
   \begin{equation*}
     \Gamma, X, \Delta \vdash \lambdaExpr{x : T_{1}}t_{2} : T_{1} \rightarrow T_{2}
   \end{equation*}
   Par le lemme d'inversion, nous obtenons
   \begin{equation*}
     \Gamma, X, \Delta, x : T_{1} \vdash t_{2} : T_{2}
   \end{equation*}
   Par hypothèse de récurrence,
   \begin{equation*}
     \Gamma, [X \rightarrow S]\Delta, x : [X \rightarrow S]T_{1} \vdash [X
     \rightarrow S]t_{2} : [X \rightarrow S]T_{2}
   \end{equation*}
   En utilisant (T-ABS), nous avons
   \begin{equation*}
     \Gamma, [X \rightarrow S]\Delta \vdash \lambdaExpr{x : [X \rightarrow S]T_{1}}[X
     \rightarrow S]t_{2} : [X \rightarrow S]T_{2}
   \end{equation*}
   En utilisant la définition de la fonction de substitution, nous concluons
   \begin{equation*}
     \Gamma, [X \rightarrow S]\Delta \vdash [X \rightarrow S]\lambdaExpr{x :
       T_{1}} t_{2} : [X \rightarrow S]T_{2}
   \end{equation*}
%  \item (T-T-APP). Nous avons
%    \begin{equation*}
%      \Gamma, X, \Delta \vdash t[T'] : [Y \rightarrow T']T
%    \end{equation*}
%    Par le lemme d'inversion, nous avons
%    \begin{equation*}
%      \Gamma, X, \Delta \vdash t : \forall Y . T
%    \end{equation*}
%    Par hypothèse de récurrence,
%     \begin{align*}
%       & & \Gamma, [X \rightarrow S]\Delta \vdash [X \rightarrow S]t : [X \rightarrow S] \forall Y . T \\
%       & = & \Gamma, [X \rightarrow S]\Delta \vdash [X \rightarrow S]t : \forall Y . [X \rightarrow S]T
%    \end{align*}
%    En utilisant (T-T-APP), nous obtenons
%     \begin{equation*}
%        \Gamma, [X \rightarrow S]\Delta \vdash [X \rightarrow S]t[T'] : [Y \rightarrow T'][X \rightarrow S]T
%    \end{equation*}
%\end{itemize}
   

\end{proof}

\begin{theorem} [de préservation du typage]
  \label{thm:system-f-preservation}
  Soit $\Gamma \vdash t : T$ et $t \rightarrow t'$. Alors $\Gamma \vdash t' :
  T$.
\end{theorem}

\begin{proof}
  \label{thm:system-f-preservation-proof}. Nous procédons par induction sur
  l'arbre de dérivation du jugement de typage $\Gamma \vdash t : T$. Par rapport
  à la preuve du $\lambda$-calcul simplement typé, il nous
  suffit d'ajouter des cas pour (T-T-ABS) et (T-T-APP).

  \begin{itemize}
  \item (T-T-ABS). Ce cas n'est pas possible car il n'existe pas de règle d'évaluation.
  \item (T-T-APP). Nous avons $t = t_{1}[T_{1}]$. Par le lemme d'inversion, nous
    avons $\Gamma \vdash t_{1} : \forall X . \, T_{12}$ et $T = [X \rightarrow
  T_{1}] T_{12}$. Les deux règles d'évaluation possibles sont (E-T-ABS) et
  (E-T-APP).
    \begin{itemize}
    \item (E-T-APP). Nous utilisons l'hypothèse de récurrence et (E-T-APP).
    \item (E-T-ABS). Alors, $t_{1} = \lambdaExprType{X}{t_{12}}$ avec $\Gamma, X
      \vdash t_{12} : T_{12}$ et $t'_{1} = [X \rightarrow T_{1}] t_{12}$. Par
     \ref{thm:system-f-preservation-substitution-type}, nous avons $\Gamma \vdash
     [X \rightarrow T_{1}] t_{12} : [X \rightarrow T_{1}] T_{12}$. Nous concluons
     en se souvenant que $[X \rightarrow T_{1}]T_{12} = T$.
    \end{itemize}
  \end{itemize}
\end{proof}
