\chapter*{Remerciements}

En premier lieu, je remercie François Pottier pour avoir accepté de me suivre
pour ce mémoire et de m'avoir permis d'intégrer l'INRIA Paris pendant toute la
durée de celui-ci. Sans nos discussions, ses disponibilités, ses conseils et ses
remarques, ce travail n'aurait pas pu être réalisé.

Je remercie également Christophe Troestler pour m'avoir aidé à choisir mon sujet
de mémoire ainsi que les conseils quant à la rédaction de ce document.

Je remercie chaque membre de l'équipe Gallium de
l'INRIA avec qui j'ai discuté et qui m'ont permis de découvrir de nouveaux
domaines dans la recherche informatique, plus ou moins éloigné du sujet de mon mémoire.

Je remercie également Vincent Balat qui m'a permis de découvrir lors de mon
stage différents chercheurs dans le domaine de la recherche dans les langages de
programmation. Sans ses conseils et son aide, je n'aurais eu l'idée de contacter
les membres de l'équipe Gallium afin d'obtenir un sujet.

Ensuite, je tiens à remercier Paul-André Melliès pour, dans un premier temps,
m'avoir invité à suivre son cours de lambda-calcul et catégories à l'ENS Ulm qui
m'a donné l'envie d'explorer plus en profondeur le lien entre l'informatique
théorique et les catégories\footnote{Malheureusement pas abordées dans ce sujet.}, et,
dans un second temps, pour sa disponibilité et ses conseils lors de la recherche
de mon sujet de mémoire.

Entre autres, je remercie chaque personne ayant porté ou portant de l'intérêt à mon
travail, ce qui me pousse à continuer d'explorer ce sujet par la suite.

Je remercie aussi les chercheurs et développeurs travaillant sur
DOT\footnote{Dependent Object Type}, travail de recherche sur lequel mon travail
est basé, pour leurs disponibilités et leurs réponses à mes questions. En
particulier, je remercie Nada Amin, dont la thèse est consacrée à DOT, pour ses
réponses à mes emails.

Pour finir, je remercie chaque professeur m'ayant suivi pendant ces années
d'études.