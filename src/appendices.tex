\appendix
\chapter{Preuve par induction sur les termes et les types}

Les termes ainsi que les types d'un langage sont définis de manière inductive.
Par exemple, pour le $\lambda$-calcul simplement typé, la grammaire des termes
est définie par
\begin{align*}
  t ::= & \, & \text{terme} \\
        & \; x & \text{var} \\
        & \; t \, t & \text{app} \\
        & \; \lambdaExpr{x : T}{t} & \text{abs}
\end{align*}

et la grammaire des types, en supposant que nous avons uniquement \verb|Bool|
(pour les booléens) comme type de base, est définie par
\begin{align*}
  T ::= & \, & \text{types} \\
        & \; Bool & \text{type des booléens} \\
        & \; T \rightarrow T & \text{type des fonctions}
\end{align*}

Ces définitions récursives sur les termes et les types nous permettent de
définir récursivement des fonctions agissant sur les termes et les types. Par
exemple, nous pouvons définir de manière inductive une fonction
\verb|size| sur les termes et les types de la manière suivante.

\begin{align*}
  size(x) & = 1 \\
  size(t_{1} t_{2}) & = size(t_{1}) + size(t_{2}) \\
  size(\lambdaExpr{x : T}{t}) & = size(t) + 1
\end{align*}

\begin{align*}
  size(Bool) & = 1 \\
  size(T_{1} \rightarrow T_{2}) & = size(T_{1}) + size(T_{2}) \\
\end{align*}

Si nous nous représentons les termes et les types en forme d'arbre, la
définition se résume à la définition du nombre de noeuds de l'arbre. Cette
représentation et la définition de fonctions comme \verb|size| nous permettent de
raisonner par induction sur le nombre de noeuds de l'arbre en utilisant l'induction sur
les naturels, comme le montre la preuve d'unicité de type pour le
$\lambda$-calcul simplement typé.

Nous supposons pour la plupart des grammaires que nous disposons d'une telle
fonction qui permette de raisonner inductivement sur la structure des
programmes ou des types, la fonction \verb|size| étant souvent facile à définir.

%Certaines preuves nécessitent une induction sur deux paramètres naturels comme celles
%du lemme d'affaiblissement et du lemme de permutation pour le $\lambda$-calcul
%simplement typé.

%En effet, pour le lemme de permutation, pour le cas des abstractions, l'argument
%complet est:
%
%\og Par hypothèse et le lemme d'inversion, $\Gamma \vdash \lambdaExpr{x : T_{1}}
%t' : T_{1} \rightarrow T_{2}$ et $\Gamma, x : T_{1} \vdash t' : T_{2}$. Par
%hypothèse de récurrence, $\Delta, x : T_{1} \vdash t' : T_{2}$. Par $(T-ABS)$,
%$\Delta \vdash \lambdaExpr{x : T_{1}}{t'} : T_{1} \rightarrow T_{2}$. \fg
%
%Cependant, l'hypothèse de récurrence est \og $\Gamma \vdash t : T$ implique $\Delta
%\vdash t : T$ \fg, et non \og $\Gamma, x : T_{1} \vdash t : T$ implique $\Delta, x :
%T_{1} \vdash t : T$ \fg : le contexte n'est pas le même.
%
%L'argument reste pourtant vrai : le principe de récurrence utilisé est celui sur
%$\naturel^{2}$ muni de l'ordre lexicographique en utilisant comme premier
%paramètre la hauteur de l'arbre de typage (qui diminue strictement dans le cas donné) et
%comme second la taille du contexte (qui augmente). L'énoncé complet du lemme de
%permutation devrait être:
%
%\og Soit $\Gamma \vdash t : T$ dont l'arbre de dérivation est de hauteur $n$ et
%dont la cardinalité de $\Gamma$ est $m$. Alors, pour toute permutation $\Delta$ de
%$\Gamma$, $\Delta \vdash t : T$ \fg.
%
%Pour rappel, le principe de récurrence sur $\naturel^{2}$ est le suivant :
%\begin{proposition}
%  Soit $P$ une proposition sur $\naturel^{2}$.
%
%  Si, pour tout $(m, n) \in \naturel^{2}$, $P(m', n')$ est vrai pour tout $(m',
%  n') \leq (m, n)$, alors, $P(m, n)$ est vrai pour tout $(m, n) \in \naturel^{2}$.
%\end{proposition}
%
%et se démontre en plusieurs lemmes :
%
%\begin{lemma} [Principe d'induction sur un ensemble bien ordonné]
%  Soit $(X, \leq)$ un ensemble bien ordonné, alors le principe d'induction est
%  vrai sur $X$, c'est-à-dire:
%
%  Soit $P$ une proposition sur $X$.
%  Si pour tout $x \in X$, quelque soit $y \in X$ tel que $y \leq x$, $P(y)$ est
%  vrai, alors $P(x)$ est vrai pour tout $x \in X$.
%\end{lemma}
%
%\begin{proof}
%  Même principe que la preuve sur $\naturel$.
%\end{proof}
%
%\begin{lemma}
%  Soit $(X, \leq)$ un ensemble bien ordonné. Alors $(X^{2}, \leq_{l})$ où
%  $\leq_{l}$ est l'ordre lexicographique est bien ordonné.
%\end{lemma}
%
%\begin{proof}
%  Soit $S \subseteq X^{2}$.
%
%  Posons $S_{1} = \GSsetDef{x \in X}{\exists y \in X \text{ tel que } (x, y) \in S}$.
%  Comme $S_{1} \subseteq X$ et $X$ est bien ordonné, $S_{1}$ possède un minimum.
%  Notons le $\min{S_{1}}$.
%
%  Posons $T = \GSsetDef{y \in X}{(\min{S_{1}}, y) \in S}$. Comme $T \subseteq
%  X$, $T$ possède un minimum. Notons le $\min{T}$.
%
%  Alors, $s = (\min{S_{1}}, \min{T})$ est le minimum de $S$. En effet, si $(x, y)
%  \in S$, on a $x \in S_{1}$ car $(x, y) \in S$, donc $\min{S_{1}} \leq x$ et si
%  $\min{S_{1}} = x$, alors, comme $y \in T$, $\min{T} \leq y$. Par construction,
%  $s \in S$.
%\end{proof}
%
%De manière générale, le dernier lemme peut se démontrer, en utilisant les mêmes
%arguments, pour $X^{n}$ où $n$ est un naturel quelconque.