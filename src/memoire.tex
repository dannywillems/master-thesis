\documentclass[10pt,a4paper]{memoire-umons}

\usepackage[utf8]{inputenc}
\usepackage[T1]{fontenc}
\usepackage[francais]{babel}
\usepackage{amssymb,amsmath,amsthm}
\usepackage{fancyvrb}

% Pour la coloration du code. Le dossier de sortie doit être ajouté car un
% fichier pyg est créé pendant la compilation et est nécéssaire pour la suite de
% la compilation.
\usepackage[outputdir=_build]{minted}

%Pour les règles d'inférence.
\usepackage{mathpartir}

% Used for tree representation.
\usepackage{qtree}

\usepackage{url}

\usepackage{hyperref}% hyperliens dans le PDF, pas pour impression

\title{Vers un langage typé pour la programmation modulaire}
\author{Danny \textsc{Willems}}
\date{2016--2017}
\directeur{François Pottier}
\codirecteurs{Christophe Troestler}
\service{Service d'Analyse Numérique}
%\rapporteurs{
%  Christophe Troestler
%}
\discipline{Mathématiques}

% Compile uniquement certains morceaux sans perdre les références
% automatiques et la table des matières des parties déjà compilées :
%\includeonly{introduction,chapitre1}

\usepackage{amsfonts}
\usepackage{amssymb}
\usepackage{amsmath}
\usepackage{amsthm}
\usepackage{mathrsfs}

\newtheorem{definition}{Définition}[chapter]

\newtheorem{proposition}[definition]{Proposition}
\newtheorem{lemma}[definition]{Lemme}
\newtheorem{corollary}[definition]{Corollaire}
\newtheorem{theorem}[definition]{Théorème}

\newtheorem*{exemple}{Exemple}
\newtheorem*{question}{Questions}
\newtheorem*{remarque}{Remarque}
\newtheorem*{notation}{Notation}

\newtheorem{exercice}{Exercice}[chapter]

\input{latex-macros-math/macros_math}

\begin{document}

\chapter*{Remerciements}

En premier lieu, je remercie François Pottier pour avoir accepté de me suivre
pour ce mémoire et de m'avoir permis d'intégrer l'INRIA Paris pendant toute la
durée de celui-ci. Sans nos discussions, ses disponibilités, ses conseils et ses
remarques, ce travail n'aurait pas pu être réalisé.

Je remercie également Christophe Troestler pour m'avoir aidé à choisir mon sujet
de mémoire ainsi que les conseils quant à la rédaction de ce document.

Je remercie chaque membre de l'équipe Gallium de
l'INRIA avec qui j'ai discuté et qui m'ont permis de découvrir de nouveaux
domaines dans la recherche informatique, plus ou moins éloigné du sujet de mon mémoire.

Je remercie également Vincent Balat qui m'a permis de découvrir lors de mon
stage différents chercheurs dans le domaine de la recherche dans les langages de
programmation. Sans ses conseils et son aide, je n'aurais eu l'idée de contacter
les membres de l'équipe Gallium afin d'obtenir un sujet.

Ensuite, je tiens à remercier Paul-André Melliès pour, dans un premier temps,
m'avoir invité à suivre son cours de lambda-calcul et catégories à l'ENS Ulm qui
m'a donné l'envie d'explorer plus en profondeur le lien entre l'informatique
théorique et les catégories, et,
dans un second temps, pour sa disponibilité et ses conseils lors de la recherche
de mon sujet de mémoire.

Entre autres, je remercie chaque personne ayant porté ou portant de l'intérêt à mon
travail, ce qui me pousse à continuer d'explorer ce sujet par la suite.

Je remercie aussi les chercheurs et développeurs travaillant sur
DOT\footnote{Dependent Object Type}, travail de recherche sur lequel mon travail
est basé, pour leurs disponibilités et leurs réponses à mes questions. En
particulier, je remercie Nada Amin, dont la thèse est consacrée à DOT, pour ses
réponses à mes emails.

Pour finir, je remercie chaque professeur m'ayant suivi pendant ces années
d'études.

\tableofcontents

\chapter{Introduction}

La programmation modulaire est un principe de développement consistant à séparer une
application en composants plus petits appelés \textit{modules}. Le langage de
programmation OCaml contient un langage de modules qui permet aux développeurs
d'utiliser la programmation modulaire. Dans ce langage de module, un module est
un ensemble de types et de valeurs, les types des valeurs pouvant dépendre des
types définis dans le même module.
OCaml étant un langage fortement typé, les modules possèdent également
un type, appelé dans ce cas \textit{signature}.

%% TODO exemple d'un module OCaml et de sa signature.

Bien que les modules soient bien intégrés dans OCaml, une
distinction est faite entre le langage de base, contenant les types dits \og de
bases \fg comme les entiers, les chaînes de caractères ou les fonctions, et le
langage de module. En particulier, le terme \textit{foncteur} est employé à la
place de \textit{fonction} pour parler des fonctions prenant un module en
paramètres et en retournant un autre. De plus, il n'est pas possible de définir
des fonctions prenant un module et un type de base et retournant un module (ou un
type de base).

D'un autre côté, dans les types de bases d'OCaml se trouvent les
\textit{enregistrements}. Ces derniers sont des ensembles de couples
\textit{(label, valeur)}, et ressemblent aux modules. Cependant, la différence
majeure entre eux se situent dans la possibilité de définir des types dans un
module.

Ce mémoire vise à définir dans un premier temps un calcul typé dans lequel le
langage de modules est confondu avec le langage de base grâce aux
enregistrements et dans un second temps, de fournir une implémentation en OCaml
des algorithmes de typage et de sous-typage.

%Une preuve de la sûreté de ce calcul ainsi
%qu'un interpréteur avec un algorithme de sous-typage et d'inférence de type est
%fourni.

Les chapitres sont organisés afin de comprendre la construction d'un tel calcul
à partir du plus simple des calculs, le $\lambda$-calcul.

Dans le chapitre 1, nous présentons \textit{le $\lambda$-calcul non typé}, un calcul
minimal qui contient des termes pour les variables, pour les abstractions (afin
de représenter des fonctions) et des applications (afin de représenter
l'application d'une fonction à un paramètre). Nous discuterons également de la
sémantique que nous attribuons à ce calcul. 

Dans le chapitre 2, nous introduisons la notion de type et nous l'appliquons au
$\lambda$-calcul, ce qui nous donnera \textit{le $\lambda$-calcul simplement typé}. Nous
discuterons de la notion de \textit{sureté du typage} à travers \textit{les théorèmes
de préservation et de progression} que nous démontrons pour ce calcul typé.

Dans les chapitres 3, 4 et 5, nous enrichissons le $\lambda$-calcul simplement
typé avec la notion de polymorphisme qui permet d'attribuer plusieurs types à un
terme. Le chapitre 3 se concentre sur \textit{le polymorphisme avec sous-typage},
illustré avec les enregistrements. Nous parlerons aussi de l'implémentation de
ce calcul et nous montrerons qu'un travail est nécessaire pour passer de la
théorie à l'implémentation.
Dans le chapitre 4, nous parlons de
\textit{polymorphisme paramétré} qui, combiné au $\lambda$-calcul simplement typé, forme
le calcul appelé \textit{System F}.
Le chapitre 5 se charge de combiner ces deux notions de polymorphismes dans un
calcul appelé \textit{System $F_{<:}$}.
Une preuve des théorèmes de préservation et de progression seront donnés pour
les calculs définis dans les chapitres 3 et 4.

Dans le chapitre 6, nous parlerons de la notion de type récursif et nous
étudierons \textit{le $\lambda$-calcul simplement typé avec type récursif}.

Pour finir, dans le chapitre 7, nous complétons les enregistrements définis
dans le chapitre 3 avec les \textit{types chemins dépendants} qui offre la possibilité
d'ajouter des types dans les enregistrements. Ce dernier chapitre comportera en
plus des types chemins dépendants, chaque notion étudiée précédemment,
c'est-à-dire le $\lambda$-calcul simplement typé, les types récursifs, les
enregistrements, le polymorphisme par sous-typage et le polymorphisme paramétré.

La principale difficulté de ce travail se trouve dans l'étude des types
chemins dépendants, sujet de recherche récent et moins bien compris que les
calculs comme \textit{System $F$} ou \textit{System $F_{<:}$}.

% Le sujet de ce mémoire est un sujet de recherche proposé à l'INRIA aux étudiants
%du MPRI.
\chapter{$\lambda$-calcul non typé}

Dans ce chapitre, nous allons introduire les bases théoriques de la
programmation fonctionnelle en parlant du $\lambda$-calcul non typé.
Nous discutons de la syntaxe de ce langage (les termes) pour ensuite discuter de
la réduction de ceux-ci à travers la $\beta$-réduction.


\section{Syntaxe}

\begin{definition} [Syntaxe du $\lambda$-calcul]
  Soit $V$ un ensemble infini dénombrable dont les éléments sont appelés \textbf{variables}. On note $\Lambda$, appelé \textbf{l'ensemble des $\lambda$-termes}, le plus petit
  ensemble tel que:
  \begin{enumerate}
    \item $V \subseteq \Lambda$
    \item $\forall u, v \in \Lambda, uv \in \Lambda$
    \item $\forall x \in V, \forall u \in \Lambda, \lambdaExpr{x}{u} \in \Lambda$
  \end{enumerate}
\end{definition}

Un élément de $\Lambda$ est appelé un \textbf{$\lambda$-terme}, ou tout
simplement un \textbf{terme}.
Un $\lambda$-terme de la forme $uv$ est appelé \textbf{application} car
l'interprétation donnée est une fonction $u$ évaluée en $v$.
Un $\lambda$-terme de la forme $\lambdaExpr{x}{u}$ est appelé
\textbf{abstraction}, le terme $u$ étant appelé le \textbf{corps}, et est interprété comme la fonction qui envoie
$x$ sur $u$.

La plupart des ensembles que nous définirons seront définis de manière récursive
comme ci-dessus.
Pour des raisons de facilité d'écriture, la syntaxe
\begin{center}
  \begin{math}
    \Lambda ::= V \, | \, \Lambda \Lambda \, | \, V \Lambda
  \end{math}
\end{center}

ou encore
\begin{align*}
  t ::= & \, & \text{terme} \\
        & \; x & \text{var} \\
        & \; t \, t & \text{app} \\
        & \; \lambdaExpr{x}{t} & \text{abs}
\end{align*}

où $x$ parcourt l'ensemble des variables $V$ et $t$ l'ensemble des termes, sont
utilisées pour définir ces ensembles. La dernière syntaxe sera celle que nous
utiliserons tout le long de ce document car elle permet une visualisation simple
de la syntaxe des termes et permet de nommer chaque forme facilement.

Des exemples de $\lambda$-termes sont
\begin{itemize}
  \item la fonction identité : $\lambdaExpr{x}{x}$
  \item la fonction constante en $y$: $\lambdaExpr{x}{y}$
  \item la fonction qui renvoie la fonction constante pour n'importe quelle
    variable: $\lambdaExpr{y}{\lambdaExpr{x}{y}}$.
  \item l'application identité appliquée à la fonction identité:
    $(\lambdaExpr{x}{x}) (\lambdaExpr{y}{y})$
\end{itemize}

Comme le montrent le dernier exemple, des parenthèses sont utilisées pour
délimiter les termes.

Il est également possible de définir des fonctions à plusieurs paramètres à
travers la curryfication : une fonction prenant 2 paramètres sera représentée
par une fonction qui renvoie une fonction. Par exemple,
$\lambdaExpr{x}{\lambdaExpr{y}{x y}}$ est une fonction qui attend un paramètre
$x$ retournant une fonction qui attend un paramètre $y$,
mais elle peut aussi être interprétée comme une fonction à deux paramètres $x$, $y$.
%Le terme $\lambda x . \lambda y . xy$ $\lambda x . \lambda y . (x y)$ et $(\lambda x
%. \lambda y . x) y$ sont différents selon la positionnement des parenthèses.

Comme dans une formule mathématique, il est important de différencier les
variables libres et les variables liées d'un $\lambda$-terme. Par exemple, dans
le $\lambda$-terme $\lambdaExpr{x}{x}$ la variable $x$ est liée par un
$\lambda$\footnote{on dit aussi qu'elle est \og sous \fg un $\lambda$.}
tandis que dans l'expression $\lambdaExpr{x}{y}$ la variable $y$ est libre.
Nous définissons récursivement l'ensemble des variables libres et l'ensemble des
variables liées à partir des variables, des abstractions et des applications.

\begin{definition} [Ensemble de variables libres]
  L'ensemble des variables \textbf{libres} d'un terme $t$, noté $FV(t)$ est défini
  récursivement sur la structure des termes de $\Lambda$ par:
  \begin{itemize}
  \item[$\bullet$] $FV(x) = \GSset{x}$
  \item[$\bullet$] $FV(\lambdaExpr{x}{t}) = FV(t) \backslash \GSset{x}$
  \item[$\bullet$] $FV(u v) = FV(u) \union FV(v)$
  \end{itemize}
\end{definition}

\begin{definition} [Ensemble de variables liées]
  L'ensemble des variables \textbf{liées} d'un terme $t$, noté $BV(t)$ est défini
  récursivement sur la structure des termes de $\Lambda$ par:
  \begin{itemize}
  \item[$\bullet$] $BV(x) = \emptyset$
  \item[$\bullet$] $BV(\lambdaExpr{x}{t}) = BV(t) \union \GSset{x}$
  \item[$\bullet$] $BV(u v) = BV(u) \union BV(v)$
  \end{itemize}
\end{definition}

Un terme qui ne comporte pas de variable libre est dit \textit{clos}.

Il existe également des termes qui sont syntaxiquement différents, mais que nous
voudrions naturellement qu'ils soient les mêmes. Par exemple, nous voudrions que
la fonction identité $\lambdaExpr{x}{x}$ ne dépende pas de la variable liée $x$,
c'est-à-dire que les termes $\lambdaExpr{x}{x}$ et $\lambdaExpr{y}{y}$ soient un seul et
unique terme: la fonction identité. Cette égalité se résume à une substitution
de la variable $x$ par la variable $y$, ou plus généralement par un terme $u$.

Avant de donner une définition exacte, il est important de remarquer que la
substitution n'est pas une action triviale si nous ne voulons pas changer le
sens des termes. Si nous effectuons une
substitution purement syntaxique, nous pouvons alors obtenir des termes qui ne sont
plus dans la syntaxe des éléments de $\Lambda$. Par exemple, si nous
substituons toutes les occurences de $x$ par un terme $u$ dans la fonction constante
$\lambda x . y$, nous aurions $\lambda u . y$, qui n'a pas de sens car $u$ n'est
pas obligatoirement une variable.

La définition doit aussi prendre en compte les notions de variables liées et libres. En effet, si nous
prenons la fonction constante $\lambda x . y$ et que nous substituons $y$ par 
$x$ uniquement dans le corps de la fonction, nous obtenons $\lambda x . x$, qui
n'a pas le même sens que $\lambda x . y$. Cet exemple nous montre que nous
devons faire attention lorsque la variable à substituer, dans ce cas $x$, est
liée dans le terme où se passe la substitution (ici $\lambda x . y$).

Un autre exemple où la substitution n'est pas évidente est la substition de la
variable $z$ du terme $\lambda x . z$ (la fonction constante en $z$) par le terme
$\lambda y . x$ (la fonction constante en $x$). Après subtitution, nous nous
retrouvons avec le terme $\lambda x . \lambda y . x$, c'est-à-dire la fonction
qui renvoie la fonction constante pour le paramètre donné. Ce dernier exemple
nous montre que nous devons également faire attention aux variables libres du
terme substituant.

\begin{definition} [Substitution de variable par un terme]
  Soit $x \in V$ et soient $u, v \in \Lambda$. On dit que la variable $x$ est
  \textbf{substituable par $v$ dans $u$} si et seulement si $x \notin BV(u)$ et
  $FV(v) \inter BV(u) = \emptyset$.
\end{definition}

Nous définissons alors la fonction de substitution d'une variable $x$ par un
terme $v$ dans un terme $u$.

\begin{definition} [fonction de subtitution]
  Soient $x$ une variable et $u, v \in \Lambda$ tel que $x$ est substituable par
  $v$ dans $u$. On définit récursivement la fonction de substitution, notée $u[x := v]$, par:
  \begin{itemize}
  \item[$\bullet$] $x[x := v] = v$
  \item[$\bullet$] $y[x := v] = y$ (si $y \neq x$)
  \item[$\bullet$] $(u_{1} u_{2})[x := v] = (u_{1}[x := v])(u_{2}[x := v])$
  \item[$\bullet$] $(\lambda y . u)[x := v] = \lambda y . (u[x := v])$
  \end{itemize}
  $u[x := v]$ se lit \textit{x est substitué par v dans u}.
\end{definition}

Nous définissons maintenant une relation, appelée relation d'
\textbf{$\alpha$}-renommage, sur les abstractions qui capture
notre volonté d'égalité à renommage de variables près.

\begin{definition} [relation d'$\alpha$-renommage]
  Soient $x, y \in V$ et $u \in \Lambda$.
  La relation d'$\alpha$-renommage, notée $\alpha$, est définie par

  \begin{math}
    \lambda x . u \;\; \alpha \;\; \lambda y . (u[x := y])
  \end{math}

  si $x = y$ ou si $x$ est substituable par $y$ dans $u$ et $y$ n'est pas libre
  dans $u$.
\end{definition}

Nous allons étendre cette relation à tous les termes, c'est-à-dire sur tout l'ensemble $\Lambda$.

Nous notons $=_{\alpha}$ la plus petite relation comprenant $\alpha$ et tel que
\begin{itemize}
  \item[$\bullet$] $=_{\alpha}$ est réflexive, symétrique et transitive
  \item[$\bullet$] $=_{\alpha}$ passe au contexte: si $u_{1} =_{\alpha} v_{1}$
    et $u_{2} =_{\alpha} v_{2}$ alors $u_{1}u_{2} =_{\alpha} v_{1}v_{2}$ et
    $\lambda x . u_{1} =_{\alpha} \lambda x . v_{1}$.
\end{itemize}

\begin{exemple}
  \begin{enumerate}
    \item Il est clair que $\lambda x . x =_{\alpha} \lambda y . y$ par
    définition de la relation $\alpha$.
    \item De même, $\lambda x . \lambda y . x y =_{\alpha} \lambda y . \lambda x
      . y x$. En
      effet, on montre que $\lambda x . \lambda y . x y =_{\alpha} \lambda z .
      \lambda w. z
      w$ et $\lambda y . \lambda x . y x =_{\alpha} \lambda z . \lambda w . z w$ en appliquant
      deux fois la subtitution (par $z$ et par $w$). Par symétrie
      et transitivité de $=_{\alpha}$, on obtient l'égalité.
  \end{enumerate}
\end{exemple}

Par définition, la relation $=_{\alpha}$ est une relation d'équivalence. Nous
construisons alors le quotient $\Lambda \backslash =_{\alpha}$. Dans ce
quotient, les termes égaux à renommage de variable près se retrouvent dans la
même classe d'équivalence. A partir de maintenant, nous considérons $\Lambda
\backslash =_{\alpha}$, c'est-à-dire que nous parlons des termes à $\alpha$-renommage près.

\section{Sémantique}

Maintenant que nous avons introduit la syntaxe du $\lambda$-calcul, nous allons
discuter de la sémantique que nous lui associons, c'est-à-dire comment nous
effectuons des calculs avec ce langage. Les calculs
se définissent par des \textit{réductions}\footnote{On parle aussi de
\textit{réécriture}.} de termes, et en particulier des applications. Par exemple,
nous voudrions dire que $(\lambda x . x) y$, i.e. $y$ appliqué à la fonction
identité, se \t
extit{réduit} en $y$ ou encore que $(\lambda x . (\lambda y . x
y)) z$, i.e. $z$ appliqué à la fonction qui retourne la fonction
constante pour toute variable, se réduit en $\lambda y . z y$, i.e. la fonction
constante en $z$. Nous parlons également \textit{d'étape de calcul}, une étape
de calcul correspondant à une réduction effectuée.

La définition de réduction des termes passe par une relation entre les
termes appelée relation de $\beta$-réduction.
Comme pour la relation $\alpha$, nous commençons par définir une relation
$\beta$, et nous l'étendons au contexte.

\begin{definition} [Relation de $\beta$-réduction]
  Soit $\beta$ la relation sur $\Lambda$ tel que $(\lambda x . u) v \; \; \beta 
  \; \; u[x := v]$.\footnote{Ne pas oublier que nous travaillons à
    $\alpha$-renommage près.}
  La relation de $\beta$-réduction, noté $\rightarrow_{\beta}$, ou simplement
  $\rightarrow$, est la plus petite relation contenant $\beta$ qui passe au
  contexte. Nous notons $\rightarrow^{*}_{\beta}$ sa fermeture réfléxive
  transitive et $\rightarrow^{+}_{\beta}$ sa fermeture transitive.
\end{definition}

Voici quelques exemples de réductions:
\begin{exemple}
  \begin{enumerate}
  \item $(\lambda x . x) y \rightarrow y$.
  \item $(\lambda y . (\lambda x . y x)) z \rightarrow \lambda x . z x$.
  \item $(\lambda w . (\lambda x . x) v) z \rightarrow (\lambda w. v) z$ (on
    réduit à l'intérieur, c'est-à-dire $(\lambda x . x) v$).
  \item $(\lambda w . (\lambda x . x) v) z \rightarrow (\lambda x. x) v$ (on
    réduit à l'extérieur, c'est-à-dire $(\lambda w . t) z$ où $t = (\lambda x .
    x) v$).
  \item $(\lambda w . (\lambda x . x) v) z \rightarrow^{*}_{\beta} v$
  \item $(\lambda x . xx) (\lambda x . xx) \rightarrow (\lambda x . xx)
    (\lambda x . xx)$.
  \end{enumerate}
\end{exemple}

Un élément de la forme $(\lambda x . u) v$ est appelé \textit{redex}. En
analysant les termes que $\beta$ met en relation, la
$\beta$-réduction consiste donc à réécrire les redex.

Nous définissons aussi les \textit{valeurs} qui sont les termes finaux
possibles d'une $\beta$-réduction. Dans le cas du $\lambda$-calcul, les valeurs
sont les abstractions.

Un terme $t$ qui peut être réduit, c'est-à-dire qu'il existe $u$ tel que $t
\rightarrow u$, est dit \textit{réductible}. Sinon, il est dit
\textit{irréductible} ou on dit également que c'est une \textit{forme normale}.
S'il est possible de trouver une forme normale $u$ tel que $t$ se réduit en $u$,
on dit que $t$ \textit{possède une forme normale} et que $u$ est une
\textit{forme normale de $t$}.

Certains termes peuvent être réduits en des formes normales comme
dans les deux premiers exemples. Dans le premier exemple, le terme
irréductible n'est pas une valeur tandis que dans le second, nous obtenons
une valeur.

Lorsque toute réduction commençant par $t$ possède une forme normale, on dit que
le terme $t$ est \textit{fortement normalisant}, ou tout simplement
\textit{normalisant}. Lorsqu'il existe au moins une stratégie de réduction qui
permet d'obtenir un terme irréductible, on dit que le terme est \textit{faiblement normalisant}.

Le troisième et quatrième exemples montrent qu'il existe plusieurs manières,
appelées aussi \textit{stratégie de réduction}, de
réduire un terme.

Le dernier exemple montre qu'il existe des termes dont aucune réduction se
termine. Celui-ci nous montre que la $\beta$-réduction ne se termine pas toujours.
Ce fait n'est pas si étrange que ça : dans la plupart des langages de
programmation, il est possible d'écrire des programmes qui bouclent à l'infini,
c'est-à-dire que la réduction ne se termine pas.

%La relation $\rightarrow_{\beta}$ signifie que nous faisons une étape de calcul,
%contrairement à la fermeture transitive qui en réalise plusieurs. Choisir l'une
%ou l'autre définit \textit{la stratégie d'évaluation} que nous assignons à notre
%calcul. $\rightarrow_{\beta}$ définit \textit{l'évaluation à petit pas} car
%nous obtenons chaque étape de l'évaluation. La fermeture transitive définit
%\textit{l'évaluation à grand pas} qui consiste à ne regarder que la forme
%normale obtenue, si elle existe.

%Dans la suite, nous considérerons toujours que nous utilisons la stratégie
%d'évaluation à petit pas.

\subsection{Stratégies de réduction}

Nous présentons les stratégies les plus utilisées. Le terme
\begin{math}
  id (id (\lambda z . id z))
\end{math}

où $id = \lambda x . x$ sera utilisé pour interpréter chaque stratégie de réduction.

\subsubsection*{Ordre normale}

Cette méthode réduit d'abord les redex à l'extérieur, les plus à gauche.
La chaine de réduction de notre exemple est alors:

\begin{align*}
  & \underline{id \, (id \, (\lambda z . (id \, z)))} \rightarrow_{\beta}\\
  & \underline{id \, (\lambda z . (id \, z))} \rightarrow_{\beta} \\
  & \lambda z . \underline{(id \, z)} \rightarrow_{\beta} \\
  & \lambda z . z
\end{align*}

\subsubsection*{Call by name}

La stratégie appelée \textit{call-by-name} consiste à réduire les redex les plus
à gauche en premier, comme l'ordre normale. La différence est que le
call-by-name ne permet pas de réduire les redex qui sont dans le corps d'un lambda.

\begin{align*}
  & \underline{id \, (id \, (\lambda z . (id \, z)))} \rightarrow_{\beta}\\
  & \underline{id \, (\lambda z . (id \, z))} \rightarrow_{\beta} \\
  & \lambda z . (id \, z)
\end{align*}

La dernière étape de réduction de l'ordre normal n'est pas effectuée car celle-ci est
sous le lambda $\lambda z$.

\subsubsection*{Call by value}

La réduction dite \textit{call-by-value} consiste à réduire en premier les
redexes les plus à l'extérieur et réduire les arguments jusqu'à obtenir une valeur, et ensuite le corps de la fonction.
Cette méthode de réduction est la plus courante dans les langages de
programmation.

\begin{listing}
  \inputminted{OCaml}{codes/untyped-call-by-value-example.ml}
  \caption{Exemple qui montre que la stratégie de réduction utilisée par défaut
    dans OCaml est le call-by-value.}
\end{listing}

Cette stratégie appliquée à l'exemple donne:
\begin{align*}
  & id \underline{(id (\lambda z . (id \, z)))} \rightarrow_{\beta}\\
  & id \underline{(\lambda z . (id \, z))} \rightarrow_{\beta} \\
  & \lambda z . (id \, z)
\end{align*}

\label{fig:untyped-lambda-calculus-evaluation-rules}

%\noindent
%\begin{minipage}{0.45\textwidth}
%  \begin{align*}
%    t ::= & \, & \text{terme} \\
%          & \; x & \text{(var)} \\
%          & \; t \, t & \text{(abs)} \\
%          & \; \lambdaExpr{x}{t} & \text{(app)}
%  \end{align*}
%
%  \begin{align*}
%    v ::= & \, & \text{valeur} \\
%          & \; \lambdaExpr{x}{t} & \text{(app)}
%  \end{align*}
%\end{minipage}
%\begin{minipage}{0.45\textwidth}
%  \begin{center}
%    \inferrule{t \rightarrow t'}{v \, t \rightarrow v \, t'}
%    \inferrule{t \rightarrow t'}{t \, t \rightarrow v \, t'}
%  \end{center}
%\end{minipage}

   % v ::= & \, & \text{valeur} \\
   %         & \, \lambdaExpr{x}{t} &
%  \end{align*}

% \end{minipage}


%  {\begin{align}
%    t ::= & \, & \text{term} \\
%          & \, x & \text{VAR} \\
%          & \, t \, t & \text{ABS} \\
%          & \, \lambdaExpr{x}{t} & \text{APP}
%  \end{align}} & t \\
%  \begin{align*}
%    t ::= & \, & \text{term} \\
%          & \, x & \text{VAR} \\
%          & \, t \, t & \text{ABS} \\
%          & \, \lambdaExpr{x}{t} & \text{APP}
%  \end{align*} \\

Formellement, la stratégie call-by-value est définie par les \textit{règles
  d'évaluation} définies ci-dessous, le terme $v$ étant utilisé pour une valeur.

\begin{mathpar}
  \inferrule
    {t_{1} \rightarrow t_{1}'}
    {t_{1} \, t \rightarrow t_{1}' \, t} \quad (\textsc{E-APP1})
  \and
  \inferrule
    {t \rightarrow t'}
    {v \, t \rightarrow v \, t'} \quad (\textsc{E-APP2})
  \and
  \inferrule
    {(\lambdaExpr{x}{t}) v \rightarrow [x := v]t}
    {} \quad (\textsc{E-APPABS})
\end{mathpar}
\label{eval:untyped-lambda-calculus}

La notation \inferrule{t \rightarrow t'}{v \, t \rightarrow v \, t'} est
l'équivalent d'une implication où la prémisse (ici $t \rightarrow t'$) se trouve au dessus et la
conclusion en dessous (ici $v \, t \rightarrow v \, t'$). La règle E-APP1 se lit
donc \og si $t_{1}$ se réduit en $t_{1}'$, alors $t_{1} t$ se réduit en
$t_{1}'t$ \fg. Lorsque qu'une règle
ne comporte pas de conclusion comme E-APPABS, cela signifie que c'est un axiome.
Cette notation sera
utilisée tout au long de ce document, en particulier pour les règles de typage
et de sous-typage.

Les règles (E-APP1) et (E-APP2) nous disent que nous devons, lors d'une
application, réduire la fonction avant les paramètres, et ce jusqu'à obtenir une
valeur. Quant à la règle (E-APPABS), elle signifie qu'un redex se réduit toujours
en utilisant la fonction de substitution (définition de la relation $\beta$).

La relation de $\beta$-réduction pour la stratégie call-by-value est alors
définie comme le plus petit ensemble généré par les règles
\ref{eval:untyped-lambda-calculus}. Quand nous ajouterons à notre langage des
autres termes comme les enregistrements, nous mentionnerons uniquement les
règles d'évaluation, la relation de $\beta$-réduction étant implicitement
définie de la même manière.

Par la suite, nous considérerons toujours cette dernière stratégie car c'est la
plus utilisée.

La réécriture des termes est un large sujet, plus d'informations sur ce sujet
sont disponibles dans \cite{ENS-Cachan-cours-lambda-calcul}.
Dans ce cours sont traités les sujets de normalisation (forme normale, finitude
de la $\beta$-réduction), de confluence (est-ce que tout terme se réduit en une
unique forme normale) et d'une différente sémantique appelée \textit{sémantique
dénotationnelle}.

\section{Codage de termes usuels}

Le $\lambda$-calcul est assez riche pour définir des
termes usuels des langages de programmations comme les booléens (et en même temps les
conditions), les paires ou encore les entiers. Ces codages peuvent être trouvées
dans \cite{tapl-untyped-lambda-calculus}. Voici l'exemple des booléens, utilisé dans RML (\cite{rml-github}):

\begin{itemize}
  \item $true = \lambdaExpr{t}{\lambdaExpr{f}{t}}$
  \item $false = \lambdaExpr{t}{\lambdaExpr{f}{f}}$
  \item $test = \lambdaExpr{b}{\lambdaExpr{t'}{\lambdaExpr{f'}{b \, t' \, f'}}}$
\end{itemize}

Avec les définitions de $true$ et $false$, la fonction $test$ simule le fonctionnement d'une condition: si le premier
paramètre ($b$) est $true$, il renvoie $t'$, si c'est $false$, il renvoie
$f'$. En effet,

\begin{align*}
  & test \, true \, v \, w \rightarrow_{\beta} \\
  & (\lambdaExpr{b}{\lambdaExpr{t'}{\lambdaExpr{f'}{b \, t' \, f'}}}) \, true \, v \, w \rightarrow_{\beta} \\
  & (\lambdaExpr{t'}{\lambdaExpr{f'}{true \, t' \, f'}}) \, v \, w \rightarrow_{\beta} \\
  & (\lambdaExpr{f'}{true \, v \, f'}) \, w \rightarrow_{\beta} \\
  & true \, v \, w \rightarrow_{\beta} \\
  & v
\end{align*}

Un même raisonnement se fait pour $test \, false \, v \, w$, qui donne $w$.

Les fonctions $and$ et $or$ peuvent aussi être codées en
$\lambda$-calcul.

\begin{itemize}
  \item $and = \lambdaExpr{b}{\lambdaExpr{b'}{b \, b' \, false}}$
  \item $or = \lambdaExpr{b}{\lambdaExpr{b'}{b \, true \, b'}}$
\end{itemize}
\chapter{$\lambda$-calcul simplement typé}

Dans le chapitre 1, nous avons défini la syntaxe et la sémantique d'un calcul
appelé le $\lambda$-calcul non typé. Nous allons maintenant ajouter une notion de types à
chaque terme de notre calcul, ce qui nous mènera au $\lambda$-calcul simplement
typé\footnote{Plus d'informations peuvent être trouvées dans \cite{tapl-simply-typed-lambda-calculus}}.

\section{Typage, contexte de typage et règle d'inférence}

Le typage consiste à classer les termes en fonction de leur nature. Par exemple,
une abstraction est interprétée comme une fonction prenant un paramètre et
renvoyant un terme. Nous représentons cela par le type $\rightarrow$, appelé
couramment \textit{type flèche}. Un type flèche dépend naturellement de deux autres types : le
type du terme qu'il prend en paramètre (disons $T_{1}$) et le type du terme
qu'il retourne (disons $T_{2}$). Dans ce cas, l'abstraction est dite de type
$T_{1} \rightarrow T_{2}$, lu \og $T_{1}$ flèche $T_{2}$ \fg. Un
autre exemple est l'application. Une application $u \, v$ représentant une
application de $v$ à la fonction $u$, il serait naturel de dire que $u$ est un
type flèche dont le type de son paramètre est le type de $v$.

\begin{definition} [Relation de typage]
  \label{def:simply-typed-lambda-calculus-type-relation}
  Soit $\Lambda$ un ensemble de termes.
  Soit $\tau$ un ensemble, appelé \textbf{ensemble des types}, dont les éléments
  sont notés $T$.

  On définit une relation binaire $R$, appelée \textbf{relation de typage}, entre les
  termes et les éléments de $\tau$.
  
  On dit que \textbf{le terme $t \in \Lambda$ est de type $T \in \tau$} si $(t, T)
  \in R$, noté le plus souvent $t : T$. Si un terme $t$ est en relation avec au
  moins un type $T$, on dit que $t$ est \textbf{bien typé}.
\end{definition}

Cette définition de la relation de typage est générale car il suffit de se
donner un ensemble de termes et un ensemble de types. Dans ce chapitre, nous
allons nous focaliser sur les termes du $\lambda$-calcul non typé. Dans les
prochains chapitres, nous ajouterons d'autres termes comme les enregistrements
et nous devrons en conséquence leur assigner un type.

Dans ce chapitre, nous allons travailler avec l'ensemble des types dits
\textit{simples}.

\begin{definition}
  \label{def:simply-typed-lambda-calculus-types}
  Soit $B$ un ensemble de types appelés \textbf{types de base}.
  L'ensemble des \textbf{types simples} est défini par la grammaire suivante :
  \begin{align*}
    T ::= & \, & \text{types} \\
          & \; B & \text{base} \\
          & \; T \rightarrow T & \text{type des fonctions}
  \end{align*}
\end{definition}

L'ensemble de base $B$ est assez naturel : il existe souvent dans les langages
des types dits de base ou primitifs.

%Le typage est donc un moyen de spécifier l'appartenance de certains
%$\lambda$-termes à un ensemble précis de types et ainsi réduire les opérations
%possibles sur ces $\lambda$-termes.

%Dans le chapitre 1 sur le $\lambda$-calcul non typé, nous avons défini une
%relation d'évaluation, noté $\rightarrow$. Nous pouvons nous demander comment la
%relation de typage est compatible avec la relation $\rightarrow$.

\subsection*{Contexte et jugement de typage}

Nous avons déjà mentionné que, naturellement, les abstractions $\lambdaExpr{x}{t}$ ont le type
flèche, par exemple $T_{1} \rightarrow T_{2}$. Cependant, comment pouvons-nous
connaître le type des arguments, c'est-à-dire le type du paramètre que la
fonction attend ? Deux solutions sont couramment utilisées : soit annoter la
variable avec un type, soit étudier le type du corps de la fonction et
en déduire le type que le paramètre devrait avoir.
Dans la suite, nous utiliserons la première solution. Le terme de l'abstraction
se voit alors ajouter un type à son argument et devient $\lambdaExpr{x : T}{t}$.
La syntaxe des termes devient alors :

\begin{align*}
  \label{def:simply-typed-lambda-calculus-terms}
  t ::= & \, & \text{terme} \\
        & \; x & \text{var} \\
        & \; t \, t & \text{app} \\
        & \; \lambdaExpr{x : T}{t} & \text{abs}
\end{align*}

Avant de discuter des règles de typage, il convient de remarquer qu'il est
nécessaire de connaître certaines informations quand nous souhaitons typer des
termes. En effet, si nous prenons le terme $\lambdaExpr{x : T} y$ et que nous
souhaitons le typer, il est nécessaire de connaître le type de $y$. Cela nous
amène à la notion de \textit{contexte de typage}.

\begin{definition} [Contexte de typage]
  \label{def:simply-typed-lambda-calculus-context}
  Un \textbf{contexte de typage}, noté $\Gamma$, est une suite finie de couples $(x_{i},
  T_{i})$ où $x_{i}$ est une variable et $T_{i}$ est un type. Chaque $x_{i}$ est différent.

  L'union d'un contexte de typage $\Gamma$ avec un couple $(x, T)$ est notée
  $\Gamma, x : T$ et l'union de $\Gamma$ avec $\Delta$ est notée $\Gamma, \Delta$.
  Le contexte vide est noté $\emptyset$.

  Le domaine de $\Gamma$, noté $dom(\Gamma)$, est l'ensemble des $x_{i}$.
\end{definition}

La relation de typage devient alors une relation à trois composantes : le
contexte, le terme et le type. Nous parlons alors de \textit{jugement de typage}.

\begin{definition} [Jugement de typage]
  \label{def:simply-typed-lambda-calculus-judgement}
  \textbf{Un jugement de typage} est un triplet $(\Gamma, t, T)$ où $\Gamma$ est un
  contexte de typage, $t$ un terme et $T$ un type. Nous le notons le plus
  souvent $\Gamma \vdash t : T$ et nous disons \og $t$ à le type $T$ sous les
  hypothèses $\Gamma$\footnote{ou encore dans le contexte $\Gamma$} \fg.
  Si $\Gamma$
  est vide, nous omettons $\emptyset$ et le jugement devient $\vdash t : T$.
\end{definition}

\subsection*{Règle de typage et arbre de dérivation}

Maintenant, nous avons les outils pour définir nos règles de typage,
c'est-à-dire comment nous attribuons les types aux termes.

\begin{definition} [Règles de typage]
  \label{def:simply-typed-lambda-calculus-typing-rules}
  Les règles de typage pour le $\lambda$-calcul simplement typé sont
  \begin{mathpar}
    \inferrule
      {(x : T) \in \Gamma}
      {\Gamma \vdash x : T} \quad (\textsc{T-VAR})
    \and
    \inferrule
      {\Gamma, x : T_{1} \vdash t : T_{2}}
      {\Gamma, x : T_{1} \vdash \lambdaExpr{x : T_{1}}{t} : T_{1} \rightarrow T_{2}} \quad (\textsc{T-ABS})
    \and
    \inferrule
      {\Gamma \vdash u : T_{1} \rightarrow T_{2} \\ \Gamma \vdash v : T_{2}}
      {\Gamma \vdash u \, v : T_{2}} \quad (\textsc{T-APP})
  \end{mathpar}
\end{definition}

La règle (T-VAR) est évidente : si $(x, T)$ est dans le contexte, alors
$x$ est de type $T$ sous le contexte $\Gamma$.
Quant à (T-ABS), elle affirme que si le corps $t$ de l'abstraction
$\lambdaExpr{x : T_{1}}{t}$ est de type $T_{2}$, alors l'abstraction est de type
$T_{1} \rightarrow T_{2}$.
Pour finir, (T-APP) type les applications : dans le terme $u \, v$, $u$ doit
être une fonction de type $T_{1} \rightarrow T_{2}$, et $v$ doit être du même
type que celui que $u$ attend, c'est-à-dire $T_{1}$, l'application ayant le type $T_{2}$.

Le typage d'un terme produit des \textit{arbres de dérivation de typage} (ou tout simplement
\textit{une dérivation de typage}). Un arbre de dérivation de typage
est un arbre dont les noeuds sont des jugements de typage, construits à partir
des règles de typage et dont la racine est le jugement de typage du terme à
typer. La racine de l'arbre est également appelée \textit{conclusion}.
La racine de l'arbre de dérivation est le jugement de typage le plus en bas. Les branches
sont annotées par le nom des règles qui permettent de déduire le type.

Par exemple, $\lambdaExpr{x : T_{1} \rightarrow T_{2} }{\lambdaExpr{y : T_{1}} x
\, y}$ est de type $(T_{1} \rightarrow T_{2}) \rightarrow T_{1} \rightarrow
T_{2}$. Un arbre de dérivation possible est
  \\

$
\inferrule* [Right=(\textsc{T-APP})]
  {\inferrule* [Left=(\textsc{T-VAR})]
    {x : T_{1} \rightarrow T_{2} \in \Gamma}
    {\Gamma \vdash x : T_{1} \rightarrow T_{2}}
    \\
  \inferrule* [Right=(\textsc{T-VAR})]
    {y : T_{1} \in \Gamma}
    {\Gamma \vdash y : T_{1}}
  }
  {\inferrule* [Right=(\textsc{T-ABS})]
    {\Gamma, x : T_{1} \rightarrow T_{2}, y : T_{1} \vdash x \, y : T_{2}}
    {\inferrule* [Right=(\textsc{T-ABS})]
      {\Gamma, x : T_{1} \rightarrow T_{2}
        \vdash
        \lambdaExpr{y : T_{1}}{x \, y} :
        T_{1} \rightarrow T_{2}
      } 
      {\Gamma \vdash \lambdaExpr{x : T_{1} \rightarrow T_{2}}{\lambdaExpr{y :
            T_{1}}{x \, y}} : (T_{1} \rightarrow
        T_{2}) \rightarrow T_{1} \rightarrow T_{2}
      }
    }
  } 
$


%Le $\lambda$-calcul simplement typé est donc un tuple $(\Lambda, \rightarrow,
%\tau, \Gamma_{\tau})$ où
%\begin{enumerate}
%\item $\Lambda$ est l'ensemble des $\lambda$-termes.
%\item $\rightarrow$ est la relation de $\beta$-réduction.
%\item $\tau$ l'ensemble des types.
%\item $\Gamma$ est le jugement de typage (défini récursivement).
%\end{enumerate}

\section{Sûreté du typage}

Dans cette partie, nous allons aborder deux théorèmes importants : les
théorèmes de progression et de préservation du typage. En assemblant ces deux
théorèmes, nous en déduisons le principe \og les
programmes\footnote{Un programme est synonyme de terme.}
bien typés ne bloquent pas \fg. Ne pas bloquer signifie que si le programme se
termine (un programme bien typé peut contenir une boucle infinie), alors il
se réduira en une valeur du type du programme.

Ces deux théorèmes relient les deux relations précédemment définies : la
relation de typage et la relation de $\beta$-réduction.

\begin{enumerate}
  \item Progression : si un terme est bien typé, alors soit c'est une
    valeur, soit il se réduit en un terme.
    \item Préservation (du typage) : si un terme $t$ de type $T$ se réduit en un terme $t'$,
      alors $t'$ est de type $T$.
\end{enumerate}

Avant de montrer la préservation et la progression, il est nécessaire de
remarquer certains faits qui découlent immédiatement des règles de typage.

\begin{lemma} [Inversion des règles de typage]
  \label{lemma:simply-typed-lambda-calculus-inversion}
  \begin{enumerate}
    \item Si $\Gamma \vdash x : T$, alors $(x : T) \in \Gamma$
    \item Si $\Gamma \vdash \lambdaExpr{x : T_{1}}{t_{2}} : T$ alors $T = T_{1}
      \rightarrow T_{2}$ pour un $T_{2}$ tel que $t : T_{2}$.
    \item Si $\Gamma \vdash t_{1} t_{2} : T$, alors il existe $T_{1}$
      tel que $t_{1} : T_{1} \rightarrow T$ et $t_{2} : T_{1}$.
  \end{enumerate}
\end{lemma}

\begin{proof}
  \label{proof:simply-typed-lambda-calculus-inversion}
  Ces propositions découlent des règles de typage. En effet, pour la deuxième
  par exemple, la seule règle qui permet d'affirmer que $\Gamma \vdash
  \lambdaExpr{x : T_{1}}{t_{2}} : T$ est (T-ABS).
\end{proof}

Le lemme d'inversion des règles de typage dit également quelque chose de
fondamental sur les arbres de typage et qui a une énorme importance lorsque
nous souhaitons implémenter un algorithme de typage\footnote{Nous verrons par la
suite que ce n'est pas tout le temps évident de passer des règles de typage à un
algorithme de typage.}. En effet, les 3 points du
lemme nous donnent quels sont les possibles fils de la conclusion. Par
exemple, si nous devons montrer que $\Gamma \vdash \lambdaExpr{x : T}{t} : T'$,
nous sommes convaincus, par le lemme d'inversion, que le noeud précédent
provient de la règle (T-ABS). Cela implique que pour un jugement de typage
donné, il n'y a au plus qu'un seul arbre de dérivation.

Une autre remarque importante, et qui découle du fait que les arbres de
dérivations sont uniques, est l'unicité de type. Nous verrons que cette
proposition n'est pas vraie dans tous les calculs.

\begin{theorem} [Unicité du typage]
  \label{thm:simply-typed-lambda-calculus-type-unicity}
  Soit $t$ un $\lambda$-terme. Si $t$ est bien typé, alors son type est unique.
  De plus, il existe au plus un arbre de dérivation qui permet de montrer que $t$
  a ce type.
\end{theorem}

\begin{proof}
  \label{proof:simply-typed-lambda-calculus-type-unicity}
  Supposons que $t$ possède deux types, par exemple $S$ et $T$. Nous avons donc les
  jugements de typage : $\Gamma \vdash t : S$ et $\Gamma \vdash t : T$. Nous procédons
  par induction sur la structure des termes.
  \begin{itemize}
   \item $t$ est une variable $x$. Alors nous avons les jugements de typage
     $\Gamma \vdash x : S$ et $\Gamma \vdash x : T$. Par le lemme d'inversion, nous
     en déduisons que $(x, S) \in
     \Gamma$ et $(x, T) \in \Gamma$. Comme une variable ne peut apparaître
     qu'une fois dans un contexte, nous en déduisons $S = T$.

   \item $t$ est de la forme $\lambdaExpr{x : T_{1}}{t'}$. Par le lemme
     d'inversion, $S = T_{1} \rightarrow R_{1}$ et $T = T_{1} \rightarrow
     R_{2}$ avec $t' : R_{1}$ et $t' : R_{2}$. Par induction sur $t'$, nous
     déduisons que $R_{1} = R_{2}$. Donc $S = T$.

   \item $t$ est de la forme $u \, v$. Par le lemme d'inversion, il existe
     $T_{1}$ et $T_{2}$ tel que $u$ est de type $T_{1} \rightarrow S$ et de type $T_{2}
     \rightarrow T$ avec $v$ de type $T_{1}$ et de type $T_{2}$. Par induction
     sur $u$ et sur $v$, le type de de $v$ est unique ($T_{1} = T_{2}$) et celui
     de $u$ également. Nous avons donc $S = T$.
   \end{itemize}

   L'unicité de l'arbre de dérivation découle immédiatement du lemme d'inversion
   et de la remarque ci-dessus.
\end{proof}

\subsection*{Progression}

\begin{theorem} [de progression de $\lambda_{\rightarrow}$]
  \label{thm:simply-typed-lambda-calculus-progression}
  Soit $t$ un terme bien typé sans variable libre. Alors, soit $t$ est une
  valeur, soit il existe $t'$ tel que $t \rightarrow t'$.
\end{theorem}

\begin{proof}
  \label{proof:simply-typed-lambda-calculus-progression}
  Nous procédons par induction sur la structure des termes.
  \begin{itemize}
    \item Le cas d'une variable, par exemple $x$, n'est pas possible car par
      hypothèse, le terme $t$ est clos. Or, $x$ est libre dans $x$.
    \item $t$ est une abstraction. Le résultat est direct car $t$ est une valeur.
    \item $t$ est de la forme $u \, v$. $t$ étant
      bien typé, nous avons le jugement de typage $\vdash u \, v : T$. Par le lemme
      d'inversion, $u : T_{1} \rightarrow T$ et $v : T_{1}$. Par induction,
      comme $u$ (resp. $v$) est bien typé, $u$ (resp. $v$) est soit une valeur,
      soit se réduit en un $u'$ (resp. $v'$).
      \begin{itemize}
        \item Si $u$ se réduit en $u'$, alors (E-APP1) s'applique et $u \, v$
          se réduit en $u' \, v$.
        \item Si $u$ est une valeur et $v$ se réduit en $v'$, alors (E-APP2)
          s'applique et $u \, v$ se réduit en $u \, v'$.
        \item Si $u$ et $v$ sont des valeurs, (E-APPABS) s'applique.
      \end{itemize}
  \end{itemize}
\end{proof}

\subsection*{Préservation}

\begin{lemma} [de permutation]
  \label{thm:simply-typed-lambda-calculus-permutation}
  Soit $\Gamma \vdash t : T$ et soit $\Delta$ une permutation de $\Gamma$. Alors
  $\Delta \vdash t : T$.
\end{lemma}

\begin{proof}
  \label{proof:simply-typed-lambda-calculus-permutation}
  Par induction sur l'arbre de dérivation de typage.
  \begin{itemize}
  \item (T-VAR). $t$ est une variable $x$. Par hypothèse, $\Gamma \vdash x : T$
    et $(x, T) \in \Gamma$. D'où $(x, T) \in \Delta$. Par (T-VAR), $\Delta \vdash x
    : T$.
  \item (T-ABS). Nous avons $t = \lambdaExpr{x : T_{1}} t'$ et
    \begin{mathpar}
      \inferrule*[Left=(T-ABS)]
      {\Gamma, x : T_{1} \vdash t' : T_{2}}
      {\Gamma \vdash \lambdaExpr{x : T_{1}} t' : T_{1} \rightarrow T_{2}}
    \end{mathpar}
      Par hypothèse de récurrence, $\Delta, x : T_{1} \vdash t' : T_{2}$. Par
(T-ABS), $\Delta \vdash \lambdaExpr{x : T_{1}}{t'} : T_{1} \rightarrow T_{2}$.
  \item (T-APP). Nous avons donc $t = u \, v$ et
    \begin{mathpar}
      \inferrule*[Left=(T-APP)]
      {\Gamma \vdash u : T_{1} \rightarrow T \\ \Gamma \vdash v : T_{1}}
      {\Gamma \vdash u \, v : T}
    \end{mathpar}
    Par hypothèse de récurrence, $\Delta \vdash u : T_{1} \rightarrow T$ et
    $\Delta \vdash v : T_{1}$. Par (T-APP), $\Delta \vdash u \, v : T$.
  \end{itemize}
\end{proof}

\begin{lemma} [d'affaiblissement]
  \label{thm:simply-typed-lambda-calculus-weakening}
  Soit $\Gamma \vdash t : T$ et $x \notin dom(\Gamma)$.

  Alors $\Gamma, x : S \vdash t : T$.\footnote{Nous pouvons généraliser le lemme
  à un contexte $\Gamma'$ dont le domaine est distinct de celui de $\Gamma$.}
\end{lemma}

\begin{proof}
  \label{proof:simply-typed-lambda-calculus-weakening}
  Par induction sur l'arbre de dérivation de typage.
  \begin{itemize}
  \item (T-VAR). $t = y$. Le cas où $y = x$ est impossible car nous avons $(x, T)
    \in \Gamma$ et cela contredit l'hypothèse que $x \notin dom(\Gamma)$. Si $y \neq
    x$, nous avons $(y, T) \in \Gamma$ et par conséquent, $(y, T) \in \Gamma, x : S$ et nous
    concluons en utilisant (T-VAR).
  \item (T-ABS). $t = \lambdaExpr{y : T_{1}}{t'}$. Nous avons $T = T_{1}
    \rightarrow T_{2}$ et $\Gamma, y : T_{1} \vdash t' : T_{2}$. Par hypothèse de
    récurrence, on a $\Gamma, y :
    T_{1}, x : S \vdash t' : T_{2}$. Par le lemme de permutation, nous avons
    $\Gamma, x : S, y : T_{1} \vdash t' : T_{2}$ et par (T-ABS), nous déduisons
    $\Gamma, x : S \vdash \lambdaExpr{y : T_{1}}{t'} : T_{1} \rightarrow T_{2}$.
    \item (T-APP). $t = u \, v$. Nous avons $\Gamma \vdash u : T_{1} \rightarrow
      T$ et $\Gamma \vdash v : T_{1}$. Par hypothèse de récurrence, $\Gamma, x : S
      \vdash u : T_{1} \rightarrow T$ et $\Gamma, x : S \vdash v : T_{1}$. Nous
      concluons que $\Gamma, x : S \vdash u \, v : T$ par (T-APP).
  \end{itemize}
\end{proof}

\begin{lemma} [de préservation du typage pour la substitution]
  \label{thm:simply-typed-lambda-calculus-preservation-substitution}
  Soit $\Gamma, x : S \vdash t : T$ et $\Gamma \vdash s : S$.

  Alors $\Gamma \vdash [x \rightarrow s] t : T$
\end{lemma}

\begin{proof}
  \label{proof:simply-typed-lambda-calculus-preservation-substitution}
  Nous procédons par une induction sur l'arbre de dérivation du jugement $\Gamma, x : S
  \vdash t : T$.

  \begin{itemize}
  \item (T-VAR). $t = z$. Alors, par le lemme d'inversion, $(z, T) \in \Gamma, x : S$.
    Deux cas sont possibles. Si $z = x$, alors $[x \rightarrow s] z = s$ ainsi que
    $S = T$ et nous obtenons le résultat souhaité. Si $z \neq x$,
    alors $[x \rightarrow s] z = z$ et il n'y a rien à montrer car $\Gamma
    \vdash z : T$.
  \item (T-ABS). $t = \lambdaExpr{y : T_{1}}{t'}$.
    Nous avons $T = T_{1} \rightarrow R$ avec
    $\Gamma, x : S, y : T_{1} \vdash t' : R$. 

    Sans perte de généralité, nous supposons $y \notin dom(\Gamma)$.
    Rappelons que par définition de la $\beta$-réduction,
    \begin{equation*}
      [x \rightarrow s]\lambdaExpr{y : T_{1}}{t'} = \lambdaExpr{y : T_{1}}[x \rightarrow s] t'
    \end{equation*}
    Par le lemme de permutation,
    nous avons également $\Gamma, y : T_{1}, x : S \vdash t' : R$. En utilisant le lemme
    d'affaiblissement avec $\Gamma \vdash s : S$, comme $y \notin dom(\Gamma)$,
    nous obtenons $\Gamma, y : T_{1} \vdash s : S$.
    Nous appliquons alors l'hypothèse de récurrence et nous obtenons
    $\Gamma, y : T_{1} \vdash [x \rightarrow s] t' : R$. Par (T-ABS), nous avons
    $\Gamma \vdash [x \rightarrow s] \lambdaExpr{y : T_{1}}{t'} : R$ .

    \item $t = u \, v$.
      Rappelons que par définition de la $\beta$-réduction,
      \begin{equation*}
        [x \rightarrow s](u \, v) = ([x \rightarrow s] u) \, ([x \rightarrow s] v)
      \end{equation*}
      Nous avons $\Gamma, x : S
      \vdash u : T_{1} \rightarrow T$ et $\Gamma, x : S \vdash v : T_{¡}$. Par
      hypothèse de récurrence, nous avons $\Gamma \vdash [x \rightarrow s]u :
      T_{1} \rightarrow T$ et
      $\Gamma \vdash [x \rightarrow s]v : T_{1}$. Par (T-APP), nous avons
      $\Gamma \vdash [x \rightarrow s](u \, v) : T$.
  \end{itemize}
  
\end{proof}

\begin{theorem} [de préservation du typage]
  \label{thm:simply-typed-lambda-calculus-preservation}
  Soit $\Gamma \vdash t : T$ et $t \rightarrow t'$. Alors $\Gamma \vdash t' :
  T$.
\end{theorem}

\begin{proof}
  \label{proof:simply-typed-lambda-calculus-preservation}
  Par induction sur l'arbre de dérivation de $\Gamma \vdash t : T$.
  \begin{itemize}
  \item (T-VAR). $t = x$. Ce cas n'est pas possible car aucune règle de
    réduction existe pour les variables.
  \item (T-ABS). $t = \lambdaExpr{x : T_{1}}{t_{2}} : T$. Même chose que pour le
    cas des variables.
  \item (T-APP). $t = u v$. Nous avons $\Gamma
    \vdash u : T_{1} \rightarrow T$ et $\Gamma \vdash v : T_{1}$. Plusieurs cas possibles:
    \begin{itemize}
    \item $u$ se réduit en $u'$. Alors, $t' = u' \, v$ par (E-APP1). Par hypothèse
      de récurrence, nous
      obtenons $\Gamma \vdash u' : T_{1} \rightarrow T$. Par (T-APP),
      $\Gamma \vdash u' \, v : T$.
    \item $v$ se réduit en $v'$ et $u$ est une valeur. Alors, $t' = u \, v'$ par
      (E-APP2). Nous appliquons alors le même argument que pour le cas précedent.
    \item $u$ et $v$ sont des valeurs. Posons $u = \lambdaExpr{x : T_{1}} t_{2}$.
      Alors $t' = [x \rightarrow v]t_{2}$ et nous avons $\Gamma \vdash v :
      T_{1}$. Nous avons $\Gamma, x : T_{1} \vdash t_{2}
      : T$. Par le lemme de
      substitution, nous concluons $\Gamma \vdash [x \rightarrow v]t_{2} : T$.
    \end{itemize}
  \end{itemize}
\end{proof}

\chapter{$\lambda$-calcul avec sous-typage et enregistrements.}
\label{chapter:lambda-calculus-with-records}

La syntaxe des termes du $\lambda$-calcul simplement typé est pauvre : nous ne pouvons définir que des
variables, des fonctions et appliquer des termes entre eux.
La plupart des langages de programmation fournissent diverses structures de
données comme les paires, les tuples, ou encore les enregistrements.

Un enregistrement est ensemble fini de couples $(l_{i}, t_{i})$, noté $\left\{
  l_{i} = t_{i} \right\}^{1 \leq i \leq n}$ où $l_{i}$ est
un label pour le terme $t_{i}$, les couples étant séparés par des points-virgules. Par exemple, nous pouvons représenter un point
d'un plan par ses coordonnées cartésiennes, nommées $x$ et $y$ et par les termes
$t_{x}$ et $t_{y}$ pour la valeur des coordonnées. Nous notons ce terme
$\left\{ x = t_{x} ; y = t_{y} \right\}$.
Il est également intéressant de pouvoir récupérer une des coordonnées d'un
point.
Nous ajoutons pour cela le terme $t.l$ qui permet de récupérer le label $l$ du
terme $t$.

Il nous faut également typer les enregistrements. Pour cela, nous ajoutons une
nouvelle syntaxe dans les types, noté $\left\{l_{i} : T_{i}\right\}^{1 \leq i
  \leq n}$ où $l_{i}$
est un label de l'enregistrement et $T_{i}$ le type du terme référencé par le
label $l_{i}$. Pour l'exemple du point dans le plan, si nous supposons avoir un
type $\real$ pour les réels, notre point serait de type $\left\{ x : \real ; y : \real
\right\}$. Pour le typage des projections, il est naturel de dire que le type de
$t.l_{i}$ soit le type du terme $t_{i}$ de l'enregistrement $t$.

Dans ce chapitre, nous allons étendre notre ensemble de termes avec les
enregistrements ainsi que définir les règles d'évaluation et de typage pour ces
nouveaux termes pour enfin introduire dans un second temps la notion de
sous-typage qui définit le principe du \textit{polymorphisme par sous-typage}.

\section{$\lambda$-calcul simplement typé avec enregistrements}

\section*{Syntaxe}

Formellement, la syntaxe des termes et la syntaxe des types sont définies par
les grammaires suivantes :

\begin{minipage}{0.45\textwidth}
  \begin{align*}
    t ::= & \, & \text{terme} \\
          & \; x & \text{var} \\
          & \; t \, t & \text{app} \\
          & \; \lambdaExpr{x}{t} & \text{abs} \\
          & \; \left\{ l_{i} = t_{i} \right\}^{1 \leq i \leq n} & \text{enreg} \\
          & \; t.l & \text{proj}
  \end{align*}
\end{minipage}
\label{syntax-terms:lambda-calculus-with-records}
\begin{minipage}{0.45\textwidth}
  \begin{align*}
    T ::= & \, & \text{type} \\
          & \; b & \text{type de base} \\
          & \; t \rightarrow t & \text{fonction} \\
          & \; \left\{ l_{i} : t_{i} \right\}^{1 \leq i \leq n} & \text{enreg}
  \end{align*}
\end{minipage}
\label{syntax-types:lambda-calculus-with-records}

Nous allons également ajouter les enregistrements dont tous les termes sont des
valeurs comme valeurs de notre langage. Nous obtenons alors la grammaire
suivante :

\label{syntax-values:lambda-calculus-with-records}
\begin{align*}
  v ::= & \, & \text{valeur} \\
        & \; \lambdaExpr{x}{t} & \text{abs} \\
          & \; \left\{ l_{i} = v_{i} \right\}^{1 \leq i \leq n} & \text{enreg} \\
\end{align*}

\section*{Sémantique}

Il nous faut également définir comment nous réduisons nos
enregistrements. Nous ajoutons les règles d'évaluation suivantes aux règles
d'évaluation définies dans les chapitres précédents.

\begin{mathpar}
  \inferrule
  {t_{j} \rightarrow t'_{j}}
  {\left\{ l_{1} = v_{1}; ... ; l_{j} = t_{j} ; ... ; l_{n} = v_{n} \right\}
    \rightarrow \\\\ \left\{ l_{1} = v_{1}; ... ; l_{j} = t'_{j} ; ... ; l_{n} =
      v_{n} \right\}} \quad (\textsc{E-RCD})
  \and
  \inferrule
  {t_{1} \rightarrow t'_{1}}
  {t_{1}.l \rightarrow t'_{1}.l} \quad (\textsc{E-PROJ})
  \and
  \inferrule
    {\left\{l_{i} = v_{i} \right\}^{1 \leq i \leq n}.l_{j} \rightarrow v_{j}}
    {} \quad (\textsc{E-PROJ-RCD})
\end{mathpar}
\label{semantics:lambda-calculus-with-records}

La règle (E-RCD) nous dit comment les termes à l'intérieur d'un enregistrement sont
évalués. Quant à (E-PROJ-RCD), elle nous dit que nous pouvons
évaluer une projection uniquement si les termes de l'enregistrement ont tous été
réduits à des valeurs. Pour finir, (E-PROJ) nous dit comment nous simplifions
le terme dans une projection.

\section*{Règles de typage}

En plus des règles de typages du $\lambda$-calcul simplement typé, la relation
de typage comprend les règles suivantes :

\begin{mathpar}
  \inferrule
  {\forall i \in \left\{1, ..., n \right\}, \Gamma \vdash t_{i} : T_{i}}
  {\Gamma \vdash \left\{ l_{i} = t_{i} \right\}^{1 \leq i \leq n} : \left\{ l_{i} : T_{i}
    \right\}^{1 \leq i \leq n}}
  \quad (\textsc{T-RCD})
  \and
  \inferrule
  {\Gamma \vdash t_{1} : \left\{ l_{i} : T_{i} \right\}^{1 \leq i \leq n}}
  {\Gamma \vdash t_{1}.l_{j} : T_{j} } 
  \quad (\textsc{T-PROJ})
\end{mathpar}
\label{typing:lambda-calculus-with-records}

La règle (T-RCD) nous dit comment introduire un type enregistrement tandis que
(T-PROJ) nous dit comment typer une projection.

\section{Sous-typage}

Maintenant que nous avons défini la syntaxe, la sémantique et les règles de
typages de notre langage comprenant les enregistrements, nous allons définir la
notion de sous-typage ainsi que les règles pour ce langage.

Le sous-typage est une forme de polymorphisme, c'est-à-dire une possibilité
d'attribuer plusieurs types à un terme. Le polymorphisme est très courant dans
les langages orientés objets et permet, par exemple, d'élargir le champ
d'application des fonctions.
De manière plus concrêtes, supposons que nous avons défini un type $\real$
pour l'ensemble des réels et définissons le point $(5, 5)$ du plan $\real^{2}$ par
$\left\{ x = 5 ; y = 5 \right\}$\footnote{En supposant que 5 fasse partie de
  notre syntaxe et que son type soit $\real$.}.
Nous pouvons définir la fonction de projection sur l'axe des abscisses avec
\begin{equation*}
  \lambdaExpr{p : \left\{ x : \real ; y : \real \right\}}{p.x}.
\end{equation*}
Maintenant, nous souhaitons faire de même pour le point $(5, 5, 5)$ de $\real^{3}$,
représenté par $\left\{ x = 5 ; y = 5 ; z = 5 \right\}$.
Nous ne pouvons pas utiliser la fonction de projection définie pour $\real^{2}$
parce que le paramètre de la fonction est un enregistrement ne contenant que les
champs $x$ et $y$. Nous devons donc créer une nouvelle fonction, par exemple 
\begin{equation*}
  \lambdaExpr{p : \left\{ x : \real ; y : \real ; z : \real \right\}}{p.x}.
\end{equation*}
Et si nous voulions continuer pour $\real^{4}$, $\real^{5}$, etc, nous devrions à chaque
fois redéfinir une nouvelle fonction. Cependant, nous remarquons que le corps de
la fonction est toujours le même et qu'il n'a besoin que d'un enregistrement
avec au moins un champ $x$, les autres champs étant inutiles.


C'est dans ce cas que le sous-typage intervient : nous allons définir une
relation entre les types qui permet d'affiner les règles de typage. Nous dirons que
\textbf{$S$ est un sous-type de $T$} ou encore \textbf{$T$ est un supertype de $S$}, noté $S <: T$.

Pour lier la relation de typage à la relation de sous-typage, nous ajoutons une
nouvelle règle de typage

\begin{mathpar}
  \inferrule
  {\Gamma \vdash t : S \\ S <: T}
  {\Gamma \vdash t : T} \quad (\textsc{T-SUB})
\end{mathpar}

Cette dernière permet d'affirmer que si dans un contexte donné, un terme $t$ a le
type $S$ et que le type $S$ est un sous-type de $T$, alors dans le même
contexte, $t$ a le type $T$.

\subsection*{Règles de sous-typage}

Passons à la définition de la relation de sous-typage. La toute première chose
est que nous voulons que cette relation soit réflexive et transitive comme c'est
le cas dans la plupart des langages:

\begin{mathpar}
  \inferrule
  {S <: S}{} \quad (\textsc{S-REFL})
  \and
  \inferrule
  {S <: T \\ T <: U}{S <: U} \quad (\textsc{S-TRANS})
\end{mathpar}

Ensuite, nous aimerions que la relation résolve notre problème, c'est-à-dire que
nous devons pouvoir affirmer qu'un type ayant les champs $x, y, z$
est un sous-type du type ayant uniquement le champ $x$ ou uniquement le champ
$y$. Nous résumons cela par les deux règles suivantes:

\begin{mathpar}
  \inferrule
  {\left\{ l_{i} : T_{i} \right\}^{1 \leq i \leq n + k} <: \left\{ l_{i} :
      T_{i} \right\}^{1 \leq i \leq n}}
  {}
    \quad (\textsc{S-RCD-WIDTH})
  \and
  \inferrule
  {\left\{ k_{j} : s_{j} \right\}^{1 \leq j \leq n} \text{ permutation de }
    \left\{ l_{i} : t_{i} \right\} ^ {1 \leq i \leq n}}
  { \left\{ k_{j} : s_{j} \right\} ^ {1 \leq j \leq n} <:
    \left\{ l_{i} : t_{i} \right\} ^ {1 \leq i \leq n}}
  \quad (\textsc{S-RCD-PERM})
\end{mathpar}

(S-RCD-WIDTH) nous permet de \og laisser de coté \fg \, certain champ tandis que
(S-RCD-PERM) nous dit que l'ordre des champs dans un enregistrement n'a pas d'importance.

Nous souhaitons également prendre en
compte les types dans les enregistrements et leur relation de sous-typages entre
eux. Par exemple, nous aimerions qu'un point du plan $\naturel^{2}$ soit
également un point du plan $\real^{2}$.

\begin{mathpar}
  \inferrule
  {\forall i \in \GSset{1, ..., n}, S_{i} <: T_{i}}
  {\left\{ l_{i} : S_{i} \right\}^{1 \leq i \leq n} <: \left\{ l_{i} : T_{i}
    \right\}^{1 \leq i \leq n}}
  \quad (\textsc{S-RCD-DEPTH})
\end{mathpar}

(S-RCD-DEPTH) nous dit que, étant donnés deux types enregistrements $S$ et $T$
avec les mêmes labels,
si les types des champs de $S$ sont tous sous-types des types des champs de $T$,
alors $S$ est sous-type de $T$.

Pour finir, nous aimerions dire que si nous avons une fonction $f$ qui attend un
enregistrement avec un champ $x$, nous pouvons également passer en paramètre à
$f$ un enregistrement avec deux champs $x$ et $y$. Nous ajoutons en plus que le
type de retour peut être un super-type.

\begin{mathpar}
  \inferrule
  {T_{1} <: S_{1} \\ S_{2} <: T_{2}}
  {S_{1} \rightarrow S_{2} <: T_{1} \rightarrow T_{2} }
  \quad (\textsc{S-ARROW})
\end{mathpar}

Nous avons maintenant toutes les propriétés nécessaires pour résoudre notre
problème. En effet, voici un arbre de dérivation qui permet de montrer
$(\lambdaExpr{p : \left\{ x : R \right\}}{p.x}) \left\{ x = 5 ; y =
  5 \right\} : R$.

\begin{mathpar}
\inferrule* [Left=(T-APP)]
{\Gamma \vdash \lambdaExpr{p : \left\{ x : R \right\}}{p.x} : \left\{ x : R
  \right\} \rightarrow R
  \\
  \inferrule* [Right=(T-SUB)]
  {\Gamma \vdash \left\{ x = 5 ; y = 5 \right\} : \left\{ x : R ; y : R \right\}
    \\\\
    \left\{ x : R ; y : R \right\} <: \left\{ x : R \right\}
  }
  {\Gamma \vdash \left\{ x = 5 ; y = 5 \right\} : \left\{ x : R \right\}}
}
{\Gamma \vdash (\lambdaExpr{p : \left\{ x : R \right\}}{p.x}) \left\{ x = 5 ; y =
  5 \right\} : R}
\end{mathpar}

L'affirmation \inferrule{\left\{ x : R ; y : R \right\} <: \left\{ x : R
  \right\}}{} est inférée grâce à (S-RCD-WIDTH).

\section{Sûreté}

Nous allons montrer que les théorèmes de préservation et de progression
démontrés pour le $\lambda$-calcul simplement typé restent vrais en présence
d'enregistrements et du sous-typage.

Les techniques de preuves utilisées et leur structure ne sont pas très
différentes de celles employées dans le chapitre précédent. Pour ces raisons,
les preuves seront moins détaillées.

\subsection*{Préservation}

\begin{lemma} [Inversion de la règle de sous-typage]
  \label{lemma:record-subtyping-inversion-subtyping-rules}
  \begin{enumerate}
    \item Si $S <: T_{1} \rightarrow T_{2}$, alors $S$ est de la forme $S_{1}
      \rightarrow S_{2}$ avec $T_{1} <: S_{1}$ et $S_{2} \rightarrow T_{2}$.
    \item Si $S <: \left\{ l_{i} : T_{i} \right\}^{1 \leq i \leq n}$, alors $S$
      est de la forme $\left\{ k_{i} : S_{i} \right\}^{1 \leq i \leq m}$ tel que
      $(k_{i})_{1 \leq i \leq m} \subseteq (l_{i})_{1 \leq i \leq n}$ et $S_{i}
      <: T_{i}$ pour chaque $i$ tel que $l_{i} = k_{i}$.
  \end{enumerate}
\end{lemma}

\begin{proof}
  La technique de preuve reste identique : pour chaque affirmation, nous
  cherchons quelle règle peut y avoir mené et nous montrons cas par cas, en
  utilisant le principe d'induction sur l'arbre de dérivation. Seul le
  premier point est démontré, le second étant identique. Pour le premier point,
  seules les règles (S-REFL), (S-ARROW) et (S-TRANS) sont possibles, les autres
  étant pour les enregistrements.

  \begin{itemize}
  \item (S-REFL). Le résultat est direct car $S = T_{1} \rightarrow T_{2}$.
  \item (S-ARROW). Le résultat est également direct.
  \item (S-TRANS). Nous avons donc l'arbre de dérivation suivant:
    \begin{mathpar}
      \inferrule
      {S <: T \\ T <: T_{1} \rightarrow T_{2}}
      {S <: T_{1} \rightarrow T_{2}}
    \end{mathpar}
    Nous appliquons d'abord l'hypothèse de récurrence sur l'affirmation $T <:
    T_{1} \rightarrow T_{2}$ et ensuite sur l'affirmation $S <: T$.
  \end{itemize}
\end{proof}

Nous montrons également un lemme d'inversion des règles de typage comme il a été
démontré pour le $\lambda$-calcul simplementé typé.

\begin{lemma} [d'inversion des règles de typage]
  \label{lemma:subtyping-record-inversion-typing-rules}
  \begin{enumerate}
    \item Si $\Gamma \vdash \lambdaExpr{x : S_{1}} s_{2} : T_{1} \rightarrow
    T_{2}$, alors $T_{1} <: S_{1}$ et $\Gamma, x : S_{1} \vdash s_{2} : T_{2}$.
    \item Si $\Gamma \vdash \left\{ k_{i} = s_{i} \right\}^{1 \leq i \leq n} :
      \left\{ l_{i} : T_{i} \right\}^{1 \leq i \leq m}$, alors $(l_{i})^{1 \leq
        i \leq m} \subseteq (k_{i})^{1 \leq i \leq n}$ et $\Gamma \vdash s_{i} :
      T_{i}$ pour $k_{i} = l_{i}$.
  \end{enumerate}
\end{lemma}

\begin{proof}
  Par induction sur l'arbre de typage de $\Gamma \vdash t : T$.
  Encore une fois, nous ne montrons que pour le premier cas, les arguments étant
  sensiblement les mêmes pour le second.

  Les seuls règles possibles sont T-SUB ou T-ABS.

  \begin{enumerate}
    \item T-SUB. Nous avons donc l'arbre de dérivation suivant:
      \begin{mathpar}
        \inferrule* [Left=(T-SUB)]
        {\Gamma \vdash \lambdaExpr{x : S_{1}}{s_{2}} : T \\ T <: T_{1}
          \rightarrow T_{2}}
        {\Gamma \vdash \lambdaExpr{x : S_{1}}{s_{2}} : T_{1} \rightarrow T_{2}}
      \end{mathpar}
      En appliquant le lemme \ref{lemma:record-subtyping-inversion-subtyping-rules}
      sur l'affirmation $T <: T_{1} \rightarrow T_{2}$, nous obtenons $T =
      \tilde{S_{1}} \rightarrow \tilde{S_{2}}$ avec $T_{1} <: \tilde{S_{1}}$ et $\tilde{S_{2}} <: T_{2}$.
      L'affirmation de gauche devient donc
      \begin{mathpar}
        \inferrule
        {\Gamma \vdash \lambdaExpr{x : S_{1}}{s_{2}} : \tilde{S_{1}} \rightarrow \tilde{S_{2}}}
        {}
      \end{mathpar}
      En appliquant l'hypothèse de récurrence sur ce jugement de typage, nous
      obtenons $\tilde{S_{1}} <: S_{1}$ et $\Gamma, x : S_{1} \vdash s_{2} :
      \tilde{S_{2}}$.
      
      En utilisant S-TRANS avec $\tilde{S_{1}} <: S_{1}$ et $T_{1}
      <: \tilde{S_{1}}$, nous concluons avec $T_{1} <: S_{1}$.
      Pour finir, en utilisant T-SUB avec $\Gamma, x : S_{1} \vdash s_{2} :
      \tilde{S_{2}}$ et $\tilde{S_{2}} <: T_{2}$, nous obtenons $\Gamma, x : S_{1}
      \vdash s_{2} : T_{2}$.
      \item T-ABS. Le résultat est direct et $T_{1} = S_{1}$.
  \end{enumerate}
\end{proof}

Nous pouvons maintenant démontrer le même lemme de substitution défini dans le
chapitre précédent.

\begin{lemma} [de préservation de typage par substitution]
  \label{lemma:subtyping-record-substitution}
  Soit $\Gamma, x : S \vdash t : T$ et $\Gamma \vdash s : S$.

  Alors $\Gamma \vdash [x \rightarrow s] t : T$
\end{lemma}

\begin{proof}
  Même technique de preuve que dans le chapitre précédent. Il suffit d'ajouter 
  des cas pour les enregistrements et les projections en utilisant
  respectivement T-RCD et T-PROJ et de manière générale pour T-SUB (hypothèse de
  récurrence sur le jugement de gauche et utilisation de T-SUB pour conclure).
\end{proof}

Et nous concluons avec le théorème de progression.

\begin{theorem} [de préservation du typage]
  Si $t$ est un terme bien typé sans variable libre, alors soit $t$ est une
  valeur, soit il existe $t'$ tel que $t \rightarrow t'$.
\end{theorem}

\begin{proof}
  Nous procédons de la même manière que pour le cas du $\lambda$-calcul
  simplement typé, c'est-à-dire sur le jugement $\Gamma \vdash t : T$.
  \begin{enumerate}
    \item T-VAR. Nous avons alors $t = x$ : pas possible car il n'y a pas de
      réduction pour les variables.
      \item T-ABS. $t = \lambdaExpr{x : T_{1}}{t'}$ : déjà une valeur.
      \item T-APP. $t = t_{1} \, t_{2}$. Nous avons
          $\Gamma \vdash t_{1} : T_{1} \rightarrow T$, $\Gamma \vdash t_{2}
          : T_{1}$.
          Les possibles règles d'évaluation
          sont E-APP1, E-APP2 et E-APPABS. Pour E-APP1 et E-APP2, même
          raisonnement que pour le $\lambda$-calcul simplement typé.

          Quant à E-APPABS, nous obtenons $t_{1} = \lambdaExpr{x : S_{1}}{t'}$ et
          $t_{2} = v$. En utilisant \ref{lemma:subtyping-record-substitution},
          nous déduisons $T_{1} <: S_{1}$ et $\Gamma, x : S \vdash t' : T$. Nous
          concluons avec le lemme de substitution qui donne $\Gamma, x : S \vdash
          [x \rightarrow v]t' : T$.

       \item T-RCD. La seule règle d'évaluation est E-RCD qui réduit un des
         termes de l'enregistrement. Il suffit d'appliquer l'hypothèse de
         récurrence sur ce terme et d'utiliser T-RCD pour conclure.

         \item T-PROJ. Nous obtenons $t = t_{1}.l_{j}$, $\Gamma \vdash t_{1}
           : \left\{ l_{i} : T_{i} \right\}^{1 \leq i \leq n}$ et $T = T_{j}$.
           Deux règles d'évaluation sont possibles : E-PROJ ou E-PROJ-RCD. Si
           c'est E-PROJ, nous appliquons l'hypothèse de récurrence. Sinon,
           E-PROJ-RCD nous dit que $t_{1} = \left\{ k_{i} = v_{i} \right\}^{1
             \leq i \leq m}$ et par le lemme
           \ref{lemma:record-subtyping-inversion-subtyping-rules}, nous avons
           $(l_{i})_{1 \leq i \leq n} \subseteq (k_{i})_{1 \leq i \leq m}$ et
           $\Gamma \vdash v_{i} : T_{i}$ pour $k_{i} = l_{i}$. Nous concluons
           que $\Gamma \vdash v_{j} : T_{j}$.
           \item T-SUB. Par hypothèse de récurrence et en utilisant T-SUB.
  \end{enumerate}
\end{proof}

\subsection*{Progression}

Nous démontrons avant un lemme appelé \textit{lemme des formes canoniques}.

\begin{lemma} [des formes canoniques]
  Soit $v$ une valeur sans variable libre.
  \begin{enumerate}
  \item Si $v$ est de type $T_{1} \rightarrow T_{2}$, alors $v$ est de la forme $\lambdaExpr{x : S_{1}}{t}$.
    \item Si $v$ est de type $\left\{ l_{i} : T_{i} \right\}^{1 \leq i \leq n}$,
      alors $v$ est de la forme $\left\{ k_{i} = v_{i} \right\}^{1 \leq i \leq
        m}$ où $(l_{i})_{1 \leq i \leq n} \subseteq
      (k_{i})_{1 \leq i \leq m}$.
  \end{enumerate}
\end{lemma}

\begin{proof}
  
\end{proof}

\begin{theorem} [de progression] 
  Soit $t$ un terme bien typé sans variable libre. Alors soit $t$ est une
  valeur, soit $t$ se réduit en $t'$
\end{theorem}

\begin{proof}
  Par induction sur l'arbre de dérivation de typage de $t$.
\end{proof}

\section{Type Top et type Bottom}

Finalement, il
est courant d'ajouter dans les types un type qui est super-type de tous les
autres types, souvent appelé \verb|Top|\footnote{En Java, le type Top
  est représenté par la classe Object.} ainsi qu'un type qui est sous-type de
tous les autres types, souvent appelé \verb|Bottom|.

La syntaxe des types est donc finalement

\begin{align*}
  T ::= & \, & \text{type} \\
        & \; b & \text{type de base} \\
        & \; t \rightarrow t & \text{fonction} \\
        & \; \left\{ l_{i} : t_{i} \right\}^{1 \leq i \leq n} & \text{enreg} \\
        & \; Top & \text{Top} \\
        & \; Bottom & \text{Bottom} \\
\end{align*}

et nous ajoutons les règles de typage suivantes.

\begin{mathpar}
  \inferrule* [Left=(S-TOP)] {T <: Top}{}
  \and
  \inferrule* [Right=(S-BOTTOM)] {Bottom <: T}{}
\end{mathpar}


\chapter{System F}

\section{Sureté de System F}
\chapter{Système $F_{<:}$}
\label{chapter:system-f-sub}

Dans les chapitres \ref{chapter:lambda-calculus-with-records} et
\ref{chapter:system-f}, nous avons défini deux notions de polymorphisme: le
polymorphisme par sous-typage à travers les enregistrements et le polymorphisme
paramétré.

Les deux mécanismes permettent de résoudre deux problèmes différents :
\begin{enumerate}
  \item le sous-typage permet d'affiner notre relation de typage à travers la
    relation $<:$. Par exemple, il nous est permis de définir la fonction
    identité sur les réels ($\lambdaExpr{x : \real} x$) et de l'appliquer à un
    entier car $\naturel$ est un sous-type de $\real$.
  \item le polymorphisme paramétré permet de quantifier sur les types et de
    créer des abstractions de type. Par exemple, nous pouvons
    définir la fonction identité polymorphe $\lambdaExprType{X}{\lambdaExpr{x :
    X}{x}}$ et l'appliquer aux types $\real$ et $\naturel$, mais également à
    n'importe quel autre type comme $\naturel \rightarrow \real$ ou encore $\real
    \rightarrow (\naturel \rightarrow \real)$.
\end{enumerate}

Dans ce chapitre, nous allons unifier ces deux notions en un langage appelé
Système $F_{<:}$\footnote{prononcé \og système F sub \fg}. L'idée principale de
Système $F_{<:}$ est d'élargir notre relation de sous-typage sur les variables de
type pour les borner et ainsi restreindre les types qui peuvent être appliqués
aux abstractions de type. De cette façon, une même abstraction de type dépendante d'une
variable $X$, bornée supérieurement par $\real$, pourra accepter les types $\naturel$ et
$\real$, mais pas $\naturel \rightarrow \real$ ou encore $\real \rightarrow
\real$. Nous ne considérons donc plus seules les variables de type, mais liées
à une borne supérieure. Nous notons $X <: T$ pour dire que la variable de type
$X$ est borné supérieurement par $T$.  

Ajouter une borne supérieure permet d'affiner le type de la variable $X$ en
obligeant le type appliqué à avoir certaines caractéristiques. Par exemple, la
fonction $\lambdaExprType{X <: \left\{ z : \real \right\}}{\lambdaExpr{x :
    X}{\, x.z}}$ oblige, lors d'une application de type, de donner un type
enregistrement avec au moins le champ $z$ qui est de type réel.

Il est naturel de se demander pourquoi nous ne donner qu'une borne
supérieure aux variables. Nous verrons que DOT leur attribue également une borne inférieure.

Dans ce chapitre, nous considérons également le type \verb|Top|.

\section{Syntaxe}

La syntaxe de Système $F_{<:}$ est celle de Système $F$ à laquelle nous ajoutons
une borne supérieure aux variables des abstractions de type :

\begin{minipage}{0.45\textwidth}
  \begin{align*}
    t ::= & \, & \text{terme} \\
          & \; x & \text{var} \\
          & \; t \, t & \text{app} \\
          & \; \lambdaExpr{x : T}{t} & \text{abs} \\
          & \; \lambdaExprType{X <: T}{t} & \text{type abs} \\
          & \; t[T] & \text{type app}
  \end{align*}
\end{minipage}
\begin{minipage}{0.45\textwidth}
  \begin{align*}
    T ::= & \, & \text{type} \\
          & \; X & \text{type var} \\
          & \; T \rightarrow T & \text{fonction} \\
          & \; \forall X <: T. \, T & \text{universel} \\
          & \; Top & \text{Top}
  \end{align*}
\end{minipage}
\\
\\
\\
Les valeurs restent les mêmes :
\begin{align*}
  v ::= & \, & \text{valeur} \\
        & \; \lambdaExpr{x : T}{t} & \text{abs} \\
        & \; \lambdaExprType{X <: T}{t} & \text{type abs}
\end{align*}

\section{Sémantique}

Au niveau de la sémantique, les règles restent les mêmes que pour Système F :

\begin{mathpar}
  \inferrule
    {t_{1} \rightarrow t_{1}'}
    {t_{1} \, t \rightarrow t_{1}' \, t} \quad (\textsc{E-APP1})
  \and
  \inferrule
    {t \rightarrow t'}
    {v \, t \rightarrow v \, t'} \quad (\textsc{E-APP2})
  \and
  \inferrule
    {(\lambdaExpr{x : T}{t}) v \rightarrow [x := v]t}
    {} \quad (\textsc{E-APPABS})
  \and
  \inferrule
  {t \rightarrow t'}
  {t[T] \rightarrow t'[T]} \quad (\textsc{E-T-APP})
  \and
  \inferrule
  {(\lambdaExprType{X <: T_{1}}{t})[T] \rightarrow [X \rightarrow T]t}
  {} \quad (\textsc{E-T-ABS})
\end{mathpar}
\label{eval:system-f-sub}

\section{Contexte de typage}

Au niveau du contexte de typage, à la place d'avoir uniquement la variable de
type, nous allons également ajouter la borne supérieure.

Il est nécessaire de faire attention à l'ordre d'apparition des variables. Par
exemple, nous dirons que $X <: Y, Y <: Top$ est mal formé car $Y$
apparaît comme borne avant sa définition ($Y <: Top$). Bien que ça soit une
définition informelle, nous considérerons uniquement les contextes de typage
tels qu'une variable $Y$ n'apparaît jamais comme borne supérieure à gauche de sa
définition.

La syntaxe du contexte de typage est donc :
\begin{align*}
  \Gamma ::= & \, & \text{contexte} \\
        & \; \emptyset & \\
        & \; \Gamma, x : T & \\
        & \; \Gamma, X <: T & \\
\end{align*}

\section{Règles de typage et de sous-typage}

Comme le contexte comporte des hypothèses de sous-typage, nous ajoutons un
contexte $\Gamma$ dans les règles de sous-typage. Une affirmation $S <: T$
devient $\Gamma \vdash S <: T$.

Les règles de typage deviennent :

\begin{mathpar}
  \inferrule
  {(x : T) \in \Gamma}
  {\Gamma \vdash x : T}
  \quad (\textsc{T-VAR})
  \and
  \inferrule
  {\Gamma, x : T_{1} \vdash t : T_{2}}
  {\Gamma \vdash \lambda (x : T_{1}) t : T_{1} \rightarrow T_{2}}
  \quad (\textsc{T-ABS})
  \and
  \inferrule
  {\Gamma \vdash u : T_{1} \rightarrow T_{2} \\ \Gamma \vdash v : T_{1}}
  {\Gamma \vdash u \, v : T_{2}}
  \quad (\textsc{T-APP})
  \\
  \inferrule
  {\Gamma, X <: T_{1} \vdash t : T}
  {\Gamma \vdash \lambdaExprType{X <: T_{1}}{t} : \forall X <: T_{1}. \, T}
  \quad (\textsc{T-T-ABS})
  \and
  \inferrule
  {\Gamma \vdash t_{1} : \forall X <: T_{1}. \, T \\ \Gamma \vdash T' <: T_{1}}
  {\Gamma \vdash t_{1}[T'] : [X \rightarrow T']T}

  \quad (\textsc{T-T-APP})
  \and
  \inferrule
  {\Gamma \vdash t : S \\ \Gamma \vdash S <: T}
  {\Gamma \vdash t : T}
  \quad (\textsc{T-SUB})
\end{mathpar}

Quant aux règles de sous-typage, nous avons :

\begin{mathpar}
  \inferrule
  {\Gamma \vdash S <: S}
  {}
  \quad (\textsc{S-REFL})
  \and
  \inferrule
  {\Gamma \vdash S <: T \\ \Gamma \vdash T <: U}
  {\Gamma \vdash S <: U}
  \quad (\textsc{S-TRANS})
  \and
  \inferrule
  {\Gamma \vdash S <: Top}
  {}
  \quad (\textsc{S-TOP})
  \\
  \inferrule
  {X <: T \in \Gamma}
  {\Gamma \vdash X <: T}
  \quad (\textsc{S-TVAR})
  \and
  \inferrule
  {\Gamma \vdash T_{1} <: S_{1} \\ \Gamma \vdash S_{2} <: T_{2}}
  {\Gamma \vdash S_{1} \rightarrow S_{2} <: T_{1} \rightarrow T_{2}}
  \quad (\textsc{S-ARROW})
  \and
  \inferrule
  {\Gamma \vdash T_{1} <: S_{1} \\ \Gamma, X <: T_{1} \vdash S_{2} <: T_{2}}
  {\Gamma \vdash \forall X <: S_{1}. S_{2} <: \forall X <: T_{1}. T_{2}}
  \quad (\textsc{S-ALL})
\end{mathpar}

Remarquons que dans (S-ALL), la borne supérieur de la variable $X$ peut être
différente.\footnote{Une variante, appelée Kernel $F_{<:}$, oblige la borne
  supérieure à être la même. Nous ne discuterons pas de celle-ci dans ce document.}

\section{Sûreté}

Les théorèmes de préservation et de progression restent vrais pour Système
$F_{<:}$. Cependant, nous ne les démontrerons pas car ils nécessitent des lemmes
techniques ainsi qu'une définition formelle de contexte bien formé et ce n'est
pas le sujet principal de ce document. La structure
et les techniques restent les mêmes : lemme d'inversion des relations de typage,
lemme des formes canoniques, lemme d'inversion de la relation de sous-typage, lemme de
substitution des types, preuve par induction structurelle, etc.
Des preuves des théorèmes peuvent être trouvées dans
\cite{tapl-bounded-quantification}.

\section{Indécidabilité du sous-typage}

Étant donnés deux types $S$ et $T$, la question $S <: T$ a-t-elle
toujours une réponse ? En d'autres termes, la question du sous-typage est-elle
décidable ?

Il a été démontré qu'il n'existe pas d'algorithme correct et complet de sous-typage donnant une
réponse sur toute question $S <: T$.
Un algorithme $A(\Gamma, S, T)$ est \textit{correct} si lorsque $A(\Gamma, S,
T)$ a comme sortie oui, $\Gamma \vdash S <: T$. Un algorithme est
\textit{complet} si, lorsque $\Gamma \vdash S <: T$, la sortie de $A(\Gamma, S,
T)$ est oui.

Plus d'informations peuvent être trouvées
dans \cite{tapl-bounded-quantification-metatheory}.

Le problème de l'indécidabilité du sous-typage de Système $F_{<:}$ est important
quand nous considérons l'écriture d'un algorithme de sous-typage pour ce calcul.
En effet, cela implique que sur certaines entrées, l'algorithme peut
diverger\footnote{Si bien sûr nous ne définissons pas des règles pour l'arrêter
  dans le cas où il n'y a pas de réponse.}.

\chapter{DOT}

%%% TODO: x peut apparaitre dans son type.

Dans ce chapitre, nous présentons DOT, un calcul développé récemment pour le
langage Scala. Ce calcul ajoute les types, alors
appelés types dépendants, dans les enregistrements. Des types
récursifs, c'est-à-dire des types qui peuvent faire référence à eux-mêmes, sont
également définis\footnote{Les types récursifs et leurs règles présents dans DOT ne sont pas
  définis de manière habituelle. Plusieurs chapitres y sont consacrés dans
  \cite{tapl-recursive-types}. Nous n'étudierons pas les différences dans ce document.
}.
Nous montrerons que DOT peut être vu comme une
extension de Système $F_{<:}$ bien que sa syntaxe soit différente et ne possède
pas de variables de type.

Plusieurs définitions du calcul DOT existent et sont dispersées à travers
plusieurs documents comme \cite{nada-amin-thesis}, \cite{OOPSLA-DOT-2016},
\cite{POPL-2017-DOT} ou encore \cite{WF-DOT-2016}. Dans ce document, nous avons
fait le choix d'utiliser \cite{WF-DOT-2016} car la syntaxe et les règles sont
proches des calculs présentés dans ce mémoire.

\section{Syntaxe}

La syntaxe des termes de DOT est définie par la grammaire suivante :
\begin{minipage}{0.45\textwidth}
  \begin{align*}
    t ::= & \, & \text{terme} \\
          & \; x, y & \text{var} \\
          & \; \lambdaExpr{x : T}{t} & \text{abs} \\
          & \; x \, y & \text{app} \\
          & \; \localLetBinding{x}{t}{t} & \text{let} \\
          & \; \nu(x : T^{x})d & \text{rec} \\
          & \; x.a & \text{champ proj} \\
\end{align*}
\end{minipage}
\begin{minipage}{0.45\textwidth}
  \begin{align*}
    d ::= & \, & \text{decl} \\
          & \; \left\{ a = t \right\} & \text{champ} \\
          & \; \left\{ A = T \right\} & \text{type} \\
          & \; d \wedge d & \text{aggregation}
\end{align*}
\end{minipage}

et la syntaxe des types par la grammaire suivante :

\begin{align*}
  S, T ::= & \, & \text{type} \\
           & \; Top & \text{top} \\
           & \; Bottom & \text{bottom} \\
           & \; \forall(x : S) T^{x} & \text{fonction} \\
           & \; \left\{ A : S .. T \right\} & \text{type decl} \\
           & \; \left\{ a : T \right\} & \text{champ decl} \\
           & \; x.A & \text{type proj} \\
           & \; \mu(x : T^{x}) & \text{rec} \\
           & \; S \wedge T & \text{inter}
\end{align*}

Nous retrouvons (var) et (abs) pour les variables et les
abstractions.
À la différence des autres calculs, une application n'est pas constituée de deux
termes mais de deux variables\footnote{Nous pouvons cependant utiliser (let)
  pour retrouver des applications entre termes.}. Nous retrouvons la projection d'un champ avec
(champ proj). Comme pour les applications, seule une variable peut être
utilisée\footnote{Même remarque que pour les applications.}.

(let) permet de créer des variables locales comme il est possible de le faire en OCaml.
(let) est utilisé pour lier un terme à une variable et pouvoir utiliser cette
dernière dans un autre terme\footnote{On dit aussi que $x$ est est un binding
  local.}.

Enfin, (rec) est l'équivalent des enregistrements à la différence qu'il permet
de définir des termes récursifs, c'est-à-dire des termes
dont le corps peut faire référence à eux-mêmes à travers la variable $x$ qui
doit obligatoirement être accompagnée d'un type. Le
corps d'un terme récursif est une déclaration. Une déclaration est soit un
couple $(a, t)$ où $a$ est un nom de champ et $t$ un terme comme pour les
enregistrements, soit une déclaration de type représentée par un couple $(A,
T)$ où $A$ est le nom de la déclaration et $T$ le type associé. Un enregistrement peut contenir plusieurs déclarations de champ et
de type en utilisant (aggregation). Ces déclarations peuvent être mutuellement
dépendantes grâce à la variable $x$.
Le domaine d'une déclaration $d$, noté $dom(d)$, est défini comme l'ensemble des
labels.

Quant aux types, nous retrouvons $Top$ et $Bottom$ comme décrits dans le
chapitre \ref{chapter:lambda-calculus-with-records}. Le type fonction est
également présent avec une syntaxe différente et il devient dépendant,
c'est-à-dire que le type de retour peut dépendre du paramètre\footnote{La
  notation $T^{x}$ signifie que $x$ peut être libre dans $T$.}. (champ decl) est le type d'un
champ d'un enregistrement. (type decl) est le type d'une déclaration d'un type
dans un enregistrement et possède une borne inférieure $S$ et une borne
supérieure $T$. (inter) permet de typer le corps d'un enregistrement contenant
plusieurs déclarations et ce dernier est encapsulé dans un type récursif grâce à (rec). Les
types des déclarations peuvent être mutuellement dépendants grâce à la variable
$x$ du type récursif.

Enfin, les types dépendants (type proj), en parallèle de (champ proj),
permettent d'accéder à un type interne d'un enregistrement.

L'écriture de programmes DOT est lourde et fastidieuse. Des exemples \; concrets et
écrits dans un langage plus lisible et proche d'OCaml seront donnés dans le
chapitre \ref{chapter:rml}.

\section{Sémantique}

La sémantique est laissée de côté actuellement.

\section{Règles de typage}

Nous séparons les règles de typage en deux catégories : celles pour les termes
et celles pour les définitions.

Les règles de typage pour les termes sont les suivantes :

\begin{mathpar}
  \inferrule
  {\Gamma, x : T, \Gamma' \vdash x : T}
  {} \quad (\textsc{T-VAR})
  \and
  \inferrule
  {\Gamma \vdash t : S \\ \Gamma \vdash S <: T}
  {\Gamma \vdash t : T} \quad (\textsc{T-SUB})
  \and
  \inferrule
  {\Gamma, x : T \vdash t : U \\ x \notin FV(T)}
  {\Gamma \vdash \lambdaExpr{x : T}{t} : \forall(x : T) U^{x}} \quad (\textsc{ALL-I})
  \and
  \inferrule
  {\Gamma \vdash x : \forall(z : S) T^{z} \\ \Gamma \vdash y : S}
  {\Gamma \vdash x \; y : [z := y] T^{z}} \quad (\textsc{ALL-E})
  \and
  \inferrule
  {\Gamma \vdash t : T \\ \Gamma, x : T \vdash u : U \\ x \notin FV(U)}
  {\Gamma \vdash \localLetBinding{x}{t}{u} : U} \quad (\textsc{LET})
  \and
  \inferrule
  {\Gamma \vdash x : T \\ \Gamma \vdash x : U}
  {\Gamma \vdash x : T \wedge U} \quad (\textsc{AND-I})
  \and
  \\
  \inferrule
  {\Gamma, x : T \vdash d : T}
  {\Gamma \vdash \nu(x : T^{x})d : \mu(x : T^{x})} \quad (\textsc{$\left\{ \; \right\}$-I})
  \and
  \inferrule
  {\Gamma \vdash x : T^{x}}
  {\Gamma \vdash x : \mu(z : T^{z})}  \quad (\textsc{VAR-PACK})
  \and
  \inferrule
  {\Gamma \vdash x : \mu(z : T^{z})}
  {\Gamma \vdash x : T^{x}}  \quad (\textsc{VAR-UNPACK})
  \and
  \inferrule
  {\Gamma \vdash x : \left\{ a : T \right\}}
  {\Gamma \vdash x.a : T} \quad (\textsc{FLD-E})
\end{mathpar}

Les règles de typage pour les définitions sont :

\begin{mathpar}
  \inferrule
  {}
  {\Gamma \vdash \left\{ A = T \right\} : \left\{ A : T  .. T \right\}} \quad (\textsc{TYP-I})
  \and
  \inferrule
  {\Gamma \vdash d_{1} : T_{1} \\ \Gamma \vdash d_{2} : T_{2} \\ dom(d_{1})
    \cap dom(d_{2}) = \emptyset}
  {\Gamma \vdash d_{1} \wedge d_{2} : T_{1} \wedge T_{2}} \quad (\textsc{ANDDEF-I})
  \and
  \inferrule
  {\Gamma \vdash t : T}
  {\Gamma \vdash \left\{ a : t \right\} : \left\{ a : T \right\}} \quad (\textsc{FLD-I})
\end{mathpar}

Nous retrouvons les règles (T-VAR) et (T-SUB) comme dans les précédents calculs.
(ALL-I) est l'équivalent de (T-ABS) et (ALL-E) est l'équivalent de (T-APP).
Remarquons que comme $T$ peut dépendre de $z$, il est nécessaire d'effectuer une
substitution de $z$ par $y$ dans le type de $x y$ car $z$ n'existe pas en
dehors du type fonction.

(LET) nous dit que le terme $t$ et la variable $x$ à laquelle nous lions ce
terme doivent avoir le même type, le type du terme en entier est celui de $u$.
(AND-I) nous dit que si une variable a deux types $T$ et $U$, alors elle a
également le type intersection $T \wedge U$. ($\left\{ \; \right\}$-I) type les termes
récursifs et oblige la variable du terme récursif à avoir le même type que celui dans la déclaration.
(FLD-E) est l'équivalent de (T-PROJ) que nous avions défini pour les
enregistrements.
Pour finir, (VAR-PACK) nous dit que nous pouvons
contruire un type récursif à partir de n'importe quel autre type. De l'autre
coté, (VAR-UNPACK) nous dit qu'un terme ayant un type récursif possède également
le type à l'intérieur de ce dernier.

Quant aux définitions, (TYP-I) donne le même type à la borne inférieure et à la
borne supérieure d'une déclaration de type. (FLD-I) ressemble à (T-RCD)
restreint à un enregistrement avec un champ. (ANDDEF-I) type l'union de deux
déclarations dont les champs ont obligatoirement des noms différents.

\section{Règles de sous-typage}

\begin{mathpar}
  \inferrule
  {\Gamma \vdash T <: Top}
  {}
  \quad (\textsc{S-TOP})
  \and
  \inferrule
  {\Gamma \vdash Bottom <: T}
  {}
  \quad (\textsc{S-Bottom})
  \and
  \inferrule
  {\Gamma \vdash S <: T \\ \Gamma \vdash T <: U}
  {\Gamma \vdash S <: U}
  \quad (\textsc{S-TRANS})
  \and
  \inferrule
  {\Gamma \vdash T <: T}
  {}
  \quad (\textsc{S-REFL})
  \and
  \inferrule
  {\Gamma \vdash S_{2} <: S_{1} \\ \Gamma, x : S_{2} \vdash T_{1} <: T_{2}}
  {\Gamma \vdash \forall(x : S_{1}) T_{1} <: \forall(x : S_{2}) T_{2}}
  \quad (\textsc{ALL<:ALL})
  \and
  \inferrule
  {\Gamma \vdash x : \left\{ A : S .. T \right\}}
  {\Gamma \vdash S <: x.A}
  \quad (\textsc{<: SEL})
  \and
  \inferrule
  {\Gamma \vdash x : \left\{ A : S .. T \right\}}
  {\Gamma \vdash x.A <: T}
  \quad (\textsc{SEL <:})
  \and
  \inferrule
  {\Gamma \vdash T \wedge U <: T}
  {}
  \quad (\textsc{AND-1-<:})
  \and
  \inferrule
  {\Gamma \vdash T \wedge U <: U}
  {}
  \quad (\textsc{AND-2-<:})
  \and
  \inferrule
  {\Gamma \vdash S <: T \\ \Gamma \vdash S <: U}
  {\Gamma \vdash S <: T \wedge U}
  \quad (\textsc{<: AND})
  \and
  \inferrule
  {\Gamma \vdash S_{2} <: S_{1} \\ \Gamma \vdash T_{1} <: T_{2}}
  {\Gamma \vdash \left\{ A : S_{1} .. T_{1} \right\} <: \left\{ A : S_{2} ..
      T_{2} \right\}}
  \quad (\textsc{TYP<:TYP})
  \and
  \inferrule
  {\Gamma \vdash T <: U}
  {\Gamma \vdash \left\{ a : T \right\} <: \left\{ a : U \right\}}
  \quad (\textsc{FLD <: FLD})
\end{mathpar}

Comme pour le chapitre \ref{chapter:lambda-calculus-with-records}, nous
avons (S-REFL), (S-TRANS), (S-TOP), (S-BOTTOM) et l'équivalent de (S-ARROW), (ALL<:ALL).

(<: SEL) (resp. (SEL <:)) nous dit qu'une projection de type est un super-type (resp.
sous-type) de sa borne inférieure (resp. supérieure).

(FLD <: FLD) est l'équivalent de (S-RCD-DEPTH) pour un enregistrement à un
unique champ.

À travers (TYP<:TYP), les déclarations de type sont naturellement contravariantes
pour les bornes inférieures et covariantes pour les bornes supérieures.

Pour les intersections, les règles (AND-1-<:) et (AND-2-<:) nous donnent
l'équivalent de (S-RCD-WIDTH). Quant à (<:AND), cette dernière règle nous dit
que si un même type est sous-type de deux types différents, alors il est
également sous-type de leur intersection.

Bien que l'analogie avec les enregistrements et les règles de sous-typage du
chapitre \ref{chapter:lambda-calculus-with-records} permette de comprendre
l'utilité des règles, il faut remarquer que ces dernières sont plus expressives. En
effet, le type intersection n'est pas défini uniquement pour les déclarations,
mais pour n'importe quel type. Cela signifie qu'il est autorisé d'écrire $\mu(x
: T) \wedge \left\{ a : S .. T \right\}$ ou encore $x.A \wedge \left\{ a : S ..
  T \right\} \wedge y.A \wedge \mu(x : U)$.
De même, la variable $x$ dans un type dépendant n'est pas nécessairement un
terme récursif mais peut être une abstraction. Par exemple, nous pouvons écrire
$let \; y = \lambdaExpr{x : T}{t} \; in \; y.A$. Bien que ça n'ait pas de sens, la
syntaxe le permet.
La difficulté de DOT réside dans cette expressivité.

\section{Problème d'échappement}

Supposons que nous ayons un type $\real$ et les nombres et prenons l'exemple suivant :
\begin{align*}
  & let \; y = \\
  &\; \; \; \nu(x : \mu(z : \left\{ A : Bottom .. Top \right\} \wedge \left\{ a : z.A \right\})) \left\{ A = \real \right\} \wedge \left\{ a = 5 \right\} \\
  & in \; y.a
\end{align*}
Le type de l'expression est $y.A$. Cependant, la variable $y$ n'existe pas en
dehors du let car c'est une variable locale. C'est un exemple du
\textit{problème d'échappement}\footnote{\og Avoidance problem \fg \; en
anglais.}. C'est pour éviter cela que la condition $x \notin
FV(U)$ est présente dans la règle (LET).

Nous verrons dans le chapitre \ref{chapter:rml} qu'il est nécessaire d'y faire
attention lors de l'implémentation, en particulier lorsque nous faisons des
bindings locaux de variables.


\section{Problème des mauvaises bornes}

DOT permet de définir des déclarations de types avec une borne inférieure et une
borne supérieure. Cette liberté dans les types, bien qu'intéressante à première
vue, pose un problème dans le système de sous-typage.
En effet, le type $\left\{ A : Top .. Bottom \right\}$ est autorisé. Cependant,
si $x : \left\{ A : Top .. Bottom \right\}$, alors nous pouvons montrer que
$Top <: Bottom$ en utilisant (SEL <:), (<: SEL) et la transitivité. Nous avons
alors $\Gamma, x : \left\{ A : Top .. Bottom \right\} \vdash S <: T$ pour tous
$S$ et $T$.
\footnote{Divers articles sur DOT \cite{OOPSLA-DOT-2016} avancent que restreindre la
déclaration (type) $\left\{ A = T \right\}$ à une égalité permet d'éviter ce
problème lors de l'exécution.}

Nous disons qu'un type est \textit{bien formé} s'il ne possède pas de mauvaises
bornes, sinon nous disons qu'il est \textit{mal formé}. Nous disons qu'un
contexte est \textit{bien formé} s'il ne contient que des variables dont les types
sont bien formés. Sinon, le contexte est dit \textit{mal formé}.

\section{Encodage de Système $F_{<:}$}

Bien que la syntaxe soit différente de Système $F_{<:}$ et que DOT ne comporte
pas de variables de type, nous allons montrer que DOT permet d'encoder Système
$F_{<:}$. De plus, les jugements vrais pour le système de typage et de
sous-typage de Système $F_{<:}$ restent vrais pour le système de typage et de
sous-typage de DOT à travers cet encodage.

Notons $\mathcal{V}_{DOT}$ l'ensemble des variables de DOT et $\mathcal{V}_{T,
  F}$ l'ensemble des variables de type de Système $F_{<:}$. Ces ensembles 
étant infinis dénombrables, il existe une fonction injective $f$ entre
$\mathcal{V}_{T, F}$ et $\mathcal{V}_{DOT}$. Nous notons $x_{X} \in
\mathcal{V}_{DOT}$ l'image de $X \in \mathcal{V}_{T, F}$ par la fonction $f$.

Nous définissons alors la fonction ${}^{*}$ qui à chaque terme (resp. type) de
Système $F_{<:}$ associe un terme (resp. type) de DOT.

\begin{align*}
  X^{*} & = && x_{X}.A \\
  Top^{*} & = && Top \\
  (S \rightarrow T)^{*} & = && \forall(x : S^{*}) T^{*} \\
  (\forall X <: S . \, T)^{*} & = && \forall(x_{X} : \left\{ A : Bottom .. S^{*} \right\}) T^{*} \\ \\
  x^{*} & = && x \\
  (\lambdaExpr{x : T}{t})^{*} & = && \lambdaExpr{x : T^{*}}{t^{*}} \\
  (\lambdaExprType{X <: S}t)^{*} & = && \lambdaExpr{x_{X} : \left\{ A : Bottom .. S^{*} \right\}}{t^{*}} \\
  (t \; u)^{*} & = && let \; x \; = \; t^{*} \; in \\
              &  && let \; y \; = \; u^{*} \; in \\
              &  && x \; y \\
   (t[U])^{*}  & = && let \; x \; = \; t^{*} \; in \\
              &  && let \; y_{Y} \; = \; \nu(z : \left\{ A : U^{*} .. U^{*} \right\}) \left\{ A = U^{*} \right\} \; in \\
              &  && x \; y_{Y}
\end{align*}
Quant aux contextes, la fonction ${}^{*}$ est définie naturellement
inductivement par :
\begin{align*}
  (X <: Top)^{*} & = x_{X} : \left\{ A : Bottom .. Top \right\} \\
  (x : T)^{*} & = x : T^{*}
\end{align*}

Cependant, DOT est plus riche syntaxiquement que Système $F_{<:}$ car, par
exemple, il n'est pas possible en $F_{<:}$ de donner une borne inférieure à la
variable d'une abstraction alors que DOT le permet.


Notons $\Gamma \vdash_{F} t : T$ (resp. $\Gamma \vdash_{D} t : T$) un jugement
de typage pour le système de typage de Système $F_{<:}$ (resp. DOT). Nous faisons
de même pour $\Gamma \vdash_{F} S <: T$.

\begin{theorem}
  Si $\Gamma \vdash_{F} S <: T$, alors $\Gamma^{*} \vdash_{D} S^{*} <: T^{*}$.
\end{theorem}

\begin{proof}
  Par induction sur le jugement de sous-typage $\Gamma \vdash_{F} S <: T$.
  \begin{itemize}
  \item[$\bullet$] (S-TVAR). Utilisation de (<: SEL).
  \item[$\bullet$] (S-ALL). Pour le membre de gauche, nous utilisons (TYP<:TYP).
  \item[$\bullet$] Les cas (S-REFL), (S-TRANS), (S-TOP) et (S-ARROW) sont directs
    en utilisant l'encodage et l'hypothèse de récurrence.
    %Nous avons
    %\begin{mathpar}
    %  \inferrule
    %  {\Gamma \vdash T_{1} <: S_{1} \\ \Gamma, X \vdash S_{2} <: T_{2}}
    %  {\Gamma \vdash \forall X <: S_{1} . S_{2} <: \forall X <: T_{1} . T_{2}}
    %\end{mathpar}
    %et nous devons montrer que
    %\begin{mathpar}
    %  \inferrule
    %  {\Gamma^{*} \vdash_{D} \forall(x : \mu(z : \left\{ A : Bottom .. S_{1}^{*}
    %    \right\})) S_{2}^{*} <: \forall(x : \mu(z : \left\{ A : Bottom .. T_{1}^{*}\right\})) T_{2}^{*}}
    %  {}
    %\end{mathpar}
  \end{itemize}
\end{proof}

\begin{theorem}
  Si $\Gamma \vdash_{F} t : T$, alors $\Gamma^{*} \vdash_{D} t^{*} : T^{*}$.
\end{theorem}

Nous acceptons ce théorème. Une preuve peut être trouvée dans
\cite{WF-DOT-2016} et se fait par induction sur la dérivation $\Gamma \vdash_{F}
t : T$. Celle-ci repose sur deux lemmes intermédiaires pour la
substitution des types dépendants.

% TODO ajouter références d'articles qui disent que le sous-typage de DOT est indécidable.
Ces propriétés de DOT impliquent que la question du sous-typage de DOT est
indécidable\cite{WF-DOT-2016, nada-amin-thesis}.
\footnote{Plus précisément, il est nécessaire de démontrer
$\Gamma^{*} \vdash_{D} S^{*} <: T^{*} \Rightarrow \Gamma \vdash_{F} S <: T$.
Montrer ce dernier lemme n'est pas évident à cause de (S-TRANS) qui
peut faire apparaître un type qui ne provient pas obligatoirement d'un type de
Système $F_{<:}$. Néanmoins, d'après certains articles\cite{OOPSLA-DOT-2016, nada-amin-thesis}, la règle (S--TRANS) peut être
poussée plus haut dans l'arbre de dérivation si celle-ci est utilisée comme
dernière règle. Cette question n'a pas été plus étudiée mais nous
accepterons ces faits pour la suite.}

%% Historique : https://en.wikipedia.org/wiki/CLU_(programming_language)

\section{Sûreté}

Les théorèmes de préservation et de progression
restent vrais. Cependant, nous ne les démontrerons pas dans ce document car ils
nécessitent plusieurs lemmes techniques. Différentes preuves de la sûreté de DOT
existent, en utilisant des techniques et des sémantiques différentes. Nous
redirigeons le lecteur aux références données au début de ce chapitre pour plus d'informations.

\chapter{RML}
\label{chapter:rml}

Dans ce chapitre, nous proposons un algorithme de typage et de sous-typage ainsi
qu'un langage de surface pour DOT. Une implémentation en OCaml du langage et des
algorithmes sont disponibles sur Github\cite{rml-github} sous le nom de RML.

Nous parlerons également des difficultés rencontrées lors de l'implémentation
d'un calcul théorique et nous remarquerons que DOT nécessite des règles et des termes
supplémentaires pour résoudre certains problèmes d'implémentation.

%Plusieurs dépendances comme
%\verb|ocamllex| comme lexeur, \verb|menhir|\cite{menhir} pour la
%génération du parseur, \verb|AlphaLib|\cite{alphalib} pour la gestion des
%variables, \verb|PPrint|\cite{pprint} pour pouvoir afficher plus clairement les termes et les
%types et enfin \verb|ANSITerminal|\cite{ansiterminal} pour ajouter des couleurs
%lors de l'affichage.

Sur la page principale du projet se trouve, en anglais, une description complète de la syntaxe de RML,
du but du langage ainsi que des exemples.
%Cette documentation est plus orientée
%pour les développeurs ayant une base en OCaml et les personnes souhaitant utiliser le langage pour
%écrire des programmes ; les concepts théoriques sont donc omis.
La structure du projet, la méthode de compilation et l'exécution des
algorithmes sont clairement expliquées sur la page du projet. Pour ces raisons,
ces détails ne seront pas dupliqués dans ce document.
Il est fortement conseillé au lecteur de lire la page principale pour avoir une
idée générale avant de continuer.

\section{Langage de surface}

Par langage de surface, nous désignons une syntaxe plus simple et plus pratique
à utiliser pour écrire des programmes d'un calcul. Pour RML, une syntaxe proche
de celle d'OCaml est utilisée. En particulier, la syntaxe des modules OCaml est utilisée pour les
enregistrements récursifs $\nu(x : T)d$ ou encore la syntaxe des types récursifs $\mu(x :
T)$ est remplacée par la syntaxe des signatures des modules en OCaml.

Les tableaux ci-dessous regroupent l'ensemble des correspondances \\ syntaxiques entre DOT et
RML.

\begin{center}
$\arraycolsep=1.0pt\def\arraystretch{1.5}
\begin{array}{|c|c|}
  \hline
  DOT & RML \\
  \hline \hline
  \text{Bottom} & \text{Nothing} \\
  \text{Top} & \text{Any} \\
  \hline
  \left\{ \text{A : Bottom .. Top} \right\} & \text{type a} \\
  \left\{ \text{A : S .. T} \right\} & \text{type a = S  .. T} \\
  \left\{ \text{A : S .. Top} \right\} & \text{type a $:>$ S} \\
  \left\{ \text{A : Bottom .. T} \right\} & \text{type a $<:$ T} \\
  \left\{ \text{a : T} \right\} & \text{val a : T} \\
  d_{1} \wedge d_{2} & d_{1} \; d_{2} \\
  \hline
  \mu(x : T) & \text{sig(x) T end} \\
  \mu(self : T) & \text{sig T end} \\
  \hline
  \forall(x : S) T & S \rightarrow T \text{ (si $x \notin T$)}\\
  \forall(x : S) T & \text{forall(x : S) T} \\
  \hline
  T_{1} \wedge T_{2} & T_{1} \text{ \& } T_{2} \\
  T \wedge \left\{ \text{A : L .. U } \right\} & \text{T with type a = L .. U} \\
  T \wedge \left\{ \text{A : S .. T } \right\} \wedge \left\{ \text{A' : S' ..
  T' } \right\} & \text{T with type a = S .. T and a' = S' .. T'} \\
  \hline
\end{array}
$
\captionof{table}{Correspondance entre DOT et RML pour la syntaxe des types}
\end{center}

\begin{center}
$\arraycolsep=1.0pt\def\arraystretch{1.5}
\begin{array}{|c|c|}
  \hline
  DOT & RML \\
  \hline \hline
  \lambdaExpr{x : T} t & fun(x : T) \rightarrow t \\
  \hline
  \left\{ \text{A = T} \right\} & \text{type a = T} \\
  \left\{ \text{a = t} \right\} & \text{let a = t} \\
  \hline
  \nu(x : T) d & \text{sig(x) T end : struct(x) d end } \\
  \nu(self : T) d & \text{sig T end : struct d end } \\
  \hline
\end{array}
$
\captionof{table}{Correspondance entre DOT et RML pour la syntaxe des termes}
\end{center}

% TODO Ajouter un tableau

Le langage de surface est entièrement décrit sur la page du
projet\cite{rml-github}. Dans la suite, nous
utiliserons aussi bien la syntaxe du langage de surface que la syntaxe de DOT
selon le besoin et la facilité d'écriture.

\section{Implémentation des ASTs}

Les grammaires des termes, des types et des déclarations, définies dans le
fichier \verb|grammar/grammar.cppo.ml|, sont implémentées par des types sommes
appelés respectivement \verb|term|, \verb|typ| et \verb|decl|. Chaque type est
paramétré par deux types \verb|'bn| et \verb|'fn| qui représentent
respectivement le type des variables liées et le type des variables libres.

\subsection*{Gestion des variables}

La gestion des variables liées et des variables libres est l'une des tâches les plus
fastidieuses lors de l'implémentation d'un langage. En effet, il
est nécessaire de gérer l'unicité des variables, le renommage ou encore l'environnement
pour se souvenir des variables déjà utilisées, ce dernier grandissant quand nous
rentrons dans un lambda et se réduisant quand nous en sortons. De plus,
cette tâche n'est pas très intéressante au niveau algorithmique, implique très
souvent des erreurs d'inattention et il existe diverses méthodes pour
représenter et gérer les variables.

% TODO Préciser que j'ai tenté d'implémenter la gestion ?
AlphaLib\cite{alphalib} est une librairie basée sur
visitors\cite{visitors} qui fournit des macros\footnote{en utilisant
  cppo, d'où l'extension cppo.ml pour le fichier définissant la grammaire.}
générant des fonctions pour gérer les variables. Ces macros permettent par
exemple :

\begin{enumerate}
  \item de définir plusieurs représentations des variables. Dans RML, nous utilisons deux
  représentations : \textit{brute} (types \verb|raw_term|,
  \verb|raw_typ| et \verb|raw_decl|) et \textit{nominative}
  (types \verb|nominal_term|, \verb|nominal_typ| et \verb|nominal_decl|). Dans la
  représentation dite brute, les variables sont représentées par des chaînes de
  caractères tandis que dans la représentation nominative, les variables sont
  des atomes représentés par un type \verb|AlphaLib.Atom.t|, fourni par
  AlphaLib. Cette dernière attribue à chaque nouvelle variable un entier
  unique pour obtenir une représentation unique.
\item de passer d'une représentation à une autre. La représentation brute est
  utilisée par le parseur et la conversion vers la représentation nominative est
  réalisée directement après la lecture du parseur.
\item d'obtenir toutes les variables libres ou liées d'un terme.
\item dans le cas de la représentation nominative, de générer des nouveaux atomes.
\item d'utiliser des types polymorphes prédéfinis comme \verb|abs| qui représentent de manière générale
  le comportement d'une abstraction. Lorsqu'un type \verb|abs|
  est rencontré, la variable de l'abstraction est automatiquement ajoutée dans
  l'environnement. \verb|abs| est utilisé dans RML pour les types récursifs,
  les abstractions et les fonctions dépendantes.
\end{enumerate}

\begin{listing}
  \inputminted{OCaml}{codes/grammar.ml}
  \caption{Implémentation de la grammaire des termes officiels de DOT en
    utilisant AlphaLib. field\_label est un alias de type pour string.}
  \label{lst:implementation-grammar-terms-examples}
\end{listing}

La figure \ref{lst:implementation-grammar-terms-examples} montre l'implémentation
de la syntaxe abstraite des termes.

Plus d'informations et d'explications sur AlphaLib et son fonctionnement peuvent
être trouvées dans \cite{alphalib-paper}.

\section{Contexte de typage}

Un contexte est implémenté comme un dictionnaire dont les clefs sont des atomes
et les valeurs sont les types nominatifs de ces atomes. Cette implémentation se
trouve dans le fichier \verb|typing/contextType.ml| et contient également
des fonctions pour afficher un contexte donné.

%TODO Gestion d'un environnement non vide, avec une librairie standard. On regarde
%alors maintenant les termes top level qui sont composés soit d'un let top level
%(sans in), soit d'un terme. Les termes top level sont là pour étendre
%l'environnement tandis que les termes usuels non. Dans l'implémentation, cela se
%traduit par un type somme Grammar.TopLevelTerm. Lorsque nous rencontrons un
%terme top level qui est un let, nous appelons la fonction \verb|$read_top_level_term|
%qui se charge de...

%\section{Complexité des algorithmes}
%
%\begin{itemize}
%  \item Donner un détail sur la complexité des algorithmes de sous-typages et de
%  typages. %\end{itemize}

\section{Meilleures bornes d'un type dépendant}

Un problème qui se pose lors de l'écriture des algorithmes, est : étant donnés une variable
$x$ et un contexte, quelle est la borne inférieure ou la borne supérieure du
type dépendant $x.A$ ? La question se pose, par exemple, quand nous devons
comparer deux types dépendants $x.A$ et $y.A'$.

En d'autres termes, pour la borne supérieure, pour une variable $x$ de type $T$ et
un champ $A$, quel est le plus petit type $U$ tel que $T <: \left\{ A : L .. U
\right\}$ ?

Remarquons d'abord que la question n'a pas toujours de réponse. En effet, si $x$
est de type \verb|Top|\footnote{Bien que la question n'ait pas de sens, nous
  pouvons la poser car les règles ne l'empêchent pas.}, le seul super-type de
$T$ est \verb|Top|. En OCaml, nous représentons le résultat de l'algorithme par un
type \verb|option| où \verb|None| est renvoyé si la question n'a pas de réponse
et \verb|Some(T)| si la réponse est $T$. Nous omettons le \verb|Some| pour la
description ci-dessous.

Cet algorithme n'est jamais discuté ni décrit dans les différents
documents au sujet de DOT. Comprendre son utilité et son importance ainsi que
son écriture n'a pas été facile.

L'algorithme prend comme
paramètres, en plus de la variable $x$, du type $T$ et du label $A$, un contexte
$\Gamma$ et une direction. La direction est soit \verb|Lower| pour la
plus grande borne inférieure soit \verb|Upper| pour la plus petite borne
supérieure. L'algorithme travaille sur la structure de $T$ :

\begin{itemize}
\item Si $T = Bottom$, nous renvoyons $Bottom$ pour la borne supérieure et
  $None$ pour la borne inférieure.
\item Si $T = Top$, nous renvoyons $Top$ pour la borne inférieure et
  $None$ pour la borne supérieure.
\item Si $T = \forall(x : S_{1}) S_{2}$, nous renvoyons $None$ car nous
  ne pouvons pas comparer un enregistrement avec une fonction.
\item Si $T = \mu(x : S^{x})$, nous appelons récursivement l'algorithme sur
  $S^{x}$ et renvoyons ce résultat.
\item Si $T = T_{1} \wedge T_{2}$, nous
  appliquons l'algorithme récursivement sur $T_{1}$ et sur $T_{2}$. Notons
  $T'_{1}$ et $T'_{2}$ les résultats. Plusieurs cas possibles :
  \begin{itemize}
  \item Si $T'_{1} = T'_{2} = None$\footnote{Cela peut se
    passer si $T'_{1}$ et $T'_{2}$ sont des fonctions.}, nous renvoyons $None$.
  \item Si $T'_{1} = None$ et $T'_{2} \neq None$, nous renvoyons $T'_{2}$.
  \item Si $T'_{2} = None$ et $T'_{1} \neq None$, nous renvoyons $T'_{1}$.
  \item Sinon, nous renvoyons $T'_{1} \wedge T'_{2}$.
  \end{itemize}
\item Si $T = \left\{ A : L .. U \right\}$, nous renvoyons $U$ pour la borne
  supérieure et $L$ pour la borne inférieure.
\item Si $T = \left\{ a : U \right\}$, nous renvoyons $U$.
\item Si $T = y.A'$, nous appelons récursivement l'algorithme avec les paramètres
  $y$, le type de $y$, le label $A'$, le contexte $\Gamma$ et la direction et
  nous notons $S$ la réponse.
  \begin{itemize}
  \item Si $S = None$, nous renvoyons $None$. Cela signifie qu'il n'y avait déjà
  pas de réponse pour $y$ (par exemple, si $y$ est de type fonction).
  \item Sinon, nous savons que $S$ est le plus petit $U$ tel que $T_{y} <:
    \left\{ A' 
    : L .. U \right\}$ où $T_{y}$ est le type de $y$. Nous savons alors que $S$
  est la plus petite borne supérieure de $T$, c'est-à-dire le plus petit super-type de $T$. Nous
  appelons alors récursivement l'algorithme en remplaçant $T$ par $S$.
  \end{itemize}
\end{itemize}

L'algorithme actuel est satisfaisant sur différents exemples. Cependant, nous
remarquons que le cas où $T = y.A'$ n'est pas entièrement satisfaisant. Il serait nécessaire de
montrer que la réponse pour $T$ est la même que pour $S$.\footnote{Plusieurs
tentatives de démonstrations ont échoué, et la possibilité que la complétude soit
fausse a été envisagée.
Cependant, vu les bons résultats, nous acceptons l'algorithme actuel.} Nous
pouvons seulement dire que la réponse actuelle est un candidat, mais non le
meilleur. Concernant la terminaison de l'algorithme, les appels récursifs
éliminent les types dépendants, les intersections et les types récursifs,
c'est-à-dire les cas impliquant un appel récursif pour se ramener aux cas de
base. Nous pouvons donc affirmer que l'algorithme se termine toujours.

L'implémentation OCaml est la fonction \verb|best_bound_of_recursive_type|.
\verb|least_upper_bound_of_recursive_type| est l'algorithme pour la direction \\
\verb|Upper| tandis que \verb|greatest_lower_bound_of_recursive_type| est pour
la direction \verb|Lower|.

Un algorithme semblable est implémenté pour le cas où nous cherchons le plus
petit type fonction pour une variable donnée. Le retour de l'algorithme est
différent pour le cas des fonctions et des déclarations de type et de champ. Le
même argument est réalisé pour la forme $y.A'$.
L'implémentation OCaml est la fonction \verb|least_upper_bound_of_dependent_function|.

\section{Algorithme de typage}

L'algorithme de typage se trouve dans le fichier \verb|typer.ml| du dossier \verb|typing|.
L'implémentation est relativement fidèle aux règles de typage.
La fonction principale est \verb|typ_of_internal| qui prend en paramètre un
contexte et un terme. 

Cette fonction contient essentiellement un pattern matching sur la structure des
termes et applique la règle de typage appropriée selon la forme. Le problème
d'échappement est traité dans le cas de la règle (LET) à travers la
fonction \verb|check_avoidance_problem| du module \verb|CheckUtils|. Les
fonctions \verb|least_upper_bound_of_dependent_function| et \\
\verb|least_upper_bound_of_recursive_type|, expliquées précédemment, sont
utilisées respectivement dans les règles (ALL-E) et (FLD-E). La règle (SUB) est
utilisée dans (ALL-E) pour vérifier que l'argument est un sous-type que la
fonction attend.

% TODO Décrire l'algorithme comme il a été fait pour best\_bound ??

\subsection*{Typage des enregistrements récursifs}

Dans la syntaxe des termes de DOT, un enregistrement récursif est toujours accompagné de
son type. Lorsque nous écrivons des programmes, cela implique que lorsque nous
définissons des modules, nous devons donner leur signature. Du
coté développeur, cela n'est pas très pratique. Nous notons $\nu(x) d$
l'enregistrement récursif $\nu(x : T) d$ quand $T$ n'est pas mentionné.

Pour simplifier l'écriture de programmes, nous ajoutons un
algorithme pour inférer le type d'une liste de
déclarations. Lorsqu'un enregistrement récursif $\nu(x) d$ est rencontré,
nous commençons par ajouter dans le contexte $x$ avec le type \verb|Top|.
Ensuite, nous allons parcourir la déclaration $d$ et affiner le type de $x$ en
fonction de la forme de $d$.
\begin{itemize}
\item Si $d = \left\{ A = S \right\}$ et $x : T$, nous renvoyons $T \wedge
  \left\{ A : S .. S \right\}$.
\item Si $d = \left\{ a = t \right\}$ et $x : T$, nous utilisons l'algorithme de
  typage sur $t$ et le contexte $\Gamma, x : T$ et notons $S$ le résultat. Nous
  renvoyons $T \wedge \left\{ a : S \right\}$.
\item Si $d = d_{1} \wedge d_{2}$ et $x : T$, nous appelons récursivement
  l'algorithme sur $d_{1}$ avec $\Gamma, x : T$, et notons $T_{1}$ le résultat. Nous
  effectuons alors un second appel sur $d_{2}$ avec $\Gamma, x : T \wedge T_{1}$.
\end{itemize}
Notons $T$ le type renvoyé par l'algorithme, le type de l'enregistrement $\nu(x)
d$ est alors $\mu(x : T)$.

La figure \ref{code:rml-typing-recursive-records} montre un exemple
d'enregistrement récursif et le type inféré.

\begin{listing}
  \inputminted{OCaml}{codes/typing_recursive_records.rml}
  \caption{Exemple de typage d'un terme récursif dont le type n'est pas
    mentionné.}
  \label{code:rml-typing-recursive-records}
\end{listing}

L'algorithme n'est pas parfait : il échouera par exemple si des champs sont
mutuellement récursifs ou si une déclaration dépend d'une autre définie plus haut. La figure \ref{code:rml-typing-recursive-records-fail}
montre un exemple sur lequel l'algorithme échoue.
Cependant, celui-ci est satisfaisant pour une partie des exemples et permet
d'écrire de programmes DOT moins verbeux.

\begin{listing}
  \inputminted{OCaml}{codes/typing_recursive_records_fail.rml}
  \caption{Exemple de typage d'un terme récursif sur lequel l'algorithme échoue.
  L'algorithme va tenter de typer la fonction fail, mais elle a besoin du type
  de la fonction plus, qui ne sera inféré que par après.}
  \label{code:rml-typing-recursive-records-fail}
\end{listing}


\section{Algorithme de sous-typage}

L'algorithme de sous-typage se trouve dans le fichier \verb|subtype.ml| du dossier
\verb|typing| et travaille essentiellement sur la structure des deux types
donnés grâce à un pattern matching. Le but de cet algorithme est, étant donnés
deux types $S$ et $T$, de déterminer si $S$ est un sous-type de $T$.

Remarquons que pour une question posée, il est possible
d'utiliser plusieurs règles.
En effet, pour répondre à $S_{1} \wedge S_{2} <: T_{1} \wedge T_{2}$, nous
pouvons utiliser (<: AND), (AND-1-<:) ou (AND-2-<:).

De plus, l'ordre dans le pattern matching influence la réponse. Par exemple,
prenons le cas où $S_{1} = T_{1} = \left\{ A : S .. T \right\}$ et $T_{1} =
T_{2} = \left\{ a : U\right\}$. La question $S_{1} \wedge S_{2} <: T_{1} \wedge
T_{2}$ est alors vraie par réflexivité ou encore par l'arbre de dérivation
suivant :

\begin{mathpar}
  \inferrule
  {
     \inferrule
     {\Gamma \vdash
       \left\{ A : S .. T \right\} <: 
       \left\{ A : S .. T \right\}
     }
     {
       \Gamma \vdash
       \left\{ A : S .. T \right\}
       \wedge
       \left\{ a : U \right\}
       <:
       \left\{ A : S .. T \right\}
     }
     \inferrule
     {\Gamma \vdash
       \left\{ a : U \right\} <:
       \left\{ a : U \right\}
     }
     {
       \Gamma \vdash
       \left\{ A : S .. T \right\}
       \wedge
       \left\{ a : U \right\}
       <:
       \left\{ a : U \right\}
     }
  }
  {\Gamma \vdash
     \left\{ A : S .. T \right\} \wedge \left\{ a : U \right\}
     <:
     \left\{ A : S .. T \right\} \wedge \left\{ a : U \right\}
  }
\end{mathpar}

où (<:-AND) est utilisé en premier suivi de (AND-1-<:) et (AND-2-<:) sur chaque
branche. Cependant, si nous utilisions (AND-1-<:) (ou (AND-2-<:)) en premier,
nous serions incapables d'arriver au résultat voulu.

L'algorithme actuel a été implémenté de façon pragmatique, l'implémentation des
règles théoriques n'étant pas simple comme peuvent le montrer
\cite{tapl-metatheory-subtyping} et
\cite{tapl-bounded-quantification-metatheory} pour d'autres calculs.

Une difficulté supplémentaire de l'implémentation de DOT est que les règles de
typage et de sous-typage sous mutuellement dépendantes. Il est naturel et
courant d'avoir les règles de typage qui dépendent du sous-typage à travers
(T-SUB), mais non l'inverse. Dans DOT, les règles de sous-typage dépendent du
typage des variables à travers (SEL <:) ainsi que (<: SEL) et demandent
d'implémenter des algorithmes supplémentaires comme \verb|best_bound|.

%Par exemple, pour répondre à la question
%$x.A <: y.A$, nous pouvons soit utilisé (<: SEL) en premier et ensuite utilisé
%(SEL <:) ou l'inverse. De plus, il est possible qu'appliquer (<: SEL) en premier
%donne une réponse fausse alors que (SEL <:) donne une réponse vraie. TODO exemple.
%Le même problème se pose avec les règles (AND), (AND-1) et (AND-2).

Voici, dans l'ordre,
comment l'algorithme gère les différents cas en fonction de la structure de $S$
et $T$.

\subsection*{$S <: Top$ ou $Bottom <: T$}

Nous appliquons (TOP) ou (BOTTOM) en fonction du cas. Ces règles peuvent
s'appliquer directement dans l'arbre de dérivation donc nous pouvons les placer
en premiers dans le pattern matching.

\subsection*{$\left\{ A : L ... U \right\} <: \left\{ A : L' ... U' \right\}$}

Même cas que précédemment. Il suffit d'appeler récursivement l'algorithme avec L
et L' ainsi que U et U'. Si les deux appels renvoient vrais, nous renvoyons
vrai. Nous utilisons la règle (TYP <: TYP).

\subsection*{$\left\{ a : S' \right\} <: \left\{ a : T' \right\}$}

Même cas que précédemment en utilisant la règle (FLD <: FLD). Il suffit
d'appeler l'algorithme avec la question $S' :< T'$.

\subsection*{$\forall(x : S_{1}) T_{1} <: \forall(x : S_{2}) T_{2}$}

Même cas que précédemment en utilisant la règle (ALL <: ALL). Lors de l'appel
récursif sur la question $T_{1} <: T_{2}$, nous devons ajouter la variable $x$
avec le type $S_{2}$ dans le contexte comme la règle (ALL <: ALL) le mentionne.

\subsection*{$x.A <: y.A'$}

Les premiers cas posant des difficultés sont ceux des types dépendants, par exemple
$x.A$ et $y.A'$\footnote{Les cas $x.A <: U$ et $U <: x.A$ sont traités ci-dessous.}. En effet, les règles (REFL), (SEL <:) et (<: SEL) peuvent être
employées.

L'algorithme procède par cas :
\begin{itemize}
\item Dans le cas de la réflexivité, cela signifie que $x$ et $y$ sont les mêmes
  atomes. Nous utilisons donc les fonctions fournies par \verb|AlphaLib| pour le
  vérifier. Nous appelons cette règle (UN-REFL-TYP).
\item Sinon, nous testons en premier (SEL <:). Nous ajoutons en plus (T-SUB) et
  nous utilisons \verb|best_bound| pour trouver le type de $x$ et de $y$. La règle
  \begin{mathpar}
    \inferrule
    {\Gamma \vdash x : \left\{ A : L .. U\right\}}
    {\Gamma \vdash x.A <: U}
  \end{mathpar}
  devient alors
  \begin{mathpar}
    \inferrule
    {\Gamma \vdash x : T \\ \Gamma \vdash T <: \left\{ A : L .. U\right\} \\ \Gamma \vdash U
      <: U'}
    {\Gamma \vdash x.A <: U'}
  \end{mathpar}
  Si nous avons réussi à montrer en utilisant (SEL <:), nous renvoyons cette solution. Sinon, nous
  essayons (<: SEL). Si cette dernière échoue, cela signifie que $x.A <: y.A'$
  est faux : nous renvoyons
  donc non. L'utilisation de \verb|best_bound| et (T-SUB) pour (SEL<:) nous donne
  \begin{mathpar}
    \inferrule
    {\Gamma \vdash x : T \\ \Gamma \vdash \left\{ A : L' .. U\right\} <: T \\ \Gamma \vdash L
      <: L'}
    {\Gamma \vdash L <: x.A}
  \end{mathpar}

\end{itemize}

\subsection*{Types récursifs}

Il est important de remarquer qu'il n'y a pas de règle de sous-typage pour les
types récursifs. En effet, pour comparer deux types récursifs, ou au moins un
type récursif, il est nécessaire d'utiliser (VAR-UNPACK) ou (VAR-PACK).

Du coté de l'implémentation, nous ajoutons, dans l'ordre, les deux cas
particuliers suivants.

\begin{itemize}
\item Si nous avons deux types récursifs $\nu(x_{1} : S')$ et $\nu(x_{2} : T')$,
nous créons un nouvel atome $x$ et renommons les variables internes $x_{1}$ et
$x_{2}$ par celui-ci dans $S'$ et $T'$. Un appel récursif est alors effectué
avec $S'$ et $T'$ après avoir étendu le contexte avec $x : S'$. Nous appelons
cette règle (UN-REC).
\item Si $S = \nu(x : S')$ et $T$ quelconque (resp. $T = \nu(x : T')$ et $S$
  quelconque), nous ajoutons $x$ dans le contexte avec le type $S'$ (resp. $T'$) et
  nous effectuons un appel récursif avec $S'$ et $T$ (resp. $S$ et $T'$). Étendre
  le contexte est nécessaire si $S'$ contient des champs mutuellement dépendants.
  Nous appelons ces règles respectivement (UN-<: REC) et (UN-REC <:).
\end{itemize}

\begin{listing}
  \inputminted{OCaml}{codes/rml_implementation_recursive_type.rml}
  \caption{Ces deux signatures sont identiques à l'exception de la variable
    interne. Si nous ne donnons pas le même nom à la variable interne, la
    question $self.t <: self'.t$ va être posée. Comme ce ne sont pas les mêmes
    atomes, la question $Top <: Bottom$ sera posée que nous utilisions (SEL <:) ou
    (<: SEL).}
  \label{example:rml-implementation-recursive-type-variable}
\end{listing}

Le premier cas est nécessaire pour pouvoir comparer des types récursifs qui ne
se différencient que par leur variable interne, voir la figure
\ref{example:rml-implementation-recursive-type-variable}.

\subsection*{$x.A <: T$ ou $T <: x.A$}

Comme pour le cas $x.A <: y.A'$, nous modifions la règle de sous-typage en
utilisant (T-SUB) et \verb|best_bound|.

\subsection*{Intersections}

L'ordre est également important pour les intersections et surtout en présence de
types récursifs dans l'un des membres.

Premièrement, il est nécessaire de placer (<:AND) avant (AND-1-<:) et
(AND-2-<:) pour pouvoir gérer le cas de la réflexivité comme nous l'avons vu ci-dessus.

Ensuite, comme pour (SEL <:) et (<: SEL), nous ne pouvons tester séparément
(AND-1-<:) ou (AND-2-<:) car les règles peuvent être utilisées en même temps.
Nous procédons donc de la même manière que pour (SEL <:) et (<: SEL) en testant
successivement (AND-1-<:) et (AND-2-<:)

De plus, dans le cas (<: AND), nous devons gérer les types récursifs à cause de
la règle (VAR-PACK). En effet, si nous avons la question $\mu(x : S_{1}) \wedge
\mu(x : S_{2}) <: \mu(x : S_{1} \wedge S_{2})$, nous pouvons revenir à la
question $\mu(x : S_{1} \wedge S_{2}) <: \mu(x : S_{1} \wedge S_{2})$. Si nous
utilisions directement (<: AND), nous n'arriverions pas à montrer que c'est vrai.

De manière générale, si nous avons $\mu(x : S_{1}) \wedge \mu(x : S_{2}) <: T$,
une solution est de montrer $\mu(x : S_{1} \wedge S_{2}) <: T$. En effet,
en utilisant successivement (<: AND), (VAR-UNPACK), (AND-1-<:), (AND-2-<:) et
(VAR-PACK), nous montrons que $\mu(x : S_{1} \wedge S_{2}) <: T$ implique $\mu(x
: S_{1}) \wedge \mu(x : S_{2}) <: T$.

\begin{mathpar}
  \inferrule*[Left=(<: AND)]
  {
    \inferrule*[Left=(VAR-PACK)]
    {
      \Gamma \vdash \mu(x : S_{1} \wedge S_{2}) <: T
    }
    {
      \inferrule*[Left=(AND-1-<: et AND-2-<:)]
      {\Gamma \vdash S_{1} \wedge S_{2} <: T}
      {
        \inferrule*[Left=(VAR-UNPACK)]
        {\Gamma \vdash S_{1} <: T}
        {\Gamma \vdash \mu(x : S_{1}) <: T}
        \\
        \inferrule*[Right=(VAR-UNPACK)]
        {\Gamma \vdash S_{2} <: T}
        {\Gamma \vdash \mu(x : S_{2}) <: T}
      }
    }
  }
  {\Gamma \vdash \mu(x : S_{1}) \wedge \mu(x : S_{2}) <: T}
\end{mathpar}

La méthode utilisée dans l'algorithme consiste à renommer la variable interne de
$S_{1}$ et $S_{2}$ en utilisant un même atome unique et d'appeler récursivement l'algorithme.

Une méthode similaire est utilisée pour les cas $\mu(x : S_{1}) \wedge S_{2} <:
T$ et $S_{1} \wedge \mu(x : S_{2}) <: T$.

\subsection*{Réflexivité}

La régle de réflexivité (REFL) n'est pas implémentée directement. En effet,
cette règle peut être dérivée d'autres règles.

\begin{itemize}
  \item Si $S$ est de la forme $x.A$, (UN-REFL-TYP) est utilisée.
  \item Si $S$ est de la forme $\forall(x : S') T'$, nous pouvons utiliser (ALL
    <: ALL).
  \item Si $S$ est de la forme $\nu(x : S')$, (UN-REC) est utilisée.
  \item Si $S$ est une intersection, nous pouvons utiliser (<: AND), puis
    (AND-1-<:) sur le membre de gauche et (AND-2-<:) sur le membre de droite
    comme nous l'avons montré plus haut.
  \item Si $S$ est de la forme $\left\{ A : L .. U \right\}$, (TYP <: TYP) est utilisée.
  \item Si $S$ est de la forme $\left\{ a : T \right\}$, (FLD <: FLD) est utilisée.
\end{itemize}

\subsection*{Transitivité}

Actuellement, la transitivité n'est pas gérée. En effet, pour la gérer, il
faudrait trouver un type $U$ tel que $S <: U$ et $T <: U$, ce qui n'est pas
possible, ou du moins compliqué. Pour contourner ce problème, il est
nécessaire de reformuler les règles de sous-typage et d'y introduire
explicitement (S-TRANS).

\section{Arbre de dérivation}

Les algorithmes de typage et de sous-typage nécessitent un paramètre
supplémentaire \verb|history| qui permet de se souvenir des règles
utilisées et de pouvoir ainsi reconstruire l'arbre de dérivation. Cet arbre de
dérivation peut être affiché pour une expression en utilisant l'annotation
\verb|[@show_derivation_tree]| à la fin des expressions. Par défaut, l'arbre
affiche également le contexte. Comme celui-ci peut être grand pour les
longs programmes, l'annotation \\ \verb|[@show_derivation_tree, no_context]|
affiche l'arbre de dérivation sans le contexte.

\section{Sucres syntaxiques}

La syntaxe de base de DOT n'est pas très élégante ni très pratique à utiliser.
En effet, il n'est par exemple pas possible de définir des fonctions à plusieurs
variables, de passer des termes en paramètre d'une fonction ou encore d'utiliser
un terme dans une sélection.

Pour ces raisons, des sucres syntaxiques au niveau des parseurs sont implémentés
dans le langage de surface de RML.

\subsection*{Curryfication des fonctions}

Dans RML, il est possible de définir des fonctions à plusieurs variables en les
séparant par une virgule.
Par exemple, \verb|fun(x : Int.t, y : Int.t) -> t| est équivalent à
\verb|fun(x : Int.t) -> fun(y : Int.t)|. Le parseur se charge de créer l'arbre de
syntaxe correspondant.

\subsection*{Variable interne par défaut}

DOT nécessite une variable interne lors de la définition d'un module afin de
pouvoir faire référence aux champs et types. Une variable par défaut,
\verb|self|, est utilisée dans le parseur afin d'alléger l'écriture de programme
si le nom de la variable interne n'est pas importante.

\subsection*{Enregistrement}

Les enregistrements ont la même représentation interne que les modules,
\verb|TermRecursiveRecord|. Dans le langage de surface, les enregistrements
ainsi que le type enregistrement sont définis de la même manière qu'en OCaml.
Afin d'éviter des références internes entre champs, la variable interne utilisée
est \verb|'self|. Les noms de variables commençant par des simples guillemets
n'étant pas acceptés dans le lexeur, cela implique qu'il n'est pas possible
d'avoir des champs mutuellement dépendants.

\subsection*{Termes comme paramètres et fonctions}

Une fonctionnalité intéressante et pratique est l'utilisation de
termes pour les paramètres ainsi que pour les fonctions. Ceci
est géré dans le parseur et ce dernier génère des bindings locaux. Cela se
résume par la règle suivante :

\begin{center}
  x t $\rightarrow$ let y = t in x y
\end{center}

Un point important\footnote{Ce n'est pas mentionné dans les documents sur
DOT.} est qu'il faut éviter de réaliser un binding local d'une variable car
cela pourrait provoquer des problèmes d'échappement. Cela signifie que DOT n'est
pas stable par insertion de let variable-variable. Un exemple concret est donné
par la figure \ref{code:implementation-grammar-terms-examples}.

Pour résoudre ce problème, nous pourrions, dans le cas d'un binding local
\verb|let x = y in u|, donner le type (= y) à la variable $x$. Lorsque nous
rencontrons alors la variable $x$, nous utilisons le type de $y$. Cette méthode
est utilisée dans le langage Mezzo\cite{} avec le type singleton.

\begin{listing}
  \label{code:implementation-grammar-terms-examples}
  \inputminted{OCaml}{codes/terms_binding_variable.rml}
  \caption{Exemple où un binding local d'une variable ne doit pas être généré
    afin de ne pas provoquer un problème d'échappement. Si des bindings locaux
    sont réalisés pour chaque terme, une liaison locale du module $M$ est
    créée avec la variable $n$ par exemple et le type de l'expression est $n.t$.}
\end{listing}

\subsection*{Applications de fonctions à plusieurs paramètres}

Une autre fonctionnalité importante est la possibilité d'appliquer une fonction plusieurs
paramètres. C'est aussi le parseur qui s'en occupe en générant
des bindings locaux. Voici quelques exemples :

\begin{itemize}
\item $f \; x \; y$ est interprété comme $let \; f_{x} \; = \; (f \; x) \; in \;
  (f_{x} \; y)$.
\item $f \,  x \, y \, z$ est interprété comme $let \; f_{x} \; = \; (f \; x) \;
  in \; let \; f_{y} \; = \; (f_{x} \; y) \; in \; (f_{y} \; z)$.
\end{itemize}

\section{Termes ajoutés}

\subsection*{Termes unit et entiers}

Des types basiques comme \verb|Int.t| et \verb|Unit.t| pour les entiers et le terme
\verb|unit| sont implémentés. Il est possible d'utiliser des entiers comme en
OCaml ou le terme \verb|()| pour le terme \verb|unit|.

\subsection*{Unimplemented}

L'implémentation ne se focalisant pas sur l'évaluation, la sémantique des termes
peut être laissée de coté. Pour cela, un terme \verb|Unimplemented| est présent
et de type \verb|Bottom|.

\subsection*{TermAscription}

Comme dans la plupart des langages, RML autorise l'ascription de termes,
c'est-à-dire forcer le type d'un terme. En particulier, cela permet de donner le
type voulu au terme \verb|Unimplemented|. La syntaxe est $t : T$. Dans
l'implémentation, nous vérifions que le type de $t$ est sous-type de $T$.

\subsection*{TermRecursiveRecordUntyped}

Un terme est ajouté dans la grammaire pour les modules définis sans type,
l'algorithme de typage sur les modules décrit précédemment étant utilisé pour
typer le terme.

%\section{Types de bases}

%\section{Principal}
%
%Le fichier principal est \verb|main.ml|. Celui-ci ...
%
%\begin{itemize}
%  \item Donner les mêmes exemples qu'au début, mais cette fois-ci avec RML.
%  \item Montrer ce qui est possible en RML et ce qui ne l'est pas en OCaml.
%\end{itemize}
%
%\section{Exécution des algorithmes}

\section{Exemples}

Reprenons l'exemple donné en introduction. En RML, le module \verb|Point2D| avec
des entiers peut être défini de la manière suivante :

\begin{listing}
  \inputminted{OCaml}{codes/point2d.rml}
  \caption{Point2D en RML.}
\end{listing}

Nous pouvons définir le foncteur \verb|MakePoint2D| avec le terme \verb|fun|
comme une autre fonction :

\begin{listing}
  \inputminted{OCaml}{codes/makepoint2d.rml}
  \caption{MakePoint2D en RML. Le paramètre de la fonction est un module qui
    contient au moins un champ t et une fonction add.}
  \label{code:rml-point2d}
\end{listing}

Le type du foncteur est le même que celui d'une fonction qui prend un module
en paramètre et retourne un module :

\begin{listing}
  \inputminted{OCaml}{codes/makepoint2d_sig.rml}
  \caption{Signature de MakePoint2D en RML.}
  \label{code:rml-makepoint2d}
\end{listing}

Nous pouvons également définir des listes paramétrées par un type en utilisant un foncteur
comme le montre la figure \ref{code:rml-list-functor} et créer une liste d'entier en
utilisant \verb|List Int| :

\begin{listing}
  \inputminted{OCaml}{codes/list_functor.rml}
  \caption{Une implémentation de listes polymorphes en RML en utilisant un
    foncteur. Le type t représente le type liste. Le module elem
    est le type des éléments de la liste. Remarquons que dans cette
    implémentation, une liste ne peut contenir que des éléments du même type, ce
    dernier étant fixé par le paramètre du foncteur.}
  \label{code:rml-list-functor}
\end{listing}

Cependant, cette implémentation de liste n'est pas très pratique car si nous
voulons une liste d'entiers, nous devons créer un module intermédiaire. La figure
\ref{code:rml-list-with} propose une autre implémentation de listes polymorphes
qui de plus sont covariantes (grâce à \verb|type t <: self.t|). Le mot clef
\verb|with| dénote une intersection\footnote{Comme en OCaml dans
  les foncteurs.}. Nous pouvons alors utiliser \verb|List.list with type t = Int|
pour obtenir une liste d'entiers. Malheureusement, l'algorithme actuel ne
supporte pas cette implémentation et provoque un stack overflow car le champ
\verb|tail| repose la question du sous-typage de la liste en
boucle.\footnote{Une solution est en cours de développement pour supporter les
  questions qui ont déjà été posées.}

\begin{listing}
  \inputminted{OCaml}{codes/list.rml}
  \caption{Une implémentation de listes polymorphes en RML en utilisant le mot
    clef with. Contrairement à l'implémentation de la figure
    \ref{code:rml-list-functor}, les éléments de la liste peuvent avoir un type
  différent.}
  \label{code:rml-list-with}
\end{listing}

Plus d'exemples peuvent être trouvés sur la page du projet.

\section{Travail futur}

Bien que l'implémentation actuelle donne des résultats satisfaisants sur
différents cas, diverses améliorations peuvent être effectuées. Une liste \\
non exhaustive peut être trouvée, en anglais, sur la page du
projet\cite{rml-github-issues}. En voici quelques exemples :

\begin{itemize}
  \item L'algorithme de sous-typage provoque sur certains cas des stack
    overflow. Ceci n'est pas très surprenant. En effet, à cause
    du type récursif, un arbre de dérivation n'est pas nécessairement de taille finie car
    il est possible que l'algorithme doive répondre à la même question dans un
    sous-arbre. De plus, comme le problème du sous-typage est indécidable, il
    existe des questions qui ne disposent pas de réponse.
  \item Nous nous sommes focalisés essentiellement sur l'implémentation des
    algorithmes de typage et de sous-typage, et non sur l'évaluation des termes.
    Un évaluateur pourrait être implémenté en se basant sur \cite{WF-DOT-2016}.
  \item Un interpréteur interactif.
  \item Améliorer l'algorithme d'inférence de type pour les modules. En effet,
      comme nous l'avons vu, celui-ci est relativement naïf et ne permet pas par
      exemple de gérer un module contenant des champs mutuellement dépendants.
  \item Améliorer et prouver ensuite que les algorithmes de sous-typage et de
      typage définissent bien les relations de typage et de sous-typage.
  \item Pour l'instant, il est nécessaire de donner un type lorsque nous
    utilisons un match sur une option (voir fichier \verb|stdlib/option_church.rml|), ce qui
    n'est pas courant en OCaml car le type est inféré. Cette inférence de type
    passe par la résolution d'équations et nécessite de travailler avec deux
    arbres différents.
  \item Un enregistrement récursif peut posséder plusieurs déclarations grâce à
    l'intersection. Il serait plus efficace d'implémenter cela grâce à une map
    qui associe la déclaration à un label. Cette solution n'a pas été
    implémentée directement pour rester fidèle à la grammaire de DOT.
\end{itemize}

\chapter*{Conclusion}
\addcontentsline{toc}{chapter}{Conclusion}

Dans les premiers chapitres, nous avons défini les bases théoriques de la
programmation fonctionnelle et des langages typés. Nous sommes partis de
différents calculs relativement simples comme le $\lambda$-calcul non typé et le
$\lambda$-calcul simplement typé pour arriver à un calcul plus compliqué et plus
récent (2016) appelé DOT qui unifie
le comportement des enregistrements et des modules et permet en même temps
de considérer les modules comme des citoyens de première classe. Nous
avons également montré comment DOT pouvait être interprété comme une extension
de Système $F_{<:}$.

Dans le dernier chapitre, nous avons présenté une implémentation en OCaml basée
sur DOT. Nous avons pu remarquer qu'implémenter un langage à partir de règles
théoriques n'est pas évident et cela à cause des différents arbres de
dérivations possibles pour une même question ou encore en raison des arbres de
tailles infinies.
Nous avons également remarqué qu'il était nécessaire de changer certaines règles
d'inférence pour écrire un algorithme, comme pour le cas de la réflexivité,
l'inclusion de (T-SUB) dans les règles d'inférence ou encore l'introduction de
(UN-REC) pour pouvoir comparer des types récursifs.
De plus, nous avons remarqué qu'introduire des types chemins dépendants dans le
langage ne facilite pas l'implémentation. Pour finir, il nous a été nécessaire
d'écrire des algorithmes secondaires comme \verb|best_bound|, non décrits dans
les différents articles discutant de DOT.

Nous avons également vu qu'écrire des programmes dans un calcul théorique comme
DOT n'est pas très pratique et implique de développer un langage de surface.
Dans ce langage de surface, des sucres syntaxiques sont implémentés afin de
pouvoir écrire des termes interdits dans DOT, comme l'application de termes à
une fonction. Cependant, à travers l'implémentation de ces sucres, nous avons
remarqué qu'il manquait certaines règles pour pouvoir écrire certains programmes
usuels comme le binding local d'une variable.

DOT n'est pas le seul calcul dans lequel les modules peuvent être considérés
comme citoyens de première classe. D'autres calculs ont été explorés comme
1ML\cite{1ml-paper}. Ce dernier catégorise les types en
genres\cite{tapl-higher-order-systems} afin d'affiner les règles de typage et
de sous-typage sur les types. DOT a été choisi à la place de 1ML car ce dernier possède
déjà une implémentation et les règles de typage et de sous-typage sont plus
élaborées que DOT.

% Discuter qu'il y a d'autres sémantiques intéressantes comme la sémantique
% catégorique, sujet de recherche assez récent et inexploré actuellement dans le
% cas de DOT ?

\appendix
\chapter{Preuve par récurrence sur les termes et les types}

Les termes ainsi que les types d'un langage sont définis de manière récursive.
Par exemple, pour le $\lambda$-calcul simplement typé, la grammaire des termes
est définie par
\begin{align*}
  t ::= & \, & \text{terme} \\
        & \; x & \text{var} \\
        & \; t \, t & \text{app} \\
        & \; \lambdaExpr{x : T}{t} & \text{abs}
\end{align*}

et la grammaire des types, en supposant que nous avons uniquement \verb|Bool|
(pour les booléens) comme type de base, est définie par
\begin{align*}
  T ::= & \, & \text{types} \\
        & \; Bool & \text{type des booléens} \\
        & \; T \rightarrow T & \text{type des fonctions}
\end{align*}

Ces définitions récursives sur les termes et les types nous permettent de
définir récursivement des fonctions agissant sur les termes et les types. Par
exemple, nous pouvons définir de manière inductive une fonction
\verb|size| sur les termes et les types de la manière suivante.

\begin{align*}
  size(x) & = 1 \\
  size(t_{1} t_{2}) & = size(t_{1}) + size(t_{2}) \\
  size(\lambdaExpr{x : T}{t}) & = size(t) + 1
\end{align*}

\begin{align*}
  size(Bool) & = 1 \\
  size(T_{1} \rightarrow T_{2}) & = size(T_{1}) + size(T_{2}) \\
\end{align*}

Si nous nous représentons les termes et les types en forme d'arbre, la
définition se résume à la définition du nombre de noeuds de l'arbre. Cette
représentation et la définition de fonctions comme \verb|size| nous permettent de
raisonner par induction sur le nombre de noeuds de l'arbre en utilisant l'induction sur
les naturels, comme le montre la preuve d'unicité de type pour le
$\lambda$-calcul simplement typé. Une telle induction est appelée
\textit{induction structurelle}.

Nous supposons pour la plupart des grammaires que nous disposons d'une telle
fonction qui permette de raisonner inductivement sur la structure des
programmes ou des types, la fonction \verb|size| étant souvent facile à définir.

Certaines preuves nécessitent une induction sur deux paramètres naturels comme celles
du lemme d'affaiblissement et du lemme de permutation pour le $\lambda$-calcul
simplement typé.

En effet, pour le lemme de permutation, pour le cas des abstractions, l'argument
complet est:

\og Par hypothèse et le lemme d'inversion, $\Gamma \vdash \lambdaExpr{x : T_{1}}
t' : T_{1} \rightarrow T_{2}$ et $\Gamma, x : T_{1} \vdash t' : T_{2}$. Par
hypothèse de récurrence, $\Delta, x : T_{1} \vdash t' : T_{2}$. Par $(T-ABS)$,
$\Delta \vdash \lambdaExpr{x : T_{1}}{t'} : T_{1} \rightarrow T_{2}$. \fg

Cependant, l'hypothèse de récurrence est \og $\Gamma \vdash t : T$ implique $\Delta
\vdash t : T$ \fg, et non \og $\Gamma, x : T_{1} \vdash t : T$ implique $\Delta, x :
T_{1} \vdash t : T$ \fg : le contexte n'est pas le même.

L'argument reste pourtant vrai : le principe de récurrence utilisé est celui sur
$\naturel^{2}$ muni de l'ordre lexicographique en utilisant comme premier
paramètre la taille du terme (qui diminue strictement dans le cas donné) et
comme second la taille du contexte (qui augmente).

Pour rappel, le principe de récurrence sur $\naturel^{2}$ est le suivant :
\begin{proposition}
  Soit $P$ une proposition sur $\naturel^{2}$.

  Si, pour tout $(m, n) \in \naturel^{2}$, $P(m', n')$ est vrai pour tout $(m',
  n') \leq (m, n)$, alors, $P(m, n)$ est vrai pour tout $(m, n) \in \naturel^{2}$.
\end{proposition}

et se démontre en plusieurs lemmes :

\begin{lemma} [Principe d'induction sur un ensemble bien ordonné]
  Soit $(X, \leq)$ un ensemble bien ordonné, alors le principe d'induction est
  vrai sur $X$, c'est-à-dire:

  Soit $P$ une proposition sur $X$.
  Si pour tout $x \in X$, quelque soit $y \in X$ tel que $y \leq x$, $P(y)$ est
  vrai, alors $P(x)$ est vrai pour tout $x \in X$.
\end{lemma}

\begin{proof}
  Même principe que la preuve sur $\naturel$.
\end{proof}

\begin{lemma}
  Soit $(X, \leq)$ un ensemble bien ordonné. Alors $(X^{2}, \leq_{l})$ où
  $\leq_{l}$ est l'ordre lexicographique est bien ordonné.
\end{lemma}

\begin{proof}
  Soit $S \subseteq X^{2}$.

  Posons $S_{1} = \GSsetDef{x \in X}{\exists y \in X \text{ tel que } (x, y) \in S}$.
  Comme $S_{1} \subseteq X$ et $X$ est bien ordonné, $S_{1}$ possède un minimum.
  Notons le $\min{S_{1}}$.

  Posons $T = \GSsetDef{y \in X}{(\min{S_{1}}, y) \in S}$. Comme $T \subseteq
  X$, $T$ possède un minimum. Notons le $\min{T}$.

  Alors, $s = (\min{S_{1}}, \min{T})$ est le minimum de $S$. En effet, si $(x, y)
  \in S$, on a $x \in S_{1}$ car $(x, y) \in S$, donc $\min{S_{1}} \leq x$ et si
  $\min{S_{1}} = x$, alors, comme $y \in T$, $\min{T} \leq y$. Par construction,
  $s \in S$.
\end{proof}

De manière générale, le dernier lemme peut se démontrer, en utilisant les mêmes
arguments, pour $X^{n}$ où $n$ est un naturel quelconque.

Nous gardons le terme \textit{induction structurelle} qu'importe l'ensemble sur
lequel nous utilisons le principe d'induction.

\bibliographystyle{acm}
\bibliography{memoire}% si le fichier BibTeX est memoire.bib

\end{document}
%%% Local Variables: 
%%% mode: latex
%%% TeX-master: t
%%% TeX-PDF-mode: t
%%% End: 
