\chapter{Système $F_{<:}$}
\label{chapter:system-f-sub}

Dans les chapitres \ref{chapter:lambda-calculus-with-records} et
\ref{chapter:system-f}, nous avons défini deux notions de polymorphisme: le
polymorphisme par sous-typage à travers les enregistrements et le polymorphisme
paramétré.

Les deux calculs permettent de résoudre deux problèmes différents :
\begin{enumerate}
  \item le sous-typage permet d'affiner notre relation de typage à travers la
    relation $<:$. Par exemple, il nous est permis de définir la fonction
    identité sur les réels ($\lambdaExpr{x : \real} x$) et de l'appliquer à un
    entier car $\naturel$ est un sous-type de $\real$.
  \item le polymorphisme paramétré permet de quantifier sur les types et de
    créer des fonctions indépendamment du type grâce aux abstractions de type
    pour ensuite les appliquer aux types souhaités. Par exemple, nous pouvons
    définir la fonction identité polymorphe $\lambdaExprType{X}{\lambdaExpr{x :
    X}{x}}$ et les appliquer aux types $\real$ et $\naturel$, mais également à
    n'importe quel autre type comme $\naturel \rightarrow \real$ ou encore $\real
    \rightarrow (\naturel \rightarrow \real)$.
\end{enumerate}

Dans ce chapitre, nous allons unifier ces deux notions en un langage appelé
Système $F_{<:}$\footnote{prononcé \og system F sub \fg}. L'idée principale de
Système $F_{<:}$ est d'élargir notre relation de sous-typage sur les variables de
type pour les borner et ainsi restreindre les types qui peuvent être appliqués
aux abstractions de type. De cette façon, une même abstraction de type dépendante d'une
variable $X$, bornée supérieurement par $\real$, pourra accepter les types $\naturel$ et
$\real$, mais pas $\naturel \rightarrow \real$ ou encore $\real \rightarrow
\real$. Nous ne considérons donc plus seules les variables de type, mais liées
à une borne supérieure. Nous notons $X <: T$ pour dire que la variable de type
$X$ est borné supérieurement par $T$.  

Ajouter une borne supérieure permet d'affiner le type de la variable $X$ en
obligeant le type appliqué à avoir certaines caractéristiques. Par exemple, la
fonction $\lambdaExprType{X <: \left\{ z : \real \right\}}{\lambdaExpr{x :
    X}{\, x.z}}$ oblige, lors d'une application de type, de donner un type
enregistrement avec au moins le champ $z$ qui est de type réel.

Dans ce chapitre, nous considérons également le type \verb|Top|.

\section{Syntaxe}

La syntaxe de Système $F_{<:}$ est celle de Système $F$ à laquelle nous ajoutons
une borne supérieure aux variables des abstractions de type.

\begin{minipage}{0.45\textwidth}
  \begin{align*}
    t ::= & \, & \text{terme} \\
          & \; x & \text{var} \\
          & \; t \, t & \text{app} \\
          & \; \lambdaExpr{x : T}{t} & \text{abs} \\
          & \; \lambdaExprType{X <: T}{t} & \text{type abs} \\
          & \; t[T] & \text{type app}
  \end{align*}
\end{minipage}
\begin{minipage}{0.45\textwidth}
  \begin{align*}
    T ::= & \, & \text{type} \\
          & \; X & \text{type var} \\
          & \; T \rightarrow T & \text{fonction} \\
          & \; \forall X <: T. \, T & \text{universel} \\
          & \; Top & \text{Top}
  \end{align*}
\end{minipage}
\\
\\
\\
Les valeurs restent les mêmes :
\begin{align*}
  v ::= & \, & \text{valeur} \\
        & \; \lambdaExpr{x : T}{t} & \text{abs} \\
        & \; \lambdaExprType{X <: T}{t} & \text{type abs}
\end{align*}

\section{Sémantique}

Au niveau de la sémantique, les règles restent les mêmes que Système F.

\begin{mathpar}
  \inferrule
    {t_{1} \rightarrow t_{1}'}
    {t_{1} \, t \rightarrow t_{1}' \, t} \quad (\textsc{E-APP1})
  \and
  \inferrule
    {t \rightarrow t'}
    {v \, t \rightarrow v \, t'} \quad (\textsc{E-APP2})
  \and
  \inferrule
    {(\lambdaExpr{x : T}{t}) v \rightarrow [x := v]t}
    {} \quad (\textsc{E-APPABS})
  \and
  \inferrule
  {t \rightarrow t'}
  {t[T] \rightarrow t'[T]} \quad (\textsc{E-T-APP})
  \and
  \inferrule
  {(\lambdaExprType{X <: T_{1}}{t})[T] \rightarrow [X \rightarrow T]t}
  {} \quad (\textsc{E-T-ABS})
\end{mathpar}
\label{eval:system-f-sub}

\section{Contexte de typage}

Au niveau du contexte de typage, à la place d'avoir uniquement la variable de
type, nous allons également ajouter la borne supérieure.

Il est nécessaire de faire attention à l'ordre d'apparition des variables. Par
exemple, nous dirons que $X <: Y, Y <: Top$ est mal formé car $Y$
apparaît comme borne avant sa définition ($Y <: Top$). Bien que ça soit une
définition informelle, nous considérerons uniquement les contextes de typage
tel qu'une variable $Y$ n'apparait jamais comme borne supérieure à gauche de sa
définition.

La syntaxe du contexte de typage est donc :
\begin{align*}
  \Gamma ::= & \, & \text{contexte} \\
        & \; \emptyset & \\
        & \; \Gamma, x : T & \\
        & \; \Gamma, X <: T & \\
\end{align*}

\section{Règles de typage et de sous-typage}

Comme le contexte comporte des relations de sous-typage, nous ajoutons un
contexte $\Gamma$ dans les règles de sous-typage. Une affirmation $S <: T$
devient $\Gamma \vdash S <: T$.

Les règles de typage deviennent :

\begin{mathpar}
  \inferrule
  {(x : T) \in \Gamma}
  {\Gamma \vdash x : T}
  \quad (\textsc{T-VAR})
  \and
  \inferrule
  {\Gamma, x : T_{1} \vdash t : T_{2}}
  {\Gamma \vdash \lambda (x : T_{1}) t : T_{1} \rightarrow T_{2}}
  \quad (\textsc{T-ABS})
  \and
  \inferrule
  {\Gamma \vdash u : T_{1} \rightarrow T_{2} \\ \Gamma \vdash v : T_{1}}
  {\Gamma \vdash u \, v : T_{2}}
  \quad (\textsc{T-APP})
  \\
  \inferrule
  {\Gamma, X <: T_{1} \vdash t : T}
  {\Gamma \vdash \lambdaExprType{X <: T_{1}}{t} : \forall X <: T_{1}. \, T}
  \quad (\textsc{T-T-ABS})
  \and
  \inferrule
  {\Gamma \vdash t_{1} : \forall X <: T_{1}. \, T \\ \Gamma \vdash T' <: T_{1}}
  {\Gamma \vdash t_{1}[T'] : [X \rightarrow T']T}

  \quad (\textsc{T-T-APP})
  \and
  \inferrule
  {\Gamma \vdash t : S \\ \Gamma \vdash S <: T}
  {\Gamma \vdash t : T}
  \quad (\textsc{T-SUB})
\end{mathpar}

Quant aux règles de sous-typage, nous avons :

\begin{mathpar}
  \inferrule
  {\Gamma \vdash S <: S}
  {}
  \quad (\textsc{S-REFL})
  \and
  \inferrule
  {\Gamma \vdash S <: T \\ \Gamma \vdash T <: U}
  {\Gamma \vdash S <: U}
  \quad (\textsc{S-TRANS})
  \and
  \inferrule
  {\Gamma \vdash S <: Top}
  {}
  \quad (\textsc{S-TOP})
  \\
  \inferrule
  {X <: T \in \Gamma}
  {\Gamma \vdash X <: T}
  \quad (\textsc{S-TVAR})
  \and
  \inferrule
  {\Gamma \vdash T_{1} <: S_{1} \\ \Gamma \vdash S_{2} <: T_{2}}
  {\Gamma \vdash S_{1} \rightarrow S_{2} <: T_{1} \rightarrow T_{2}}
  \quad (\textsc{S-ARROW})
  \and
  \inferrule
  {\Gamma \vdash T_{1} <: S_{1} \\ \Gamma, X <: T_{1} \vdash S_{2} <: T_{2}}
  {\Gamma \vdash \forall X <: S_{1}. S_{2} <: \forall X <: T_{1}. T_{2}}
  \quad (\textsc{S-ALL})
\end{mathpar}

Remarquons que dans (S-ALL), la borne supérieur de la variable $X$ peut être différente.

\section{Sûreté}

Les théorèmes de préservation et de progression restent vrais pour Système
$F_{<:}$. Cependant, nous ne les démontrerons pas car ils nécessitent des lemmes
techniques ainsi qu'une définition formelle de contexte bien formé et ce n'est
pas le sujet principal de ce document. La structure
et les techniques restent les mêmes : lemme d'inversion des relations de typage,
lemme des formes canoniques, lemme d'inversion de la relation de sous-typage, lemme de
substitution des types, preuve par induction structurelle, etc.
Des preuves des théorèmes peuvent être trouvées dans
\cite{tapl-bounded-quantification}.

\section{Indécidabilité du sous-typage}

Etant donnés deux types $S$ et $T$, la question $S <: T$ a-t-elle
toujours une réponse ? En d'autres termes, la question du sous-typage est-elle
décidable ?

Il a été démontré qu'il n'existe pas d'algorithme de sous-typage donnant une
réponse sur toute question $S <: T$. Plus d'informations peuvent être trouvées
dans \cite{tapl-bounded-quantification-metatheory}.

Le problème de l'indécidabilité du sous-typage de Système $F_{<:}$ est important
quand nous considérons l'écriture d'un algorithme de sous-typage pour ce calcul.
En effet, cela implique que sur certaines entrées, l'algorithme peut
boucler\footnote{Si bien sûr nous ne définissons pas des règles pour l'arrêter
  dans le cas où il n'y a pas de réponse.}.