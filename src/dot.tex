\chapter{Enregistrement avec type chemin dépendant}

Faire le lien avec le sujet initial.

\section{Implémentation}

- Préciser que tout est écrit en OCaml, avec ocamllex et menhir comme lexeur et parseur.

\subsection{Gestion des variables}

- Utilisation de AlphaLib pour les variables libres et liées ainsi que
l'exploration de l'AST. Décrire les avantages d'AlphaLib.

- Avoidance problem = le problème d'échappement. Donner les exemples qui sont
dans \verb|dsubml/test/typing/simple_wrong.dsubml|.

- Gestion d'un environnement non vide, avec une librairie standard. On regarde
alors maintenant les termes top level qui sont composés soit d'un let top level
(sans in), soit d'un terme. Les termes top level sont là pour étendre
l'environnement tandis que les termes usuels non. Dans l'implémentation, cela se
traduit par un type somme Grammar.TopLevelTerm. Lorsque nous rencontrons un
terme top level qui est un let, nous appelons la fonction \verb|$read_top_level_term|
qui se charge de...

- Possibilité de voir l'arbre de dérivation de typage.

- Utilisation du terme Unimplemented.

\subsection{Complexité des algorithmes}

- Donner un détail sur la complexité des algorithmes de sous-typages et de
typages. Expliquer pourquoi on a supprimé la règle REFL pour la remplacer par
REFL-TYP et donner la preuve d'équivalence (qui est directe, en quelques mots).

\subsection{Système d'\og actions \fg}

- Expliquer le système d'actions pour DSubML.

- Problème algorithmique pour les types chemins dépendants. Quel est vraiment le
type de \verb|x.A| ? Parler des types bien formés.

- Implémentation de DSubml et de RML. Insister sur le fait que l'extension n'est
pas si simple pour l'implémentation.

- Algorithme d'inférence (partiel) de types pour les modules.

\subsection{Sucres syntaxiques}

- fun(x : int.T) -> fun(y : int.T) <==> fun(x : int.T, y : int.T)
- sucres syntaxiques pour unit.
- Ascription et check de sous-typage.
- Choix type unimplemented.
- Choix pour le typage d'un entier dans l'algo (bottom, pour ascription).
- Séparation dans le parser des termes qui doivent avoir des parenthèses quand
on les utilise comme paramètres de fonctions et ceux qui n'en ont pas besoin.
- application de fonctions à plusieurs paramètres.

\subsection{Types de bases}

- Implémentation des types de bases.
- Entiers dans le lexeur.
- Enregistrement: RecursiveType mais avec un ID \verb|'self| pour éviter de
pouvoir définir des record récursifs.