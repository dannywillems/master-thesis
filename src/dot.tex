\chapter{Enregistrement avec type chemin dépendant}

Faire le lien avec le sujet initial.

\section{Sureté}

\section{Implémentation}

- Avoidance problem = le problème d'échappement. Donner les exemples qui sont
dans \verb|dsubml/test/typing/simple_wrong.dsubml|.

- Utilisation de AlphaLib pour les variables libres et liées ainsi que
l'exploration de l'AST. Décrire les avantages d'AlphaLib.

- Donner un détail sur la complexité des algorithmes de sous-typages et de
typages. Expliquer pourquoi on a supprimé la règle REFL pour la remplacer par
REFL-TYP et donner la preuve d'équivalence (qui est directe, en quelques mots).

- Gestion d'un environnement non vide, avec une librairie standard. On regarde
alors maintenant les termes top level qui sont composés soit d'un let top level
(sans in), soit d'un terme. Les termes top level sont là pour étendre
l'environnement tandis que les termes usuels non. Dans l'implémentation, cela se
traduit par un type somme Grammar.TopLevelTerm. Lorsque nous rencontrons un
terme top level qui est un let, nous appelons la fonction \verb|$read_top_level_term|
qui se charge de...

- Possibilité de voir l'arbre de dérivation de typage.

- Utilisation du terme Unimplemented.

- Préciser que tout est écrit en OCaml, avec ocamllex et menhir comme lexeur et parseur.

- Expliquer le système d'actions pour DSubML.

- Parler des sucres syntaxiques.