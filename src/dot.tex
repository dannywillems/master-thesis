\chapter{DOT}

%%% TODO: x peut apparaitre dans son type.

Dans ce chapitre, nous présentons DOT, un calcul développé récemment pour le
langage Scala. Ce calcul ajoute les types, alors
appelés types dépendants, dans les enregistrements. Des types
récursifs, c'est-à-dire des types qui peuvent faire référence à eux-mêmes, sont
également définis\footnote{Les types récursifs et leurs règles présents dans DOT ne sont pas
  définis de manière habituelle. Plusieurs chapitres y sont consacrés dans
  \cite{tapl-recursive-types}. Nous n'étudierons pas les différences dans ce document.
}.
Nous montrerons que DOT peut être vu comme une
extension de Système $F_{<:}$ bien que sa syntaxe soit différente et ne possède
pas de variables de type.

Plusieurs définitions du calcul DOT existent et sont dispersées à travers
plusieurs documents comme \cite{nada-amin-thesis}, \cite{OOPSLA-DOT-2016},
\cite{POPL-2017-DOT} ou encore \cite{WF-DOT-2016}. Dans ce document, nous avons
fait le choix d'utiliser \cite{WF-DOT-2016} car la syntaxe et les règles sont
proches des calculs présentés dans ce mémoire.

\section{Syntaxe}

La syntaxe des termes de DOT est définie par la grammaire suivante :
\begin{minipage}{0.45\textwidth}
  \begin{align*}
    t ::= & \, & \text{terme} \\
          & \; x, y & \text{var} \\
          & \; \lambdaExpr{x : T}{t} & \text{abs} \\
          & \; x \, y & \text{app} \\
          & \; \localLetBinding{x}{t}{t} & \text{let} \\
          & \; \nu(x : T^{x})d & \text{rec} \\
          & \; x.a & \text{champ proj} \\
\end{align*}
\end{minipage}
\begin{minipage}{0.45\textwidth}
  \begin{align*}
    d ::= & \, & \text{decl} \\
          & \; \left\{ a = t \right\} & \text{champ} \\
          & \; \left\{ A = T \right\} & \text{type} \\
          & \; d \wedge d & \text{aggregation}
\end{align*}
\end{minipage}

et la syntaxe des types par la grammaire suivante :

\begin{align*}
  S, T ::= & \, & \text{type} \\
           & \; Top & \text{top} \\
           & \; Bottom & \text{bottom} \\
           & \; \forall(x : S) T^{x} & \text{fonction} \\
           & \; \left\{ A : S .. T \right\} & \text{type decl} \\
           & \; \left\{ a : T \right\} & \text{champ decl} \\
           & \; x.A & \text{type proj} \\
           & \; \mu(x : T^{x}) & \text{rec} \\
           & \; S \wedge T & \text{inter}
\end{align*}

Nous retrouvons (var) et (abs) pour les variables et les
abstractions.
À la différence des autres calculs, une application n'est pas constituée de deux
termes mais de deux variables\footnote{Nous pouvons cependant utiliser (let)
  pour retrouver des applications entre termes.}. Nous retrouvons la projection d'un champ avec
(champ proj). Comme pour les applications, seule une variable peut être
utilisée\footnote{Même remarque que pour les applications.}.

(let) permet de créer des variables locales comme il est possible de le faire en OCaml.
(let) est utilisé pour lier un terme à une variable et pouvoir utiliser cette
dernière dans un autre terme\footnote{On dit aussi que $x$ est est un binding
  local.}.

Enfin, (rec) est l'équivalent des enregistrements à la différence qu'il permet
de définir des termes récursifs, c'est-à-dire des termes
dont le corps peut faire référence à eux-mêmes à travers la variable $x$ qui
doit obligatoirement être accompagnée d'un type. Le
corps d'un terme récursif est une déclaration. Une déclaration est soit un
couple $(a, t)$ où $a$ est un nom de champ et $t$ un terme comme pour les
enregistrements, soit une déclaration de type représentée par un couple $(A,
T)$ où $A$ est le nom de la déclaration et $T$ le type associé. Un enregistrement peut contenir plusieurs déclarations de champ et
de type en utilisant (aggregation). Ces déclarations peuvent être mutuellement
dépendantes grâce à la variable $x$.
Le domaine d'une déclaration $d$, noté $dom(d)$, est défini comme l'ensemble des
labels.

Quant aux types, nous retrouvons $Top$ et $Bottom$ comme décrits dans le
chapitre \ref{chapter:lambda-calculus-with-records}. Le type fonction est
également présent avec une syntaxe différente et il devient dépendant,
c'est-à-dire que le type de retour peut dépendre du paramètre\footnote{La
  notation $T^{x}$ signifie que $x$ peut être libre dans $T$.}. (champ decl) est le type d'un
champ d'un enregistrement. (type decl) est le type d'une déclaration d'un type
dans un enregistrement et possède une borne inférieure $S$ et une borne
supérieure $T$. (inter) permet de typer le corps d'un enregistrement contenant
plusieurs déclarations et ce dernier est encapsulé dans un type récursif grâce à (rec). Les
types des déclarations peuvent être mutuellement dépendants grâce à la variable
$x$ du type récursif.

Enfin, les types dépendants (type proj), en parallèle de (champ proj),
permettent d'accéder à un type interne d'un enregistrement.

L'écriture de programmes est lourde et fastidieuse. Des exemples concrets et
écrits dans un langage plus lisible et proche d'OCaml seront donnés dans le
chapitre \ref{chapter:rml}.

\section{Sémantique}

La sémantique est laissée de côté actuellement.

\section{Règles de typage}

Nous séparons les règles de typage en deux catégories : celles pour les termes
et celles pour les définitions.

Les règles de typage pour les termes sont les suivantes :

\begin{mathpar}
  \inferrule
  {\Gamma, x : T, \Gamma' \vdash x : T}
  {} \quad (\textsc{T-VAR})
  \and
  \inferrule
  {\Gamma \vdash t : S \\ \Gamma \vdash S <: T}
  {\Gamma \vdash t : T} \quad (\textsc{T-SUB})
  \and
  \inferrule
  {\Gamma, x : T \vdash t : U \\ x \notin FV(T)}
  {\Gamma \vdash \lambdaExpr{x : T}{t} : \forall(x : T) U^{x}} \quad (\textsc{ALL-I})
  \and
  \inferrule
  {\Gamma \vdash x : \forall(z : S) T^{z} \\ \Gamma \vdash y : S}
  {\Gamma \vdash x \; y : [z := y] T^{z}} \quad (\textsc{ALL-E})
  \and
  \inferrule
  {\Gamma \vdash t : T \\ \Gamma, x : T \vdash u : U \\ x \notin FV(U)}
  {\Gamma \vdash \localLetBinding{x}{t}{u} : U} \quad (\textsc{LET})
  \and
  \inferrule
  {\Gamma \vdash x : T \\ \Gamma \vdash x : U}
  {\Gamma \vdash x : T \wedge U} \quad (\textsc{AND-I})
  \and
  \\
  \inferrule
  {\Gamma, x : T \vdash d : T}
  {\Gamma \vdash \nu(x : T^{x})d : \mu(x : T^{x})} \quad (\textsc{$\left\{ \; \right\}$-I})
  \and
  \inferrule
  {\Gamma \vdash x : T^{x}}
  {\Gamma \vdash x : \mu(z : T^{z})}  \quad (\textsc{VAR-PACK})
  \and
  \inferrule
  {\Gamma \vdash x : \mu(z : T^{z})}
  {\Gamma \vdash x : T^{x}}  \quad (\textsc{VAR-UNPACK})
  \and
  \inferrule
  {\Gamma \vdash x : \left\{ a : T \right\}}
  {\Gamma \vdash x.a : T} \quad (\textsc{FLD-E})
\end{mathpar}

Les règles de typage pour les définitions sont :

\begin{mathpar}
  \inferrule
  {}
  {\Gamma \vdash \left\{ A = T \right\} : \left\{ A : T  .. T \right\}} \quad (\textsc{TYP-I})
  \and
  \inferrule
  {\Gamma \vdash d_{1} : T_{1} \\ \Gamma \vdash d_{2} : T_{2} \\ dom(d_{1})
    \cap dom(d_{2}) = \emptyset}
  {\Gamma \vdash d_{1} \wedge d_{2} : T_{1} \wedge T_{2}} \quad (\textsc{ANDDEF-I})
  \and
  \inferrule
  {\Gamma \vdash t : T}
  {\Gamma \vdash \left\{ a : t \right\} : \left\{ a : T \right\}} \quad (\textsc{FLD-I})
\end{mathpar}

Nous retrouvons les règles (T-VAR) et (T-SUB) comme dans les précédents calculs.
(ALL-I) est l'équivalent de (T-ABS) et (ALL-E) est l'équivalent de (T-APP).
Remarquons que comme $T$ peut dépendre de $z$, il est nécessaire d'effectuer une
substitution de $z$ par $y$ dans le type de $x y$ car $z$ n'existe pas en
dehors du type fonction.

(LET) nous dit que le terme $t$ et la variable $x$ à laquelle nous lions ce
terme doivent avoir le même type, le type du terme en entier est celui de $u$.
(AND-I) nous dit que si une variable a deux types $T$ et $U$, alors elle a
également le type intersection $T \wedge U$. ($\left\{ \; \right\}$-I) type les termes
récursifs et oblige la variable du terme récursif à avoir le même type que celui dans la déclaration.
(FLD-E) est l'équivalent de (T-PROJ) que nous avions défini pour les
enregistrements.
Pour finir, (VAR-PACK) nous dit que nous pouvons
contruire un type récursif à partir de n'importe quel autre type. De l'autre
coté, (VAR-UNPACK) nous dit qu'un terme ayant un type récursif possède également
le type à l'intérieur de ce dernier.

Quant aux définitions, (TYP-I) donne le même type à la borne inférieure et à la
borne supérieure d'une déclaration de type. (FLD-I) ressemble à (T-RCD)
restreint à un enregistrement avec un champ. (ANDDEF-I) type l'union de deux
déclarations dont les champs ont obligatoirement des noms différents.

\section{Règles de sous-typage}

\begin{mathpar}
  \inferrule
  {\Gamma \vdash T <: Top}
  {}
  \quad (\textsc{S-TOP})
  \and
  \inferrule
  {\Gamma \vdash Bottom <: T}
  {}
  \quad (\textsc{S-Bottom})
  \and
  \inferrule
  {\Gamma \vdash S <: T \\ \Gamma \vdash T <: U}
  {\Gamma \vdash S <: U}
  \quad (\textsc{S-TRANS})
  \and
  \inferrule
  {\Gamma \vdash T <: T}
  {}
  \quad (\textsc{S-REFL})
  \and
  \inferrule
  {\Gamma \vdash S_{2} <: S_{1} \\ \Gamma, x : S_{2} \vdash T_{1} <: T_{2}}
  {\Gamma \vdash \forall(x : S_{1}) T_{1} <: \forall(x : S_{2}) T_{2}}
  \quad (\textsc{ALL<:ALL})
  \and
  \inferrule
  {\Gamma \vdash x : \left\{ A : S .. T \right\}}
  {\Gamma \vdash S <: x.A}
  \quad (\textsc{<: SEL})
  \and
  \inferrule
  {\Gamma \vdash x : \left\{ A : S .. T \right\}}
  {\Gamma \vdash x.A <: T}
  \quad (\textsc{SEL <:})
  \and
  \inferrule
  {\Gamma \vdash T \wedge U <: T}
  {}
  \quad (\textsc{AND-1-<:})
  \and
  \inferrule
  {\Gamma \vdash T \wedge U <: U}
  {}
  \quad (\textsc{AND-2-<:})
  \and
  \inferrule
  {\Gamma \vdash S <: T \\ \Gamma \vdash S <: U}
  {\Gamma \vdash S <: T \wedge U}
  \quad (\textsc{<: AND})
  \and
  \inferrule
  {\Gamma \vdash S_{2} <: S_{1} \\ \Gamma \vdash T_{1} <: T_{2}}
  {\Gamma \vdash \left\{ A : S_{1} .. T_{1} \right\} <: \left\{ A : S_{2} ..
      T_{2} \right\}}
  \quad (\textsc{TYP<:TYP})
  \and
  \inferrule
  {\Gamma \vdash T <: U}
  {\Gamma \vdash \left\{ a : T \right\} <: \left\{ a : U \right\}}
  \quad (\textsc{FLD <: FLD})
\end{mathpar}

Comme pour le chapitre \ref{chapter:lambda-calculus-with-records}, nous
avons (S-REFL), (S-TRANS), (S-TOP), (S-BOTTOM) et l'équivalent de (S-ARROW), (ALL<:ALL).

(<: SEL) (resp. (SEL <:)) nous dit qu'une projection de type est un super-type (resp.
sous-type) de sa borne inférieure (resp. supérieure).

(FLD <: FLD) est l'équivalent de (S-RCD-DEPTH) pour un enregistrement à un
unique champ.

À travers (TYP<:TYP), les déclarations de type sont naturellement contravariantes
pour les bornes inférieures et covariantes pour les bornes supérieures.

Pour les intersections, les règles (AND-1-<:) et (AND-2-<:) nous donnent
l'équivalent de (S-RCD-WIDTH). Quant à (<:AND), cette dernière règle nous dit
que si un même type est sous-type de deux types différents, alors il est
également sous-type de leur intersection.

Bien que l'analogie avec les enregistrements et les règles de sous-typage du
chapitre \ref{chapter:lambda-calculus-with-records} permette de comprendre
l'utilité des règles, il faut remarquer que ces dernières sont plus expressives. En
effet, le type intersection n'est pas défini uniquement pour les déclarations,
mais pour n'importe quel type. Cela signifie qu'il est autorisé d'écrire $\mu(x
: T) \wedge \left\{ a : S .. T \right\}$ ou encore $x.A \wedge \left\{ a : S ..
  T \right\} \wedge y.A \wedge \mu(x : U)$.
De même, la variable $x$ dans un type dépendant n'est pas nécessairement un
terme récursif mais peut être une abstraction. Par exemple, nous pouvons écrire
$let \; y = \lambdaExpr{x : T}{t} \; in \; y.A$. Bien que ça n'ait pas de sens, la
syntaxe le permet.
La difficulté de DOT réside dans cette expressivité.

\section{Problème d'échappement}

Supposons que nous ayons un type $\real$ et les nombres et prenons l'exemple suivant :
\begin{align*}
  & let \; y = \\
  &\; \; \; \nu(x : \mu(z : \left\{ A : Bottom .. Top \right\} \wedge \left\{ a : z.A \right\})) \left\{ A = \real \right\} \wedge \left\{ a = 5 \right\} \\
  & in \; y.a
\end{align*}
Le type de l'expression est $y.A$. Cependant, la variable $y$ n'existe pas en
dehors du let car c'est une variable locale. C'est un exemple du
\textit{problème d'échappement}\footnote{\og Avoidance problem \fg \; en
anglais.}. C'est pour éviter cela que la condition $x \notin
FV(U)$ est présente dans la règle (LET).

Nous verrons dans le chapitre \ref{chapter:rml} qu'il est nécessaire d'y faire
attention lors de l'implémentation, en particulier lorsque nous faisons des
bindings locaux de variables.


\section{Problème de mauvaises bornes}

DOT permet de définir des déclarations de types avec une borne inférieure et une
borne supérieure. Cette liberté dans les types, bien qu'intéressante à première
vue, pose un problème dans le système de sous-typage.
En effet, le type $\left\{ A : Top .. Bottom \right\}$ est autorisé. Cependant,
si $x : \left\{ A : Top .. Bottom \right\}$, alors nous pouvons montrer que
$Top <: Bottom$ en utilisant (SEL <:), (<: SEL) et la transitivité. Nous avons
alors $\Gamma, x : \left\{ A : Top .. Bottom \right\} \vdash S <: T$ pour tous
$S$ et $T$.
\footnote{Divers articles sur DOT \cite{OOPSLA-DOT-2016} avancent que restreindre la
déclaration (type) $\left\{ A = T \right\}$ à une égalité permet d'éviter ce
problème lors de l'exécution.}

Nous disons qu'un type est \textit{bien formé} s'il ne possède pas de mauvaises
bornes, sinon nous disons qu'il est \textit{mal formé}. Nous disons qu'un
contexte est \textit{bien formé} s'il ne contient que des variables dont les types
sont bien formés. Sinon, le contexte est dit \textit{mal formé}.

\section{Encodage de Système $F_{<:}$}

Bien que la syntaxe soit différente de Système $F_{<:}$ et que DOT ne comporte
pas de variables de type, nous allons montrer que DOT permet d'encoder Système
$F_{<:}$. De plus, les jugements vrais pour le système de typage et de
sous-typage de Système $F_{<:}$ restent vrais pour le système de typage et de
sous-typage de DOT à travers cet encodage.

Notons $\mathcal{V}_{DOT}$ l'ensemble des variables de DOT et $\mathcal{V}_{T,
  F}$ l'ensemble des variables de type de Système $F_{<:}$. Ces ensembles 
étant infinis dénombrables, il existe une fonction injective $f$ entre
$\mathcal{V}_{T, F}$ et $\mathcal{V}_{DOT}$. Nous notons $x_{X} \in
\mathcal{V}_{DOT}$ l'image de $X \in \mathcal{V}_{T, F}$ par la fonction $f$.

Nous définissons alors la fonction ${}^{*}$ qui à chaque terme (resp. type) de
Système $F_{<:}$ associe un terme (resp. type) de DOT.

\begin{align*}
  X^{*} & = && x_{X}.A \\
  Top^{*} & = && Top \\
  (S \rightarrow T)^{*} & = && \forall(x : S^{*}) T^{*} \\
  (\forall X <: S . \, T)^{*} & = && \forall(x_{X} : \left\{ A : Bottom .. S^{*} \right\}) T^{*} \\ \\
  x^{*} & = && x \\
  (\lambdaExpr{x : T}{t})^{*} & = && \lambdaExpr{x : T^{*}}{t^{*}} \\
  (\lambdaExprType{X <: S}t)^{*} & = && \lambdaExpr{x_{X} : \left\{ A : Bottom .. S^{*} \right\}}{t^{*}} \\
  (t \; u)^{*} & = && let \; x \; = \; t^{*} \; in \\
              &  && let \; y \; = \; u^{*} \; in \\
              &  && x \; y \\
   (t[U])^{*}  & = && let \; x \; = \; t^{*} \; in \\
              &  && let \; y_{Y} \; = \; \nu(z : \left\{ A : U^{*} .. U^{*} \right\}) \left\{ A = U^{*} \right\} \; in \\
              &  && x \; y_{Y}
\end{align*}
Quant aux contextes, la fonction ${}^{*}$ est définie naturellement
inductivement par :
\begin{align*}
  (X <: Top)^{*} & = x_{X} : \left\{ A : Bottom .. Top \right\} \\
  (x : T)^{*} & = x : T^{*}
\end{align*}

Cependant, DOT est plus riche syntaxiquement que Système $F_{<:}$ car, par
exemple, il n'est pas possible en $F_{<:}$ de donner une borne inférieure à la
variable d'une abstraction alors que DOT le permet.


Notons $\Gamma \vdash_{F} t : T$ (resp. $\Gamma \vdash_{D} t : T$) un jugement
de typage pour le système de typage de Système $F_{<:}$ (resp. DOT). Nous faisons
de même pour $\Gamma \vdash_{F} S <: T$.

\begin{theorem}
  Si $\Gamma \vdash_{F} S <: T$, alors $\Gamma^{*} \vdash_{D} S^{*} <: T^{*}$.
\end{theorem}

\begin{proof}
  Par induction sur le jugement de sous-typage $\Gamma \vdash_{F} S <: T$.
  \begin{itemize}
  \item[$\bullet$] (S-TVAR). Utilisation de (<: SEL).
  \item[$\bullet$] (S-ALL). Pour le membre de gauche, nous utilisons (TYP<:TYP).
  \item[$\bullet$] Les cas (S-REFL), (S-TRANS), (S-TOP) et (S-ARROW) sont directs
    en utilisant l'encodage et l'hypothèse de récurrence.
    %Nous avons
    %\begin{mathpar}
    %  \inferrule
    %  {\Gamma \vdash T_{1} <: S_{1} \\ \Gamma, X \vdash S_{2} <: T_{2}}
    %  {\Gamma \vdash \forall X <: S_{1} . S_{2} <: \forall X <: T_{1} . T_{2}}
    %\end{mathpar}
    %et nous devons montrer que
    %\begin{mathpar}
    %  \inferrule
    %  {\Gamma^{*} \vdash_{D} \forall(x : \mu(z : \left\{ A : Bottom .. S_{1}^{*}
    %    \right\})) S_{2}^{*} <: \forall(x : \mu(z : \left\{ A : Bottom .. T_{1}^{*}\right\})) T_{2}^{*}}
    %  {}
    %\end{mathpar}
  \end{itemize}
\end{proof}

\begin{theorem}
  Si $\Gamma \vdash_{F} t : T$, alors $\Gamma^{*} \vdash_{D} t^{*} : T^{*}$.
\end{theorem}

Nous acceptons ce théorème. Une preuve peut être trouvée dans
\cite{WF-DOT-2016} et se fait par induction sur la dérivation $\Gamma \vdash_{F}
t : T$. Celle-ci repose sur deux lemmes intermédiaires pour la
substitution des types dépendants.

% TODO ajouter références d'articles qui disent que le sous-typage de DOT est indécidable.
Cette propriété de DOT implique que la question du sous-typage de DOT est
indécidable\cite{WF-DOT-2016, nada-amin-thesis}.
\footnote{Plus précisément, il est nécessaire de démontrer
$\Gamma^{*} \vdash_{D} S^{*} <: T^{*} \Rightarrow \Gamma \vdash_{F} S <: T$.
Montrer ce dernier lemme n'est pas évident à cause de (S-TRANS) qui
peut faire apparaître un type qui ne provient pas obligatoirement d'un type de
Système $F_{<:}$. Néanmoins, d'après certains articles\cite{OOPSLA-DOT-2016, nada-amin-thesis}, la règle (S--TRANS) peut être
poussée plus haut dans l'arbre de dérivation si celle-ci est utilisée comme
dernière règle. Cette question n'a pas été plus étudiée mais nous
accepterons ces faits pour la suite.}

%% Historique : https://en.wikipedia.org/wiki/CLU_(programming_language)

\section{Sûreté}

Les théorèmes de préservation et de progression
restent vrais. Cependant, nous ne les démontrerons pas dans ce document car ils
nécessitent plusieurs lemmes techniques. Différentes preuves de la sûreté de DOT
existent, en utilisant des techniques et des sémantiques différentes. Nous
redirigeons le lecteur aux références données au début de ce chapitre pour plus d'informations.
