\chapter{DOT}

Faire le lien avec le sujet initial.

\section{Syntaxe}

La syntaxe des termes de DOT est définie par la grammaire suivante :
\begin{minipage}{0.45\textwidth}
  \begin{align*}
    t ::= & \, & \text{terme} \\
          & \; x, y, z & \text{var} \\
          & \; x \, y & \text{app} \\
          & \; \lambdaExpr{x : T}{t} & \text{abs} \\
          & \; \localLetBinding{x}{t}{t} & \text{let} \\
          & \; \nu(x : T)d & \text{rec} \\
          & \; x.a & \text{champ proj} \\
\end{align*}
\end{minipage}
\begin{minipage}{0.45\textwidth}
  \begin{align*}
    d ::= & \, & \text{decl} \\
          & \; \left\{ a : t \right\} & \text{champ} \\
          & \; \left\{ A = T \right\} & \text{type} \\
          & \; d \wedge d & \text{aggregation}
\end{align*}
\end{minipage}

et la syntaxe des types par la grammaire suivante :

\begin{align*}
  S, T ::= & \, & \text{type} \\
           & \; Top & \text{top} \\
           & \; Bottom & \text{bottom} \\
           & \; \left\{ A : S .. T \right\} & \text{type decl} \\
           & \; x.A & \text{type proj} \\
           & \; \forall(x : S) T & \text{fonction} \\
           & \; \left\{ a : T \right\} & \text{champ decl} \\
           & \; \mu(x : T) & \text{rec} \\
           & \; S \wedge T & \text{inter}
\end{align*}

Le domaine d'une déclaration $d$, noté $dom(d)$ est défini comme l'ensemble des
noms de ses champs.

\section{Sémantique}

La sémantique est laissée de coté actuellement.

\section{Règles de typage}

Nous séparons les règles de typage en deux catégories : celles pour les termes
et celles pour les définitions.

Les règles de typage pour les termes sont les suivantes :

\begin{mathpar}
  \inferrule
  {\Gamma, x : T, \Gamma' \vdash x : T}
  {} \quad (\textsc{T-VAR})
  \and
  \inferrule
  {\Gamma \vdash t : S \\ \Gamma \vdash S <: T}
  {\Gamma \vdash t : T} \quad (\textsc{T-SUB})
  \and
  \inferrule
  {\Gamma, x : T \vdash t : U \\ x \notin FV(T)}
  {\Gamma \vdash \lambdaExpr{x : T}{t} : \forall(x : T) U} \quad (\textsc{ALL-I})
  \and
  \inferrule
  {\Gamma \vdash x : \forall(z : S) T \\ \Gamma \vdash y : S}
  {\Gamma \vdash x \; y : [z := y] T} \quad (\textsc{ALL-E})
  \and
  \inferrule
  {\Gamma \vdash t : T \\ \Gamma, x : T \vdash u : U \\ x \notin FV(U)}
  {\Gamma \vdash \localLetBinding{x}{t}{u} : U} \quad (\textsc{LET})
  \and
  \inferrule
  {\Gamma \vdash x : T \\ \Gamma \vdash x : U}
  {\Gamma \vdash x : T \wedge U} \quad (\textsc{AND-I})
  \and
  \\
  \inferrule
  {\Gamma, x : T \vdash d : T}
  {\Gamma \vdash \nu(x : T)d : \mu(x : T)} \quad (\textsc{$\left\{ \; \right\}$-I})
  \and
  \inferrule
  {\Gamma \vdash x : T^{x}}
  {\Gamma \vdash x : \mu(z : T^{z})}  \quad (\textsc{VAR-PACK})
  \and
  \inferrule
  {\Gamma \vdash x : \mu(z : T^{z})}
  {\Gamma \vdash x : T^{x}}  \quad (\textsc{VAR-UNPACK})
  \and
  \inferrule
  {\Gamma \vdash x : \left\{ a : T \right\}}
  {\Gamma \vdash x.a : T} \quad (\textsc{FLD-E})
\end{mathpar}

Et les règles de typage pour les définitions sont :

\begin{mathpar}
  \inferrule
  {\Gamma \vdash \left\{ A = T \right\}} 
  {\Gamma \vdash \left\{ A : T  .. T \right\}} \quad (\textsc{TYP-I})
  \and
  \inferrule
  {\Gamma \vdash d_{1} : T_{1} \\ \Gamma \vdash d_{2} : T_{2} \\ dom(d_{1})
    \cap dom(d_{2}) = \emptyset}
  {\Gamma \vdash d_{1} \wedge d_{2} : T_{1} \wedge T_{2}} \quad (\textsc{ANDDEF-I})
  \and
  \inferrule
  {\Gamma \vdash t : T}
  {\Gamma \vdash \left\{ a : t \right\} : \left\{ a : T \right\}} \quad (\textsc{FLD-I})
\end{mathpar}

Nous retrouvons les règles (T-VAR) et (T-SUB) comme dans les précédents calculs.
(ALL-I) est l'équivalent de (T-ABS) et (ALL-E) est l'équivalent de (T-APP).
(LET) nous dit que le terme $t$ et la variable $x$ à laquelle nous lions ce
terme doivent avoir le même type, le type du terme en entier est celui de $u$.
(AND-I) nous dit que si une variable a deux types $T$ et $U$, alors il a
également le type intersection $T \wedge U$. ($\left\{ \; \right\}$-I) type les terme
récursifs et oblige que la variable du terme récursif ait le même type que la déclaration.
(FLD-E) est l'équivalent de (T-PROJ) que nous avions défini pour les
enregistrements.
Pour finir, (VAR-PACK) nous dit que lorsque nous pouvons
contruire un type récursifs à partir de n'importe quel autre type. De l'autre
coté, (VAR-UNPACK) nous dit qu'un terme ayant un type récursif possède également
le type à l'intérieur de ce dernier.

Quant aux définitions, (TYP-I) donne le même type à la borne inférieure et à la
borne supérieure d'une déclaration de type. (FLD-I) ressemble à (T-RCD)
restreint à un enregistrement avec un champ. (ANDDEF-I) type l'union de deux
déclarations dont les champs ont obligatoirement des noms différents.

\subsection*{Problème d'échappement}

Supposons que nous ayons un type $\real$ et prenons l'exemple suivant :
\begin{align*}
  & let \; y = \\
  &\; \; \; \nu(x : \mu(z : \left\{ A : Bottom .. Top \right\} \wedge \left\{ a : z.A \right\})) \left\{ A : \real \right\} \wedge \left\{ a : z.A \right\} \\
  & in \; y.a
\end{align*}

Le type de l'expression est $y.A$. Cependant, la variable $y$ n'existe pas en
dehors du let car c'est une variable locale. C'est un exemple du
\textit{problème d'échappement}\footnote{\og Avoidance problem \fg \; en
anglais.}. Pour éviter cela, dans la règle (LET), une condition supplémentaire
est ajoutée : $x \notin FV(U)$.

Nous verrons dans le chapitre \ref{chapter:rml} qu'il est nécessaire d'y faire
attention lors de l'implémentation, en particulier lorsque nous faisons des
bindings locaux de variables.

\section{Règles de sous-typage}

\begin{mathpar}
  \inferrule
  {\Gamma \vdash T <: Top}
  {}
  \quad (\textsc{S-TOP})
  \and
  \inferrule
  {\Gamma \vdash Bottom <: T}
  {}
  \quad (\textsc{S-Bottom})
  \and
  \inferrule
  {\Gamma \vdash S <: T \\ \Gamma \vdash T <: U}
  {\Gamma \vdash S <: U}
  \quad (\textsc{S-TRANS})
  \and
  \inferrule
  {\Gamma \vdash T <: T}
  {}
  \quad (\textsc{S-REFL})
  \and
  \inferrule
  {\Gamma \vdash S_{2} <: S_{1} \\ \Gamma, x : S_{2} \vdash T_{1} <: T_{2}}
  {\Gamma \vdash \forall(x : S_{1}) T_{1} <: \forall(x : S_{2}) T_{2}}
  \quad (\textsc{ALL<:ALL})
  \and
  \inferrule
  {\Gamma \vdash x : \left\{ A : S .. T \right\}}
  {\Gamma \vdash S <: x.A}
  \quad (\textsc{<: SEL})
  \and
  \inferrule
  {\Gamma \vdash x : \left\{ A : S .. T \right\}}
  {\Gamma \vdash x.A <: T}
  \quad (\textsc{SEL <:})
  \and
  \inferrule
  {\Gamma \vdash T \wedge U <: T}
  {}
  \quad (\textsc{$AND_{1}$ <:})
  \and
  \inferrule
  {\Gamma \vdash T \wedge U <: U}
  {}
  \quad (\textsc{$AND_{2}$ <:})
  \and
  \inferrule
  {\Gamma \vdash S <: T \\ \Gamma \vdash S <: U}
  {\Gamma \vdash S <: T \wedge U}
  \quad (\textsc{<: AND})
  \and
  \inferrule
  {\Gamma \vdash S_{2} <: S_{1} \\ \Gamma \vdash T_{1} <: T_{2}}
  {\Gamma \vdash \left\{ A : S_{1} .. T_{1} \right\} <: \left\{ A : S_{2} ..
      T_{2} \right\}}
  \quad (\textsc{TYP<:TYP})
  \and
  \inferrule
  {\Gamma \vdash T <: U}
  {\Gamma \vdash \left\{ a : T \right\} <: \left\{ a : U \right\}}
  \quad (\textsc{FLD <: FLD})
\end{mathpar}

\section{Encodage de System $F_{<:}$}

Montrer l'inclusion comme dans WF, comment les variables de types sont gérées.

Cette propriété de DOT implique que la question du sous-typage de DOT est
indécidable.

\section{Notion de bonne formation}


\section{Sûreté}

Ne pas tout démontrer, voir les théorèmes dans WF.