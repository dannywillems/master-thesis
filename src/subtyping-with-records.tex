\chapter{$\lambda$-calcul avec sous-typage et enregistrements.}
\label{chapter:lambda-calculus-with-records}

La syntaxe des termes du $\lambda$-calcul simplement typé est pauvre : nous ne pouvons définir que des
variables, des fonctions et appliquer des termes entre eux.
La plupart des langages de programmation fournissent diverses structures de
données comme les paires, les tuples, ou encore les enregistrements.

Un enregistrement est ensemble fini de couples $(l_{i}, t_{i})$, noté $\left\{
  l_{i} = t_{i} \right\}^{1 \leq i \leq n}$ où $l_{i}$ est
un label pour le terme $t_{i}$, les couples étant séparés par des
points-virgules. Par exemple, nous pouvons représenter un point d'un plan par
ses coordonnées cartésiennes, nommées $x$ et $y$, et par les termes $t_{x}$ et
$t_{y}$ pour la valeur des coordonnées. Nous notons ce terme $\left\{ x = t_{x}
; y = t_{y} \right\}$. Il est également intéressant de pouvoir récupérer une des
coordonnées d'un point. Pour cette raison, nous ajoutons le terme $t.l$ qui permet d'accéder au label $l$ du terme $t$.

Il nous faut également typer les enregistrements. Pour cela, nous ajoutons une
nouvelle syntaxe dans les types, noté $\left\{l_{i} : T_{i}\right\}^{1 \leq i
  \leq n}$ où $l_{i}$
est un label de l'enregistrement et $T_{i}$ le type du terme référencé par le
label $l_{i}$. Pour l'exemple du point dans le plan réel, si nous supposons avoir un
type $\real$ pour les réels, notre point serait de type $\left\{ x : \real ; y : \real
\right\}$. Pour le typage des projections, il est naturel de dire que le type de
$t.l_{i}$ soit le type du terme $t_{i}$ de l'enregistrement $t$.

Dans ce chapitre, nous allons étendre notre ensemble de termes avec les
enregistrements ainsi que définir les règles d'évaluation et de typage pour ces
nouveaux termes pour enfin introduire dans un second temps la notion de
sous-typage qui définit le principe du \textit{polymorphisme par sous-typage}.

\section{$\lambda$-calcul simplement typé avec enregistrements}

\section*{Syntaxe}

Formellement, la syntaxe des termes et la syntaxe des types sont définies par
les grammaires suivantes :

\begin{minipage}{0.45\textwidth}
  \begin{align*}
    t ::= & \, & \text{terme} \\
          & \; x & \text{var} \\
          & \; t \, t & \text{app} \\
          & \; \lambdaExpr{x : T}{t} & \text{abs} \\
          & \; \left\{ l_{i} = t_{i} \right\}^{1 \leq i \leq n} & \text{enreg} \\
          & \; t.l & \text{proj}
  \end{align*}
\end{minipage}
\label{syntax-terms:lambda-calculus-with-records}
\begin{minipage}{0.45\textwidth}
  \begin{align*}
    T ::= & \, & \text{type} \\
          & \; T \rightarrow T & \text{fonction} \\
          & \; \left\{ l_{i} : T_{i} \right\}^{1 \leq i \leq n} & \text{enreg}
  \end{align*}
\end{minipage}
\label{syntax-types:lambda-calculus-with-records}
\\
\\
\\
Nous allons également ajouter les enregistrements dont tous les termes sont des
valeurs comme valeurs de notre langage. Nous obtenons alors la grammaire
suivante :
\label{syntax-values:lambda-calculus-with-records}
\begin{align*}
  v ::= & \, & \text{valeur} \\
        & \; \lambdaExpr{x : T}{t} & \text{abs} \\
          & \; \left\{ l_{i} = v_{i} \right\}^{1 \leq i \leq n} & \text{enreg} \\
\end{align*}

\section*{Sémantique}

Il nous faut également définir comment nous réduisons nos
enregistrements. Nous ajoutons les règles d'évaluation suivantes aux règles
d'évaluation définies dans les chapitres précédents.

\begin{mathpar}
  \inferrule
  {t_{j} \rightarrow t'_{j}}
  {\left\{ l_{1} = v_{1}; ... ; l_{j} = t_{j} ; ... ; l_{n} = t_{n} \right\}
    \rightarrow \\\\ \left\{ l_{1} = v_{1}; ... ; l_{j} = t'_{j} ; ... ; l_{n} =
      t_{n} \right\}} \quad (\textsc{E-RCD})
  \and
  \inferrule
  {t_{1} \rightarrow t'_{1}}
  {t_{1}.l \rightarrow t'_{1}.l} \quad (\textsc{E-PROJ})
  \and
  \inferrule
    {\left\{l_{i} = v_{i} \right\}^{1 \leq i \leq n}.l_{j} \rightarrow v_{j}}
    {} \quad (\textsc{E-PROJ-RCD})
\end{mathpar}
\label{semantics:lambda-calculus-with-records}

La règle (E-RCD) nous dit comment les termes à l'intérieur d'un enregistrement sont
évalués. Quant à (E-PROJ-RCD)\footnote{Nous supposons que $j \in \GSset{1, ...,
n}$.}, elle nous dit que nous pouvons évaluer une projection uniquement si les
termes de l'enregistrement ont tous été réduits à des valeurs. Pour finir,
(E-PROJ) nous dit comment nous simplifions le terme dans une projection.

\section*{Règles de typage}

En plus des règles de typage du $\lambda$-calcul simplement typé, la relation
de typage comprend les règles suivantes :

\begin{mathpar}
  \inferrule
  {\forall i \in \left\{1, ..., n \right\}, \Gamma \vdash t_{i} : T_{i}}
  {\Gamma \vdash \left\{ l_{i} = t_{i} \right\}^{1 \leq i \leq n} : \left\{ l_{i} : T_{i}
    \right\}^{1 \leq i \leq n}}
  \quad (\textsc{T-RCD})
  \and
  \inferrule
  {\Gamma \vdash t_{1} : \left\{ l_{i} : T_{i} \right\}^{1 \leq i \leq n}}
  {\Gamma \vdash t_{1}.l_{j} : T_{j} } 
  \quad (\textsc{T-PROJ})
\end{mathpar}
\label{typing:lambda-calculus-with-records}

La règle (T-RCD) nous dit comment introduire un type enregistrement tandis que
(T-PROJ) nous dit comment typer une projection.

\section{Sous-typage}

Maintenant que nous avons défini la syntaxe, la sémantique et les règles de
typage de notre langage comprenant les enregistrements, nous allons définir la
notion de sous-typage ainsi que les règles pour ce langage.

Le sous-typage est une forme de polymorphisme, c'est-à-dire une possibilité
d'attribuer plusieurs types à un terme. Le polymorphisme est très courant dans
les langages orientés objets et permet, par exemple, d'élargir le champ
d'application des fonctions.
De manière plus concrète, supposons que nous ayons défini un type $\real$
pour l'ensemble des réels et définissons le point $(5, 5)$ du plan $\real^{2}$ par
$\left\{ x = 5 ; y = 5 \right\}$\footnote{En supposant que 5 fasse partie de
  notre syntaxe et que son type soit $\real$.}.
Nous pouvons définir la fonction de projection sur l'axe des abscisses avec
\begin{equation*}
  \lambdaExpr{p : \left\{ x : \real ; y : \real \right\}}{p.x}.
\end{equation*}
Maintenant, nous souhaitons faire de même pour le point $(5, 5, 5)$ de $\real^{3}$,
représenté par $\left\{ x = 5 ; y = 5 ; z = 5 \right\}$.
Nous ne pouvons pas utiliser la fonction de projection définie pour $\real^{2}$
parce que le paramètre de la fonction est un enregistrement ne contenant que les
champs $x$ et $y$. Nous devons donc créer une nouvelle fonction, par exemple 
\begin{equation*}
  \lambdaExpr{p : \left\{ x : \real ; y : \real ; z : \real \right\}}{p.x}.
\end{equation*}
Et si nous voulions continuer pour $\real^{4}$, $\real^{5}$, etc, nous devrions à chaque
fois redéfinir une nouvelle fonction. Cependant, nous remarquons que le corps de
la fonction est toujours le même et qu'il n'a besoin que d'un enregistrement
avec au moins un champ $x$, les autres champs étant inutiles.


C'est dans ce cas que le sous-typage intervient : nous allons définir une
relation entre les types qui permet d'affiner les règles de typage. Nous dirons que
\textbf{$S$ est un sous-type de $T$} ou encore \textbf{$T$ est un supertype de $S$}, noté $S <: T$.

Pour lier la relation de typage à la relation de sous-typage, nous ajoutons une
nouvelle règle de typage

\begin{mathpar}
  \inferrule
  {\Gamma \vdash t : S \\ S <: T}
  {\Gamma \vdash t : T} \quad (\textsc{T-SUB})
\end{mathpar}

Cette dernière permet d'affirmer que si dans un contexte donné, un terme $t$ a le
type $S$ et que le type $S$ est un sous-type de $T$, alors dans le même
contexte, $t$ a le type $T$.

\subsection*{Règles de sous-typage}

Passons à la définition de la relation de sous-typage. La toute première chose
est que nous voulons que cette relation soit réflexive et transitive comme c'est
le cas dans la plupart des langages:

\begin{mathpar}
  \inferrule
  {S <: S}{} \quad (\textsc{S-REFL})
  \and
  \inferrule
  {S <: T \\ T <: U}{S <: U} \quad (\textsc{S-TRANS})
\end{mathpar}

Ensuite, nous aimerions que la relation résolve notre problème, c'est-à-dire que
nous devons pouvoir affirmer qu'un type ayant les champs $x, y, z$
est un sous-type du type ayant uniquement le champ $x$ ou uniquement le champ
$y$. Nous résumons cela par les deux règles suivantes:

\begin{mathpar}
  \inferrule
  {\left\{ l_{i} : T_{i} \right\}^{1 \leq i \leq n + k} <: \left\{ l_{i} :
      T_{i} \right\}^{1 \leq i \leq n}}
  {}
    \quad (\textsc{S-RCD-WIDTH})
  \and
  \inferrule
  {\left\{ k_{j} : S_{j} \right\}^{1 \leq j \leq n} \text{ permutation de }
    \left\{ l_{i} : T_{i} \right\} ^ {1 \leq i \leq n}}
  { \left\{ k_{j} : S_{j} \right\} ^ {1 \leq j \leq n} <:
    \left\{ l_{i} : T_{i} \right\} ^ {1 \leq i \leq n}}
  \quad (\textsc{S-RCD-PERM})
\end{mathpar}

(S-RCD-WIDTH) nous permet de \og laisser de coté \fg \, certain champ tandis que
(S-RCD-PERM) nous dit que l'ordre des champs dans un enregistrement n'a pas d'importance.

Nous souhaitons également prendre en
compte les types dans les enregistrements et leur relation de sous-typage entre
eux. Par exemple, nous aimerions qu'un point du plan $\naturel^{2}$ soit
également un point du plan $\real^{2}$.

\begin{mathpar}
  \inferrule
  {\forall i \in \GSset{1, ..., n}, S_{i} <: T_{i}}
  {\left\{ l_{i} : S_{i} \right\}^{1 \leq i \leq n} <: \left\{ l_{i} : T_{i}
    \right\}^{1 \leq i \leq n}}
  \quad (\textsc{S-RCD-DEPTH})
\end{mathpar}

(S-RCD-DEPTH) nous dit que, étant donnés deux types enregistrements $S$ et $T$
avec les mêmes labels,
si les types des champs de $S$ sont tous sous-types des types des champs de $T$,
alors $S$ est sous-type de $T$.

Pour finir, nous ajoutons une règle pour comparer deux types flèches.
Nous disons que le type flèche est \textit{contravariant} pour les paramètres et
\textit{covariant} pour le type de retour.

\begin{mathpar}
  \inferrule
  {T_{1} <: S_{1} \\ S_{2} <: T_{2}}
  {S_{1} \rightarrow S_{2} <: T_{1} \rightarrow T_{2} }
  \quad (\textsc{S-ARROW})
\end{mathpar}

Nous avons maintenant toutes les propriétés nécessaires pour résoudre notre
problème. En effet, voici un arbre de dérivation qui permet de montrer
$(\lambdaExpr{p : \left\{ x : \real \right\}}{p.x}) \left\{ x = 5 ; y =
  5 \right\} : \real$.

\begin{mathpar}
\inferrule* [Left=(T-APP)]
{\Gamma \vdash \lambdaExpr{p : \left\{ x : \real \right\}}{p.x} : \left\{ x : \real
  \right\} \rightarrow \real
  \\
  \inferrule* [Right=(T-SUB)]
  {\Gamma \vdash \left\{ x = 5 ; y = 5 \right\} : \left\{ x : \real ; y : \real \right\}
    \\\\
    \left\{ x : \real ; y : \real \right\} <: \left\{ x : \real \right\}
  }
  {\Gamma \vdash \left\{ x = 5 ; y = 5 \right\} : \left\{ x : \real \right\}}
}
{\Gamma \vdash (\lambdaExpr{p : \left\{ x : \real \right\}}{p.x}) \left\{ x = 5 ; y =
  5 \right\} : \real}
\end{mathpar}

L'affirmation \inferrule{\left\{ x : \real ; y : \real \right\} <: \left\{ x : \real
  \right\}}{} est inférée grâce à (S-RCD-WIDTH).

\section{Sûreté}

Nous allons montrer que les théorèmes de préservation et de progression
démontrés pour le $\lambda$-calcul simplement typé restent vrais en présence
d'enregistrements et du sous-typage.

Les techniques de preuves utilisées et leur structure ne sont pas très
différentes de celles employées dans le chapitre précédent. Pour ces raisons,
les preuves seront moins détaillées.

\subsection*{Préservation}

\begin{lemma} [Inversion de la règle de sous-typage]
  \label{lemma:record-subtyping-inversion-subtyping-rules}
  \begin{enumerate}
    \item Si $S <: T_{1} \rightarrow T_{2}$, alors $S$ est de la forme $S_{1}
      \rightarrow S_{2}$ avec $T_{1} <: S_{1}$ et $S_{2} <: T_{2}$.
    \item Si $S <: \left\{ l_{i} : T_{i} \right\}^{1 \leq i \leq n}$, alors $S$
      est de la forme $\left\{ k_{i} : S_{i} \right\}^{1 \leq i \leq m}$ tel que
      $(k_{i})_{1 \leq i \leq m} \subseteq (l_{i})_{1 \leq i \leq n}$ et $S_{i}
      <: T_{i}$ pour chaque $i$ tel que $l_{i} = k_{i}$.
  \end{enumerate}
\end{lemma}

\begin{proof}
  La technique de preuve reste identique : pour chaque affirmation, nous
  cherchons quelle règle peut y avoir mené et nous montrons cas par cas, en
  utilisant le principe d'induction sur l'arbre de dérivation. Seul le
  premier point est démontré, le second étant identique. Pour le premier point,
  seules les règles (S-REFL), (S-ARROW) et (S-TRANS) sont possibles.

  \begin{itemize}
  \item (S-REFL). Le résultat est direct car $S = T_{1} \rightarrow T_{2}$.
  \item (S-ARROW). Le résultat est également direct.
  \item (S-TRANS). Nous avons donc l'arbre de dérivation suivant:
    \begin{mathpar}
      \inferrule
      {S <: T \\ T <: T_{1} \rightarrow T_{2}}
      {S <: T_{1} \rightarrow T_{2}}
    \end{mathpar}
    Nous appliquons d'abord l'hypothèse de récurrence sur l'affirmation $T <:
    T_{1} \rightarrow T_{2}$ et ensuite sur l'affirmation $S <: T$. Nous
    concluons en utilisant (S-TRANS).
  \end{itemize}
\end{proof}

Nous montrons également un lemme d'inversion des règles de typage comme il a été
démontré pour le $\lambda$-calcul simplementé typé.

\begin{lemma} [d'inversion des règles de typage]
  \label{lemma:subtyping-record-inversion-typing-rules}
  \begin{enumerate}
    \item Si $\Gamma \vdash \lambdaExpr{x : S_{1}} s_{2} : T_{1} \rightarrow
    T_{2}$, alors $T_{1} <: S_{1}$ et $\Gamma, x : S_{1} \vdash s_{2} : T_{2}$.
    \item Si $\Gamma \vdash \left\{ k_{i} = s_{i} \right\}^{1 \leq i \leq n} :
      \left\{ l_{i} : T_{i} \right\}^{1 \leq i \leq m}$, alors $(l_{i})^{1 \leq
        i \leq m} \subseteq (k_{i})^{1 \leq i \leq n}$ et $\Gamma \vdash s_{i} :
      T_{i}$ pour $k_{i} = l_{i}$.
  \end{enumerate}
\end{lemma}

\begin{proof}
  Par induction sur l'arbre de typage de $\Gamma \vdash t : T$.
  Encore une fois, nous ne montrons que pour le premier cas, les arguments étant
  sensiblement les mêmes pour le second.

  Les seuls règles possibles sont (T-SUB) ou (T-ABS).

  \begin{enumerate}
    \item (T-SUB). Nous avons donc l'arbre de dérivation suivant:
      \begin{mathpar}
        \inferrule* [Left=(T-SUB)]
        {\Gamma \vdash \lambdaExpr{x : S_{1}}{s_{2}} : T \\ T <: T_{1}
          \rightarrow T_{2}}
        {\Gamma \vdash \lambdaExpr{x : S_{1}}{s_{2}} : T_{1} \rightarrow T_{2}}
      \end{mathpar}
      En appliquant le lemme \ref{lemma:record-subtyping-inversion-subtyping-rules}
      sur l'affirmation $T <: T_{1} \rightarrow T_{2}$, nous obtenons $T =
      \tilde{S_{1}} \rightarrow \tilde{S_{2}}$ avec $T_{1} <: \tilde{S_{1}}$ et $\tilde{S_{2}} <: T_{2}$.
      L'affirmation de gauche devient donc
      \begin{mathpar}
        \inferrule
        {\Gamma \vdash \lambdaExpr{x : S_{1}}{s_{2}} : \tilde{S_{1}} \rightarrow \tilde{S_{2}}}
        {}
      \end{mathpar}
      En appliquant l'hypothèse de récurrence sur ce jugement de typage, nous
      obtenons $\tilde{S_{1}} <: S_{1}$ et $\Gamma, x : S_{1} \vdash s_{2} :
      \tilde{S_{2}}$.
      
      En utilisant (S-TRANS) avec $T_{1} <: \tilde{S_{1}}$ et $\tilde{S_{1}} <:
      S_{1}$, nous concluons avec $T_{1} <: S_{1}$. Pour finir, en utilisant (T-SUB)
      avec $\Gamma, x : S_{1} \vdash s_{2} : \tilde{S_{2}}$ et $\tilde{S_{2}} <:
      T_{2}$, nous obtenons $\Gamma, x : S_{1} \vdash s_{2} : T_{2}$.
      \item (T-ABS). Le résultat est direct et $T_{1} = S_{1}$.
  \end{enumerate}
\end{proof}

Nous pouvons maintenant démontrer le même lemme de substitution défini dans le
chapitre précédent. La fonction de substitution est étendue aux projections et
aux enregistrements de manière naturelle.

\begin{lemma} [de préservation de typage par substitution]
  \label{lemma:subtyping-record-substitution}
  Soit $\Gamma, x : S \vdash t : T$ et $\Gamma \vdash s : S$.

  Alors $\Gamma \vdash [x \rightarrow s] t : T$
\end{lemma}

\begin{proof}
  Même technique de preuve que dans le chapitre précédent. Il suffit d'ajouter 
  des cas pour les enregistrements et les projections en utilisant
  respectivement (T-RCD) et (T-PROJ) et de manière générale pour (T-SUB). Pour
  (T-SUB), nous utilisons l'hypothèse de récurrence sur le jugement de gauche et
(T-SUB) pour conclure.
\end{proof}

\begin{theorem} [de préservation du typage]
  Soit $\Gamma \vdash t : T$ et soit $t'$ tel que $t \rightarrow t'$. Alors,
  $\Gamma \vdash t' : T$.
\end{theorem}

\begin{proof}
  Nous procédons de la même manière que pour le cas du $\lambda$-calcul
  simplement typé, c'est-à-dire sur le jugement $\Gamma \vdash t : T$.
  \begin{enumerate}
    \item (T-VAR). Nous avons alors $t = x$ : pas possible car il n'y a pas de
      réduction pour les variables.
      \item (T-ABS). $t = \lambdaExpr{x : T_{1}}{t_{12}}$ : déjà une valeur.
      \item (T-APP). $t = t_{1} \, t_{2}$. Nous avons
          $\Gamma \vdash t_{1} : T_{1} \rightarrow T$, $\Gamma \vdash t_{2}
          : T_{1}$.
          Les possibles règles d'évaluation
          sont (E-APP1), (E-APP2) et (E-APPABS). Pour (E-APP1) et (E-APP2), même
          raisonnement que pour le $\lambda$-calcul simplement typé.

          Quant à (E-APPABS), posons $t_{1} = \lambdaExpr{x : S_{1}}{t_{12}}$
          et $t_{2} = v$. En utilisant
          \ref{lemma:subtyping-record-substitution}, nous déduisons $T_{1} <:
          S_{1}$ et $\Gamma, x : S \vdash t_{12} : T$. Nous concluons avec le lemme de
          substitution qui donne $\Gamma, x : S \vdash [x \rightarrow v]t_{12} :
          T$.

       \item (T-RCD). La seule règle d'évaluation est (E-RCD) qui réduit un des
         termes de l'enregistrement. Il suffit d'appliquer l'hypothèse de
         récurrence sur ce terme et d'utiliser (T-RCD) pour conclure.

       \item (T-PROJ). Nous obtenons $t = t_{1}.l_{j}$, $\Gamma \vdash t_{1}
         : \left\{ l_{i} : T_{i} \right\}^{1 \leq i \leq n}$ et $T = T_{j}$.
         Deux règles d'évaluation sont possibles : (E-PROJ) ou (E-PROJ-RCD). Si
         c'est (E-PROJ), nous appliquons l'hypothèse de récurrence. Sinon,
         (E-PROJ-RCD) nous dit que $t_{1} = \left\{ k_{i} = v_{i} \right\}^{1
           \leq i \leq m}$ et par le lemme
         \ref{lemma:record-subtyping-inversion-subtyping-rules}, nous avons
         $(l_{i})_{1 \leq i \leq n} \subseteq (k_{i})_{1 \leq i \leq m}$ et
         $\Gamma \vdash v_{i} : T_{i}$ pour $k_{i} = l_{i}$. Nous concluons
         que $\Gamma \vdash v_{j} : T_{j}$.

       \item (T-SUB). Par hypothèse de récurrence et en utilisant (T-SUB).
  \end{enumerate}
\end{proof}

\subsection*{Progression}

\begin{lemma} [des formes canoniques]
  Soit $v$ une valeur sans variable libre.
  \begin{enumerate}
  \item Si $v$ est de type $T_{1} \rightarrow T_{2}$, alors $v$ est de la forme $\lambdaExpr{x : S_{1}}{t}$.
    \item Si $v$ est de type $\left\{ l_{i} : T_{i} \right\}^{1 \leq i \leq n}$,
      alors $v$ est de la forme $\left\{ k_{i} = v_{i} \right\}^{1 \leq i \leq
        m}$ où $(l_{i})_{1 \leq i \leq n} \subseteq
      (k_{i})_{1 \leq i \leq m}$.
  \end{enumerate}
\end{lemma}

\begin{proof}
  Une valeur ne peut être que de deux formes : $\lambdaExpr{x : S_{1}}t_{1}$ ou
  $\left\{ k_{i} = v_{i} \right\}^{1 \leq i \leq m}$.
  \begin{itemize}
  \item $v$ est de type $T_{1} \rightarrow T_{2}$. Les seules règles permettant
    de montrer $\vdash v : T_{1} \rightarrow T_{2}$ sont (T-SUB) et (T-ARROW).

    \begin{itemize}
    \item (T-ARROW). Le résultat est direct.
    \item (T-SUB). Nous avons alors :

      \begin{mathpar}
        \inferrule
        {\vdash v : S \\ S <: T_{1} \rightarrow T_{2}}
        {\vdash v : T_{1} \rightarrow T_{2}}
      \end{mathpar}
      En utilisant le lemme d'inversion des règles de sous-typage, nous obtenons
      $S = S_{1} \rightarrow S_{2}$ avec $T_{1} <: S_{1}$ et $S_{2} <: T_{2}$.
      
      En utilisant l'hypothèse de récurrence sur $\vdash v : S_{1} \rightarrow
      S_{2}$, nous avons $v$ qui est une abstraction.
    \end{itemize}
  \item $v$ est de type $\left\{ l_{i} : T_{i} \right\}^{1 \leq i \leq n}$. Les
    seules règles possibles sont (T-SUB) et (T-RCD). Le résultat est direct pour
    (T-RCD). Quant à (T-SUB), nous utilisons les mêmes arguments que 
    précédemment.
  \end{itemize}
\end{proof}

Et nous concluons avec le théorème de progression.

\begin{theorem} [de progression] 
  Si $t$ est un terme bien typé sans variable libre, alors soit $t$ est une
  valeur, soit il existe $t'$ tel que $t \rightarrow t'$.
\end{theorem}

\begin{proof}
  Par induction sur l'arbre de dérivation de typage de $t$.
  $t$ ne peut pas être une variable car $t$ est clos et si $t = \lambdaExpr{x :
    S_{1}}{t_{1}}$, le résultat est direct car les abstractions sont des valeurs.
  \begin{itemize}
  \item (T-APP). Alors $t = t_{1}t_{2}$, $\vdash : t_{1} : T_{1} \rightarrow
    t_{2}$ et $\vdash t_{2} : T_{1}$. Plusieurs cas sont possibles.
    \begin{itemize}
    \item Si $t_{1}$ n'est pas une valeur, (E-APP1) peut s'appliquer.
    \item Si $t_{1}$ est une valeur et $t_{2}$ non, (E-APP2) s'applique.
    \item Si $t_{1}$ et $t_{2}$ sont des valeurs, par le lemme des formes
      canoniques, comme $\vdash t_{1} : T_{1} \rightarrow T_{2}$ et $t_{1}$ n'a
      pas de variable libre, $t_{1} = \lambdaExpr{x : S_{1}}{t_{12}}$ et (E-APPABS) s'applique.
    \end{itemize}
  \item (T-RCD). Alors $t = \left\{ l_{i} = t_{i} \right\}^{1 \leq i \leq n}$ et
    $\vdash t : \left\{ l_{i} : T_{i} \right\}^{1 \leq i \leq n}$. Plusieurs cas
    sont possibles :
    \begin{itemize}
    \item si chaque $t_{i}$ est une valeur, alors le résultat est direct car $t$
      est une valeur.
    \item s'il existe $i$ tel que $t_{i}$ n'est pas une valeur, alors nous
      utilisons l'hypothèse de récurrence sur $t_{i}$ et (E-RCD) s'applique.
    \end{itemize}
  \item (T-PROJ). Nous avons alors
    $t = t_{1}.l_{j}$ et $\vdash t_{1} : T$ où $T = \left\{ l_{i} : T_{i}
    \right\}^{1 \leq i \leq n}$.

  Par hypothèse de récurrence sur $\vdash t_{1} : T$, $t_{1}$ est soit une
  valeur, soit il existe $t_{1}'$ tel que $t_{1} \rightarrow t_{1}'$. Si $t_{1}$
  est une valeur, nous utilisons le lemme
  des formes canoniques
  et (E-PROJ-RCD) s'applique. Sinon, (E-PROJ) s'applique.
  \item (T-SUB). Il suffit d'appliquer l'hypothèse de récurrence.
  \end{itemize}
\end{proof}


\section{Type Top et type Bottom}

Finalement, il
est courant d'ajouter dans les types un type qui est super-type de tous les
autres types, souvent appelé \verb|Top| ainsi qu'un type qui est sous-type de
tous les autres types, souvent appelé \verb|Bottom|.

La syntaxe des types est donc finalement

\begin{align*}
  T ::= & \, & \text{type} \\
        & \; b & \text{type de base} \\
        & \; t \rightarrow t & \text{fonction} \\
        & \; \left\{ l_{i} : t_{i} \right\}^{1 \leq i \leq n} & \text{enreg} \\
        & \; Top & \text{Top} \\
        & \; Bottom & \text{Bottom} \\
\end{align*}

et nous ajoutons les règles de typage suivantes.

\begin{mathpar}
  \inferrule* [Left=(S-TOP)] {T <: Top}{}
  \and
  \inferrule* [Right=(S-BOTTOM)] {Bottom <: T}{}
\end{mathpar}

Nous utiliserons ceux-ci dans DOT.