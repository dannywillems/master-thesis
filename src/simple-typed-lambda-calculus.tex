\chapter{$\lambda$-calcul simplement typé.}

Dans le chapitre 1, nous avons défini la syntaxe et la sémantique d'un calcul
appelé le $\lambda$-calcul non typé. Nous allons maintenant ajouter une notion de types à
chaque terme de notre calcul.

Motivation du typage.

\begin{definition} [Relation de typage]
  Soit $\tau$ un ensemble, appelé \textbf{ensemble des types}, dont les éléments
  sont notés $T_{i}$.
  Soit $\Lambda$ l'ensemble des $\lambda$ termes.

  On définit une relation binaire $R$ entre les
  $\lambda$ termes et les éléments de $\tau$.
  
  On dit que \textbf{le terme $t \in \Lambda$ a le type $T \in \tau$} si $(t, T)
  \in R$ (noté plus souvent $t : T$).
\end{definition}

Le typage est donc un moyen de spécifier l'appartenance de certains
$\lambda$-termes à un ensemble précis de type et ainsi réduire les opérations
possibles sur ces $\lambda$-termes.

Dans le chapitre 1 sur le $\lambda$-calcul non typé, nous avons défini une
relation d'évaluation, noté $\rightarrow$. Nous pouvons nous demander comment la
relation de typage est compatible avec la relation $\rightarrow$.

Dans une règle de typage, il se peut que certains termes possèdent des variables
libres (comme $\lambda x x y$). Lorsque nous $\beta$-réduisons un terme, il nous
faut connaître chaque type de chaque variable libre (dans notre cas, $y$). Nous
introduisons pour cela \textbf{le contexte de typage} qui fondamentalement est
une suite finie de couple $(x_{i}, T_{i})$ où $x_{i}$ est une variable et
$T_{i}$ est un type.

\begin{definition} [Contexte de typage]
  Un \textbf{contexte de typage}, noté $\Gamma$, est un ensemble fini de couple $(x_{i},
  T_{i})$ où $x_{i}$ est une variable et $T_{i}$ est un type.
  
  L'union d'un contexte de typage $\Gamma$ avec le couple $(x, T)$ est noté
  $\Gamma, x : T$ (à la place de $\Gamma \cup \left\{(x, T)\right\}$).
\end{definition}

\begin{definition} [Règle de typage / jugement de typage]
  Voir la définition récursive de wikipedia :
  \url{https://fr.wikipedia.org/wiki/Lambda-calcul#cite_ref-8}
  Cela permet de voir un jugement de typage comme un triplet $(\Gamma, t, T)$,
  noté $\Gamma \vdash t : T$ qu'on lit \textbf{$t$ est de type $T$ dans le
    contexte $\Gamma$}.
  On définit récursivement un jugement de typage
  \begin{enumerate}
  \item $(x : T) \in \Gamma \implies \Gamma \vdash t : T$
  \item $\Gamma, x : T \vdash t : T_{1} \implies \Gamma, x : T \vdash \lambda (x
    : T) t : T \rightarrow T_{1}$
  \item $\Gamma \vdash u : T_{1} \rightarrow T_{2} \wedge \Gamma \vdash u :
    T_{1} \implies \Gamma \vdash uv : T_{2}$
  \end{enumerate}

   Si $(t, T) \in \Gamma$, on dit alors que $t$ est bien typé dans $\Gamma$.
\end{definition}

Le contexte est utilisé pour faire des hypothèses sur chaque variable libre dans
le $\lambda$-terme $t$. Donc, on ne peut avoir un jugement de typage de type:

$\empty \vdash (\lambda (x : T_{1}) x) y$ car toutes les variables libres (en
l'occurrence ici $y$) du $\lambda$-terme ne sont pas typées dans le contexte
$\empty$ : nous ne connaissons pas le type de $y$.

Comme pour les règles d'évaluation, on écrit la définition d'un jugement de typage comme des
règles d'inférence. La définition d'un jugement de typage devient donc:

\inferrule
{(x : T) \in \Gamma}
{\Gamma \vdash x : T}
\quad (\textsc{T-VAR})

\inferrule{\Gamma, x : T \vdash t : T_{1}}{\Gamma, x : T \vdash
  \lambda (x : T) t : T_{1}}
\quad (\textsc{T-ABS})

\inferrule{\Gamma \vdash u : T_{1} \rightarrow T_{2} \\ \Gamma \vdash v : T_{2}}{\Gamma
  \vdash uv : T_{2}}
\quad (\textsc{T-APP})

Le $\lambda$-calcul simplement typé est donc un tuple $(\Lambda, \rightarrow,
\tau, \Gamma_{\tau})$ où
\begin{enumerate}
\item $\Lambda$ est l'ensemble des $\lambda$-termes.
\item $\rightarrow$ est la relation de $\beta$-réduction.
\item $\tau$ l'ensemble des types.
\item $\Gamma$ est le jugement de typage (défini récursivement).
\end{enumerate}

\section{Sûreté}

Expliquer la préservation et la progression.
Regarder dans le cours de l'ENS à la place ??

Avant de montrer la préservation et la progression, il est nécessaire de
remarquer certains faits qui découlent immédiatement des règles de typages.

\subsection*{Progression}

\begin{lemma} [Inversion des règles de typage]
  \begin{enumerate}
    \item Si $\Gamma \vdash x : T$, alors $(x : T) \in \Gamma$
    \item 
    \item 
    \item 
    \item 
    \item 
  \end{enumerate}
\end{lemma}
  
\begin{proof}
  Evident d'après les règles de typages données.
\end{proof}

Une autre remarque importante sur le calcul $\lambda_{\rightarrow}$ est
l'unicité de type pour les $\lambda$. Cette proposition est tellement
fondamentale que le terme théorème est utilisé. Cependant, cette propriété n'est
pas vraie dans tous les calculs, comme nous le montrerons quand nous
introduirons le sous-typage.

\begin{theorem}
  Tout $\lambda$-terme bien typé possède un type unique.
\end{theorem}

\begin{proof}
  
\end{proof}

\begin{theorem} [de progression de $\lambda_{\rightarrow}$]
  Soit $t$ un terme bien typé sans variable libre. Alors, soit $t$ est une
  valeur, soit il existe $t'$ tel que $t \rightarrow t'$.
\end{theorem}

\begin{proof}
  
\end{proof}

\subsection*{Préservation}

\begin{lemma} [d'affaiblissement]
  Soit $\Gamma \vdash t : T$ et $x \notin dom(\Gamma)$.

  Alors $\Gamma, x : S \vdash t : T$.
\end{lemma}

\begin{proof}
  
\end{proof}

\begin{lemma} [de préservation du typage pour la substitution]
  Soit $\Gamma, x : S \vdash t : T$ et $\Gamma \vdash s : S$.

  Alors $\Gamma \vdash [x \rightarrow s] t : T$
\end{lemma}

\begin{proof}

\end{proof}

\section*{Normalisation}

Peut être ne pas donner des preuves, mais en parler pour dire que c'est très important.

On se pose des questions sur la finitude des évaluations.

Regarder du coté du cours de l'ENS Lyon, chap 3 pour le lambda-calcul simplement typé.

\begin{definition}
  On dit qu'un $\lambda$-terme est
  \begin{itemize}
  \item \textbf{fortement normalisable} si
  toute chaine de $\beta$-réduction est finie.
  \item \textbf{faiblement normalisable} s'il existe une chaine de
    $\beta$-réduction finie.
  \end{itemize}
\end{definition}

\section{Enrichir le calcul avec des types de bases}

Remarquer qu'on peut ajouter d'autres types comme les listes, les records,
etc avec des règles de typages et des règles d'évaluations propres sans que
cela ne change la propriété de soundness.

Nous utiliserons dans la suite la syntaxe
let x = t in u qui est un alias pour une forme de lambda (la donner).
Cette syntaxe nous permet de n'étudier que les applications entre variables.
En effet, si nous avons une expression de la forme (t u), nous pouvons réduire
l'expression sous la forme (x y) avec
let x = t in
let y = u in
x y

Cependant, il faut vérifier que la sémantique reste la même. Je ne pense pas...