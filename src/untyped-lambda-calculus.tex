\chapter{$\lambda$-calcul non typé}

\section{Syntaxe}

\begin{definition} [Syntaxe du $\lambda$-calcul]
  Soit $V$ un ensemble infini dénombrable. On note $\Lambda$, appelés \textbf{l'ensemble des $\lambda$-termes}, le plus petit
  ensemble tel que:
  \begin{enumerate}
    \item $V \subseteq \Lambda$
    \item $\forall u, v \in \Lambda, uv \in \Lambda$
    \item $\forall x \in V, \forall u \in \Lambda, \lambda x.u \in \Lambda$
  \end{enumerate}

%Remarquons que nous venons seulement de définir une syntaxe ! Les
%représentations suggérées pour les \lambda-termes uv et \lambda x.u viennent de la
%sémantique le plus souvent utilisée.
\end{definition}

Un élément de $\Lambda$ est appelé un \textbf{$\lambda$-terme}.
Un $\lambda$-terme de la forme $uv$ est appelé \textbf{application} car
l'interprétation donnée est une fonction $u$ évaluée en $v$.
Un $\lambda$-terme de la forme $\lambda$ $x.u$ est appelé
\textbf{abstraction}. Il est souvent representé comme la fonction qui envoie
$x$ sur $u$.

Des exemples de $\lambda$-termes sont
\begin{itemize}
  \item la fonction identité : $\lambda x . x$
  \item la fonction constante en $y$: $\lambda x . y$
  \item la fonction qui renvoie la fonction constante pour n'importe quelle
    variable: $\lambda y . (\lambda x . y)$.
  \item l'application identité appliquée à la fonction identité: $(\lambda x . x)
    (\lambda y . y)$
\end{itemize}

Comme dans une expression mathématique, il est important de différentier les
variables libres et les variables liées d'un $\lambda$-terme. Par exemple, dans
le $\lambda$-terme $\lambda x . x$ la variable $x$ est liée par un $\lambda$
tandis que dans l'expression $\lambda x . y$ la variable $y$ est libre.

\begin{definition} [Ensemble de variables libres]
  L'ensemble des variables \textbf{libres} d'un terme $t$, noté $FV(t)$ est défini
  récursivement sur les générateurs de $\Lambda$ par:
  \begin{itemize}
  \item[$\bullet$] $FV(x) = \GSset{x}$
  \item[$\bullet$] $FV(\lambda x . t) = FV(t) \backslash \GSset{x}$
  \item[$\bullet$] $FV(u v) = FV(u) \union FV(v)$
  \end{itemize}
\end{definition}

\begin{definition} [Ensemble de variables libres]
  L'ensemble des variables \textbf{liées} d'un terme $t$, noté $BV(t)$ est défini
  récursivement sur les générateurs de $\Lambda$ par:
  \begin{itemize}
  \item[$\bullet$] $BV(x) = \emptyset$
  \item[$\bullet$] $BV(\lambda x . t) = BV(t) \union \GSset{x}$
  \item[$\bullet$] $BV(u v) = BV(v) \backslash BV(u)$
  \end{itemize}
\end{definition}

Cela nous amène à la définition suivante:
\begin{definition} [Relation d'$\alpha$-renommage]
  
\end{definition}

Les éléments $\lambda x \lambda y xy$ et $\lambda y \lambda x yx$ appartiennent
à la même classe d'équivalence, ce que nous souhaitons.

On se concentre maintenant uniquement sur les classes d'équivalence.

\section{Sémantique}

A toute syntaxe, nous associons une \textit{sémantique}, c'est-à-dire une interprétation
des termes.

Pour le $\lambda$-calcul non typé, la sémantique que nous allons définir permet
de réduire un $\lambda$-terme vers un autre $\lambda$-terme. Nous parlons de
\textit{$\beta$-réduction}, ou encore de \textit{réécriture}. La
$\beta$-réduction peut se voir comme une relation binaire sur $\Lambda$.

Sémantique opérationnelle.

\subsection{Encodage}

- Booléens: \verb|true| = $\lambda x \lambda y x$, \verb|false| = $\lambda x
\lambda y y$.
- Conditions: $\lambda condition \lambda if\_true \lambda if\_false condition
if\_true if\_false$
- paires.

\subsection{Différentes stratégies de réduction}

Parler des différentes stratégies d'évaluations.

\subsubsection*{Call by value}
\subsubsection*{Call by name}

Parler des méthodes de preuves sur le $\lambda$-calcul : preuves termes par
termes, par la taille du terme.

\subsection*{Normalisation}

Peut être ne pas donner des preuves, mais en parler pour dire que c'est très important.

On se pose des questions sur la finitude des évaluations.

Regarder du coté du cours de l'ENS Lyon, chap 3 pour le lambda-calcul simplement typé.

\begin{definition}
  On dit qu'un $\lambda$-terme est
  \begin{itemize}
  \item \textbf{fortement normalisable} si
  toute chaine de $\beta$-réduction est finie.
  \item \textbf{faiblement normalisable} s'il existe une chaine de
    $\beta$-réduction finie.
  \end{itemize}
\end{definition}

