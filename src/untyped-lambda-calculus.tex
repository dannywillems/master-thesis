\chapter{$\lambda$-calcul non typé}

Dans ce chapitre, nous allons introduire les bases théoriques de la
programmation fonctionnelle en parlant du $\lambda$-calcul non typé.
Nous discutons de la syntaxe de ce langage (les termes) pour ensuite discuter de
la réduction de ceux-ci à travers la $\beta$-réduction.


\section{Syntaxe}

\begin{definition} [Syntaxe du $\lambda$-calcul]
  Soit $V$ un ensemble infini dénombrable dont les éléments sont appelés \textbf{variables}. On note $\Lambda$, appelé \textbf{l'ensemble des $\lambda$-termes}, le plus petit
  ensemble tel que:
  \begin{enumerate}
    \item $V \subseteq \Lambda$
    \item $\forall u, v \in \Lambda, uv \in \Lambda$
    \item $\forall x \in V, \forall u \in \Lambda, \lambdaExpr{x}{u} \in \Lambda$
  \end{enumerate}
\end{definition}

Un élément de $\Lambda$ est appelé un \textbf{$\lambda$-terme}, ou tout
simplement un \textbf{terme}.
Un $\lambda$-terme de la forme $uv$ est appelé \textbf{application} car
l'interprétation donnée est une fonction $u$ évaluée en $v$.
Un $\lambda$-terme de la forme $\lambdaExpr{x}{u}$ est appelé
\textbf{abstraction}, le terme $u$ étant appelé le \textbf{corps}, et est interprété comme la fonction qui envoie
$x$ sur $u$.

La plupart des ensembles que nous définirons seront définis de manière récursive
comme ci-dessus.
Pour des raisons de facilité d'écriture, la syntaxe
\begin{center}
  \begin{math}
    \Lambda ::= V \, | \, \Lambda \Lambda \, | \, V \Lambda
  \end{math}
\end{center}

ou encore
\begin{align*}
  t ::= & \, & \text{terme} \\
        & \; x & \text{var} \\
        & \; t \, t & \text{app} \\
        & \; \lambdaExpr{x}{t} & \text{abs}
\end{align*}

où $x$ parcourt l'ensemble des variables $V$ et $t$ l'ensemble des termes, sont
utilisées pour définir ces ensembles. La dernière syntaxe sera celle que nous
utiliserons tout le long de ce document car elle permet une visualisation simple
de la syntaxe des termes et permet de nommer chaque forme facilement.

Des exemples de $\lambda$-termes sont
\begin{itemize}
  \item la fonction identité : $\lambdaExpr{x}{x}$
  \item la fonction constante en $y$: $\lambdaExpr{x}{y}$
  \item la fonction qui renvoie la fonction constante pour n'importe quelle
    variable: $\lambdaExpr{y}{\lambdaExpr{x}{y}}$.
  \item l'application identité appliquée à la fonction identité:
    $(\lambdaExpr{x}{x}) (\lambdaExpr{y}{y})$
\end{itemize}

Comme le montrent le dernier exemple, des parenthèses sont utilisées pour
délimiter les termes.

Il est également possible de définir des fonctions à plusieurs paramètres à
travers la curryfication : une fonction prenant 2 paramètres sera représentée
par une fonction qui renvoie une fonction. Par exemple,
$\lambdaExpr{x}{\lambdaExpr{y}{x y}}$ est une fonction qui attend un paramètre
$x$ retournant une fonction qui attend un paramètre $y$,
mais elle peut aussi être interprétée comme une fonction à deux paramètres $x$, $y$.
%Le terme $\lambda x . \lambda y . xy$ $\lambda x . \lambda y . (x y)$ et $(\lambda x
%. \lambda y . x) y$ sont différents selon la positionnement des parenthèses.

Comme dans une formule mathématique, il est important de différencier les
variables libres et les variables liées d'un $\lambda$-terme. Par exemple, dans
le $\lambda$-terme $\lambdaExpr{x}{x}$ la variable $x$ est liée par un
$\lambda$\footnote{on dit aussi qu'elle est \og sous \fg \, un $\lambda$.}
tandis que dans l'expression $\lambdaExpr{x}{y}$ la variable $y$ est libre.
Nous définissons récursivement l'ensemble des variables libres et l'ensemble des
variables liées à partir des variables, des abstractions et des applications.

\begin{definition} [Ensemble de variables libres]
  L'ensemble des variables \textbf{libres} d'un terme $t$, noté $FV(t)$ est défini
  récursivement sur la structure des termes de $\Lambda$ par:
  \begin{itemize}
  \item[$\bullet$] $FV(x) = \GSset{x}$
  \item[$\bullet$] $FV(\lambdaExpr{x}{t}) = FV(t) \backslash \GSset{x}$
  \item[$\bullet$] $FV(u v) = FV(u) \union FV(v)$
  \end{itemize}
\end{definition}

\begin{definition} [Ensemble de variables liées]
  L'ensemble des variables \textbf{liées} d'un terme $t$, noté $BV(t)$ est défini
  récursivement sur la structure des termes de $\Lambda$ par:
  \begin{itemize}
  \item[$\bullet$] $BV(x) = \emptyset$
  \item[$\bullet$] $BV(\lambdaExpr{x}{t}) = BV(t) \union \GSset{x}$
  \item[$\bullet$] $BV(u v) = BV(u) \union BV(v)$
  \end{itemize}
\end{definition}

Un terme qui ne comporte pas de variable libre est dit \textit{clos}.

Il existe également des termes qui sont syntaxiquement différents, mais que nous
voudrions naturellement qu'ils soient les mêmes. Par exemple, nous voudrions que
la fonction identité $\lambdaExpr{x}{x}$ ne dépende pas de la variable liée $x$,
c'est-à-dire que les termes $\lambdaExpr{x}{x}$ et $\lambdaExpr{y}{y}$ soient un seul et
unique terme: la fonction identité. Cette égalité se résume à une substitution
de la variable $x$ par la variable $y$, ou plus généralement par un terme $u$.

Avant de donner une définition exacte, il est important de remarquer que la
substitution n'est pas une action triviale si nous ne voulons pas changer le
sens des termes. Si nous effectuons une
substitution purement syntaxique, nous pouvons alors obtenir des termes qui ne sont
plus dans la syntaxe des éléments de $\Lambda$. Par exemple, si nous
substituons toutes les occurences de $x$ par un terme $u$ dans la fonction constante
$\lambda x . y$, nous aurions $\lambda u . y$, qui n'a pas de sens car $u$ n'est
pas obligatoirement une variable.

La définition doit aussi prendre en compte les notions de variables liées et libres. En effet, si nous
prenons la fonction constante $\lambda x . y$ et que nous substituons $y$ par 
$x$ uniquement dans le corps de la fonction, nous obtenons $\lambda x . x$, qui
n'a pas le même sens que $\lambda x . y$. Cet exemple nous montre que nous
devons faire attention lorsque la variable à substituer, dans ce cas $x$, est
liée dans le terme où se passe la substitution (ici $\lambda x . y$).

Un autre exemple où la substitution n'est pas évidente est la substition de la
variable $z$ du terme $\lambda x . z$ (la fonction constante en $z$) par le terme
$\lambda y . x$ (la fonction constante en $x$). Après subtitution, nous nous
retrouvons avec le terme $\lambda x . \lambda y . x$, c'est-à-dire la fonction
qui renvoie la fonction constante pour le paramètre donné. Ce dernier exemple
nous montre que nous devons également faire attention aux variables libres du
terme substituant.

\begin{definition} [Substitution de variable par un terme]
  Soit $x \in V$ et soient $u, v \in \Lambda$. On dit que la variable $x$ est
  \textbf{substituable par $v$ dans $u$} si et seulement si $x \notin BV(u)$ et
  $FV(v) \inter BV(u) = \emptyset$.
\end{definition}

Nous définissons alors la fonction de substitution d'une variable $x$ par un
terme $v$ dans un terme $u$.

\begin{definition} [fonction de subtitution]
  Soient $x$ une variable et $u, v \in \Lambda$ tel que $x$ est substituable par
  $v$ dans $u$. On définit récursivement la fonction de substitution, notée $[x := v]u$, par:
  \begin{itemize}
  \item[$\bullet$] $[x := v]x = v$
  \item[$\bullet$] $[x := v]y = y$ (si $y \neq x$)
  \item[$\bullet$] $[x := v](u_{1} u_{2}) = ([x := v]u_{1})([x := v]u_{2})$
  \item[$\bullet$] $[x := v](\lambda y . u) = \lambda y . ([x := v]u)$
  \end{itemize}
  $u[x := v]$ se lit \textit{x est substitué par v dans u}.
\end{definition}

Nous définissons maintenant une relation sur les abstractions, appelée relation d'
\textbf{$\alpha$}-renommage, qui capture
notre volonté d'égalité à renommage de variables près.

\begin{definition} [relation d'$\alpha$-renommage]
  Soient $x, y \in V$ et $u \in \Lambda$.
  La relation d'$\alpha$-renommage, notée $\alpha$, est définie par

  \begin{math}
    \lambda x . u \;\; \alpha \;\; \lambda y . (u[x := y])
  \end{math}

  si $x = y$ ou si $x$ est substituable par $y$ dans $u$ et $y$ n'est pas libre
  dans $u$.
\end{definition}

Nous allons étendre cette relation à tous les termes, c'est-à-dire sur tout l'ensemble $\Lambda$.

Nous notons $=_{\alpha}$ la plus petite relation comprenant $\alpha$ et tel que
\begin{itemize}
  \item[$\bullet$] $=_{\alpha}$ est réflexive, symétrique et transitive
  \item[$\bullet$] $=_{\alpha}$ passe au contexte: si $u_{1} =_{\alpha} v_{1}$
    et $u_{2} =_{\alpha} v_{2}$ alors $u_{1}u_{2} =_{\alpha} v_{1}v_{2}$ et
    $\lambda x . u_{1} =_{\alpha} \lambda x . v_{1}$.
\end{itemize}

\begin{exemple}
  \begin{enumerate}
    \item Il est clair que $\lambda x . x =_{\alpha} \lambda y . y$ par
    définition de la relation $\alpha$.
    \item De même, $\lambda x . \lambda y . x y =_{\alpha} \lambda y . \lambda x
      . y x$. En
      effet, on montre que $\lambda x . \lambda y . x y =_{\alpha} \lambda z .
      \lambda w. z
      w$ et $\lambda y . \lambda x . y x =_{\alpha} \lambda z . \lambda w . z w$ en appliquant
      deux fois la subtitution (par $z$ et par $w$). Par symétrie
      et transitivité de $=_{\alpha}$, on obtient l'égalité.
  \end{enumerate}
\end{exemple}

Par définition, la relation $=_{\alpha}$ est une relation d'équivalence. Nous
construisons alors le quotient $\Lambda \backslash =_{\alpha}$. Dans ce
quotient, les termes égaux à renommage de variables près se retrouvent dans la
même classe d'équivalence. A partir de maintenant, nous considérons $\Lambda
\backslash =_{\alpha}$, c'est-à-dire que nous parlons des termes à $\alpha$-renommage près.

\section{Sémantique}

Maintenant que nous avons introduit la syntaxe du $\lambda$-calcul, nous allons
discuter de la sémantique que nous lui associons, c'est-à-dire comment nous
effectuons des calculs avec ce langage. Les calculs
se définissent par des \textit{réductions}\footnote{On parle aussi de
\textit{réécriture}.} de termes, et en particulier des applications. Par exemple,
nous voudrions dire que $(\lambda x . x) y$, i.e. $y$ appliqué à la fonction
identité, se \textit{réduit} en $y$ ou encore que $(\lambda x . (\lambda y . x))
z$, i.e. $z$ appliqué à la fonction qui retourne la fonction constante pour
toute variable, se réduit en $\lambda y . z$, i.e. la fonction
constante en $z$. Nous parlons également \textit{d'étape de calcul}, une étape
de calcul correspondant à une réduction effectuée.

La définition de réduction des termes passe par une relation entre les
termes appelée relation de $\beta$-réduction.
Comme pour la relation $\alpha$, nous commençons par définir une relation
$\beta$, et nous l'étendons au contexte.

\begin{definition} [Relation de $\beta$-réduction]
  Soit $\beta$ la relation sur $\Lambda$ tel que $(\lambda x . u) v \; \; \beta 
  \; \; [x := v]u$.\footnote{Ne pas oublier que nous travaillons à
    $\alpha$-renommage près.}
  La relation de $\beta$-réduction, noté $\rightarrow_{\beta}$, ou simplement
  $\rightarrow$, est la plus petite relation contenant $\beta$ qui passe au
  contexte. Nous notons $\rightarrow^{*}_{\beta}$ sa fermeture réfléxive
  transitive et $\rightarrow^{+}_{\beta}$ sa fermeture transitive.
\end{definition}

Voici quelques exemples de réductions:
\begin{exemple}
  \begin{enumerate}
  \item $(\lambda x . x) y \rightarrow y$.
  \item $(\lambda y . (\lambda x . y x)) z \rightarrow \lambda x . z x$.
  \item $(\lambda w . (\lambda x . x) v) z \rightarrow (\lambda w. v) z$ (on
    réduit à l'intérieur, c'est-à-dire $(\lambda x . x) v$).
  \item $(\lambda w . (\lambda x . x) v) z \rightarrow (\lambda x. x) v$ (on
    réduit à l'extérieur, c'est-à-dire $(\lambda w . t) z$ où $t = (\lambda x .
    x) v$).
  \item $(\lambda w . (\lambda x . x) v) z \rightarrow^{*}_{\beta} v$
  \item $(\lambda x . xx) (\lambda x . xx) \rightarrow (\lambda x . xx)
    (\lambda x . xx)$.
  \end{enumerate}
\end{exemple}

Un élément de la forme $(\lambda x . u) v$ est appelé \textit{redex}. En
analysant les termes que $\beta$ met en relation, la
$\beta$-réduction consiste donc à réécrire les redex.

Nous définissons aussi les \textit{valeurs} qui sont les termes finaux
possibles d'une $\beta$-réduction. Dans le cas du $\lambda$-calcul, les valeurs
sont les abstractions.

Un terme $t$ qui peut être réduit, c'est-à-dire qu'il existe $u$ tel que $t
\rightarrow u$, est dit \textit{réductible}. Sinon, il est dit
\textit{irréductible} ou on dit également que c'est une \textit{forme normale}.
S'il est possible de trouver une forme normale $u$ tel que $t$ se réduit en $u$,
on dit que $t$ \textit{possède une forme normale} et que $u$ est une
\textit{forme normale de $t$}.

Certains termes peuvent être réduits en des formes normales comme
dans les deux premiers exemples. Dans le premier exemple, le terme
irréductible n'est pas une valeur tandis que dans le second, nous obtenons
une valeur.

Le troisième et quatrième exemples montrent qu'il existe plusieurs manières,
appelées aussi \textit{stratégie de réduction}, de
réduire un terme.

Lorsque toute réduction commençant par $t$ possède une forme normale, on dit que
le terme $t$ est \textit{fortement normalisant}, ou tout simplement
\textit{normalisant}. Lorsqu'il existe au moins une stratégie de réduction qui
permet d'obtenir un terme irréductible, on dit que le terme est \textit{faiblement normalisant}.

Le dernier exemple montre qu'il existe des termes dont aucune réduction se
termine. Celui-ci nous montre que la $\beta$-réduction ne se termine pas toujours.
Ce fait n'est pas si étrange que ça : dans la plupart des langages de
programmation, il est possible d'écrire des programmes qui bouclent à l'infini,
c'est-à-dire que la réduction ne se termine pas.

%La relation $\rightarrow_{\beta}$ signifie que nous faisons une étape de calcul,
%contrairement à la fermeture transitive qui en réalise plusieurs. Choisir l'une
%ou l'autre définit \textit{la stratégie d'évaluation} que nous assignons à notre
%calcul. $\rightarrow_{\beta}$ définit \textit{l'évaluation à petit pas} car
%nous obtenons chaque étape de l'évaluation. La fermeture transitive définit
%\textit{l'évaluation à grand pas} qui consiste à ne regarder que la forme
%normale obtenue, si elle existe.

%Dans la suite, nous considérerons toujours que nous utilisons la stratégie
%d'évaluation à petit pas.

\subsection{Stratégies de réduction}

Nous présentons les stratégies les plus utilisées. Le terme
\begin{math}
  id (id (\lambda z . id z))
\end{math}

où $id = \lambda x . x$ sera utilisé pour interpréter chaque stratégie de réduction.
Les redex de ce terme sont:
\begin{itemize}
\item $id (id (\lambda z . id \, z))$
\item $id (\lambda z . id \, z)$.
\item $id z$
\end{itemize}

Dans les exemples ci-dessous, le redex qui est réduit est souligné.

\subsubsection*{Ordre normale}

Cette méthode réduit d'abord les redex à l'extérieur, les plus à gauche.
La chaine de réduction de notre exemple est alors:

\begin{align*}
  & \underline{id \, (id \, (\lambda z . (id \, z)))} \rightarrow_{\beta}\\
  & \underline{id \, (\lambda z . (id \, z))} \rightarrow_{\beta} \\
  & \lambda z . \underline{(id \, z)} \rightarrow_{\beta} \\
  & \lambda z . z
\end{align*}

\subsubsection*{Call by name}

La stratégie appelée \textit{call-by-name} consiste à réduire les redex les plus
à gauche en premier, comme l'ordre normale. La différence est que le
call-by-name ne permet pas de réduire les redex qui sont dans le corps d'un lambda.

\begin{align*}
  & \underline{id \, (id \, (\lambda z . (id \, z)))} \rightarrow_{\beta}\\
  & \underline{id \, (\lambda z . (id \, z))} \rightarrow_{\beta} \\
  & \lambda z . (id \, z)
\end{align*}

La dernière étape de réduction de l'ordre normal n'est pas effectuée car celle-ci est
sous le lambda $\lambda z$.

\subsubsection*{Call by value}

La réduction dite \textit{call-by-value} consiste à réduire en premier les
redexes les plus à l'extérieur et réduire les arguments jusqu'à obtenir une valeur, et ensuite le corps de la fonction.
Cette méthode de réduction est la plus courante dans les langages de
programmation.

\begin{listing}
  \inputminted{OCaml}{codes/untyped-call-by-value-example.ml}
  \caption{Exemple qui montre que la stratégie de réduction utilisée par défaut
    dans OCaml est le call-by-value.}
\end{listing}

Cette stratégie appliquée à l'exemple donne:
\begin{align*}
  & id \underline{(id (\lambda z . (id \, z)))} \rightarrow_{\beta}\\
  & id \underline{(\lambda z . (id \, z))} \rightarrow_{\beta} \\
  & \lambda z . (id \, z)
\end{align*}

\label{fig:untyped-lambda-calculus-evaluation-rules}

Formellement, la stratégie call-by-value est définie par les \textit{règles
  d'évaluation} définies ci-dessous, le terme $v$ étant utilisé pour une valeur.

\begin{mathpar}
  \inferrule
    {t_{1} \rightarrow t_{1}'}
    {t_{1} \, t \rightarrow t_{1}' \, t} \quad (\textsc{E-APP1})
  \and
  \inferrule
    {t \rightarrow t'}
    {v \, t \rightarrow v \, t'} \quad (\textsc{E-APP2})
  \and
  \inferrule
    {(\lambdaExpr{x}{t}) v \rightarrow [x := v]t}
    {} \quad (\textsc{E-APPABS})
\end{mathpar}
\label{eval:untyped-lambda-calculus}

La notation \inferrule{t \rightarrow t'}{v \, t \rightarrow v \, t'} est
l'équivalent d'une implication où la prémisse (ici $t \rightarrow t'$) se trouve au dessus et la
conclusion en dessous (ici $v \, t \rightarrow v \, t'$). La règle E-APP1 se lit
donc \og si $t_{1}$ se réduit en $t_{1}'$, alors $t_{1} t$ se réduit en
$t_{1}'t$ \fg. Lorsque qu'une règle
ne comporte pas de conclusion comme E-APPABS, cela signifie que c'est un axiome.
Cette notation sera
utilisée tout au long de ce document, en particulier pour les règles de typage
et de sous-typage.

Les règles (E-APP1) et (E-APP2) nous disent que nous devons, lors d'une
application, réduire la fonction avant les paramètres, et ce jusqu'à obtenir une
valeur. Quant à la règle (E-APPABS), elle signifie qu'un redex se réduit toujours
en utilisant la fonction de substitution (définition de la relation $\beta$).

La relation de $\beta$-réduction pour la stratégie call-by-value est alors
définie comme le plus petit ensemble généré par les règles
\ref{eval:untyped-lambda-calculus}. Quand nous ajouterons à notre langage des
autres termes comme les enregistrements, nous mentionnerons uniquement les
règles d'évaluation, la relation de $\beta$-réduction étant implicitement
définie de la même manière.

Par la suite, nous considérerons toujours cette dernière stratégie car c'est la
plus utilisée.

La réécriture des termes est un large sujet, plus d'informations sur ce sujet
sont disponibles dans \cite{ENS-Cachan-cours-lambda-calcul}.
Dans ce cours sont traités les sujets de normalisation (forme normale, finitude
de la $\beta$-réduction), de confluence (est-ce que tout terme se réduit en une
unique forme normale) et d'une différente sémantique appelée \textit{sémantique
dénotationnelle}.

\section{Codage de termes usuels}

Le $\lambda$-calcul est assez riche pour définir des
termes usuels des langages de programmation comme les booléens (et en même temps les
conditions), les paires ou encore les entiers. Ces codages peuvent être trouvées
dans \cite{tapl-untyped-lambda-calculus}. Voici l'exemple des booléens, utilisé dans RML (\cite{rml-github}):

\begin{itemize}
  \item $true = \lambdaExpr{t}{\lambdaExpr{f}{t}}$
  \item $false = \lambdaExpr{t}{\lambdaExpr{f}{f}}$
  \item $test = \lambdaExpr{b}{\lambdaExpr{t'}{\lambdaExpr{f'}{b \, t' \, f'}}}$
\end{itemize}

Avec les définitions de $true$ et $false$, la fonction $test$ simule le fonctionnement d'une condition: si le premier
paramètre ($b$) est $true$, il renvoie $t'$, si c'est $false$, il renvoie
$f'$. En effet,

\begin{align*}
  & test \, true \, v \, w \rightarrow_{\beta} \\
  & (\lambdaExpr{b}{\lambdaExpr{t'}{\lambdaExpr{f'}{b \, t' \, f'}}}) \, true \, v \, w \rightarrow_{\beta} \\
  & (\lambdaExpr{t'}{\lambdaExpr{f'}{true \, t' \, f'}}) \, v \, w \rightarrow_{\beta} \\
  & (\lambdaExpr{f'}{true \, v \, f'}) \, w \rightarrow_{\beta} \\
  & true \, v \, w \rightarrow_{\beta} \\
  & v
\end{align*}

Un même raisonnement se fait pour $test \, false \, v \, w$, qui donne $w$.

Les fonctions $and$ et $or$ peuvent aussi être codées en
$\lambda$-calcul.

\begin{itemize}
  \item $and = \lambdaExpr{b}{\lambdaExpr{b'}{b \, b' \, false}}$
  \item $or = \lambdaExpr{b}{\lambdaExpr{b'}{b \, true \, b'}}$
\end{itemize}