\chapter{$\lambda$-calcul non typé}

\section{Syntaxe}

\begin{definition} [Syntaxe du $\lambda$-calcul]
  Soit $V$ un ensemble infini dénombrable. On note $\Lambda$ le plus petit
  ensemble tel que:
  \begin{enumerate}
    \item $V \subseteq \Lambda$
    \item $\forall u, v \in \Lambda, uv \in \Lambda$
    \item $\forall x \in V, \forall u \in \Lambda, \lambda x.u \in \Lambda$
  \end{enumerate}

  Un élément de $\Lambda$ est appelé un \textbf{$\lambda$-terme}.
  Un $\lambda$-terme de la forme \verb|uv| est appelé \textbf{application} car
  l'interprétation donnée est une fonction \verb|u| évaluée en \verb|v|.
  Un $\lambda$-terme de la forme $\lambda$ \verb|x.u| est appelé
\textbf{abstraction}. Il est souvent representé comme la fonction qui envoie
\verb|x| sur \verb|u|.

%Remarquons que nous venons seulement de définir une syntaxe ! Les
%représentations suggérées pour les \lambda-termes uv et \lambda x.u viennent de la
%sémantique le plus souvent utilisée.
\end{definition}

Des exemples de $\lambda$-termes sont
\begin{itemize}
  \item $\lambda x \, (\lambda y \, x y)$
  \item $\lambda x \, (\lambda y \, x (\lambda z \, z z) y)$
  \item $x (\lambda y \, y (\lambda z \, z z))$
\end{itemize}

Parler de la motivation du $\alpha$-renommage. Remarquer qu'il y a une notion de
variables libres et variables liées à définir.

\begin{definition} [Ensemble de variables libres et liées]

\end{definition}

Cela nous amène à la définition suivante:
\begin{definition} [Relation d'$\alpha$-renommage]
  
\end{definition}

Les éléments $\lambda x \lambda y xy$ et $\lambda y \lambda x yx$ appartiennent
à la même classe d'équivalence, ce que nous souhaitons.

On se concentre maintenant uniquement sur les classes d'équivalence.

\section{Sémantique}

A toute syntaxe, nous associons une \textit{sémantique}, c'est-à-dire une interprétation
des termes.

Pour le $\lambda$-calcul non typé, la sémantique que nous allons définir permet
de réduire un $\lambda$-terme vers un autre $\lambda$-terme. Nous parlons de
\textit{$\beta$-réduction}, ou encore de \textit{réécriture}. La
$\beta$-réduction peut se voir comme une relation binaire sur $\Lambda$.

Sémantique opérationnelle.

\subsection{Différentes stratégies de réduction}

Parler des différentes stratégies d'évaluations.

\subsubsection*{Call by value}
\subsubsection*{Call by name}

Parler des méthodes de preuves sur le $\lambda$-calcul : preuves termes par
termes, par la taille du terme.