\chapter*{Conclusion}
\addcontentsline{toc}{chapter}{Conclusion}

Dans les premiers chapitres, nous avons défini les bases théoriques de la
programmation fonctionnelle et des langages typés. Nous sommes partis de
différents calculs relativement simples comme le $\lambda$-calcul non typé et le
$\lambda$-calcul simplement typé pour arriver à un calcul plus compliqué et plus
récent (2016) appelé DOT qui unifie
le comportement des enregistrements et des modules et permet en même temps
de considérer les modules comme des citoyens de première classe. Nous
avons également montré comment DOT pouvait être interprété comme une extension
de Système $F_{<:}$.

Dans le dernier chapitre, nous avons présenté une implémentation en OCaml basée
sur DOT. Nous avons pu remarquer qu'implémenter un langage à partir de règles
théoriques n'est pas évident et cela à cause des différents arbres de
dérivations possibles pour une même question ou encore en raison des arbres de
tailles infinies.
Nous avons également remarqué qu'il était nécessaire de changer certaines règles
d'inférence pour écrire un algorithme, comme pour le cas de la réflexivité,
l'inclusion de (T-SUB) dans les règles d'inférence ou encore l'introduction de
(UN-REC) pour pouvoir comparer des types récursifs.
De plus, nous avons remarqué qu'introduire des types chemins dépendants dans le
langage ne facilite pas l'implémentation. Pour finir, il nous a été nécessaire
d'écrire des algorithmes secondaires comme \verb|best_bound|, non décrits dans
les différents articles discutant de DOT.

Nous avons également vu qu'écrire des programmes dans un calcul théorique comme
DOT n'est pas très pratique et implique de développer un langage de surface.
Dans ce langage de surface, des sucres syntaxiques sont implémentés afin de
pouvoir écrire des termes interdits dans DOT, comme l'application de termes à
une fonction. Cependant, à travers l'implémentation de ces sucres, nous avons
remarqué qu'il manquait certaines règles pour pouvoir écrire certains programmes
usuels comme le binding local d'une variable.

DOT n'est pas le seul calcul dans lequel les modules peuvent être considérés
comme citoyens de première classe. D'autres calculs ont été explorés comme
1ML\cite{1ml-paper}. Ce dernier catégorise les types en
genres\cite{tapl-higher-order-systems} afin d'affiner les règles de typage et
de sous-typage sur les types. DOT a été choisi à la place de 1ML car ce dernier possède
déjà une implémentation et les règles de typage et de sous-typage semblent
encore plus complexes que DOT.

% Discuter qu'il y a d'autres sémantiques intéressantes comme la sémantique
% catégorique, sujet de recherche assez récent et inexploré actuellement dans le
% cas de DOT ?
