\section{Références}

\begin{itemize}
  \item Pierce.
  \item Papiers officiels sur DOT et 1ML.
  \item Liste des différents talks sur DOT.
  \item Les conversations avec l'équipe de DOTTY sur Gitter.
  \item Théorème de Church-Roser. https://www.irif.fr/~saurin/Enseignement/LMFI/16-17/notes-de-cours.pdf
\end{itemize}

\begin{itemize}
  \item Pour des explications sur le contexte de jugement et une interprétation en
  forme d'arbre des arbres de dérivations de typage.
  https://mindsized.org/IMG/pdf/td08.pdf

  \textbf{Ces règles définissent un système d’inférence
similaire au calcul des séquents. Une inférence
de typage est un arbre dont chaque noeud est
une instance de règle, chaque feuille une instance
d’axiome, de telle sorte que chaque arête mette en
relation un même séquent à chaque extrémité.}

  \item 8 décembre 2016 - Conférence From DOT to DOTTY -> https://skillsmatter.com/skillscasts/8866-from-dot-to-dotty
\end{itemize}